\documentclass{book}


% Page size
\usepackage[a5paper]{geometry}


% Font
\usepackage{fontspec}
\setmainfont{Clara}


% Typesetting
\usepackage{microtype}


% Colors
\usepackage[dvipsnames]{xcolor}
\definecolor{darkgray}{gray}{0.35}


% Languages
\usepackage[english, latin]{babel}


% Figures
\usepackage{graphicx}


% Initials
\usepackage{lettrine}


% Columns
\usepackage{paracol}


% Line spacing
\usepackage{setspace}


% Paragraph spacing and indents
\setlength{\parskip}{0.75\baselineskip}
\setlength{\parindent}{0pt}


% Decorative borders
%\usepackage[object=vectorian]{pgfornament}
%\usepackage{tikz}


% Remove page numbers when using cleardoublepage
\usepackage{emptypage}


% Margin notes inside text columns
\usepackage{wrapfig}


% Math
\usepackage{amsmath}


% Put page number at bottom
\pagestyle{plain}


% Counter for english footnotes
\newcounter{engnote}

% Footnotes with their own symbols
\makeatletter
\def\@xfootnote[#1]{%
  \protected@xdef\@thefnmark{#1}%
  \@footnotemark\@footnotetext}
\makeatother

%%%% Note: Footnotes in the english columns should be entered in the latin column using engnotetext, otherwise they may appear on the wrong page.

% Commands for english footnotes
\newcommand{\engnotenum}{\textsuperscript{\arabic{engnote}\stepcounter{engnote}}}
\newcommand{\engnotetext}[1]{\vphantom{\footnotemark{}}\footnotetext{#1}}


% Block headings
\newcommand{\blockhead}[4][]{
\begin{wrapfigure}[#3]{l}{2.75cm}
\centering
\vspace{#4}
\parbox{2.75cm}{\begin{center}\footnotesize \color{BrickRed} \emph{#2}\\ #1 \end{center}}
\end{wrapfigure}
}


\newcommand*\ruleline[2]{\par\noindent\raisebox{.8ex}{\makebox[#2\linewidth]{\hrulefill\hspace{1ex}\raisebox{-.8ex}{#1}\hspace{1ex}\hrulefill}}}


\begin{document}

\pagenumbering{roman}

{\fontsize{11}{11} \selectfont





\begin{titlepage}
\begin{center}

{\fontsize{11}{11} \selectfont
\makebox[.65\textwidth][s]{\textls[110]{\textbf{THE CHRONICLE OF THE}}}

\makebox[.65\textwidth][s]{\textls[50]{\textbf{ABBEY OF BURY ST EDMUNDS}}}

\makebox[.65\textwidth][s]{\textls[70]{\textbf{BY JOCELIN OF BRAKELOND}}}
}

{\fontsize{8}{11} \selectfont
\makebox[.65\textwidth]{
\ruleline{ \textls[80]{\textbf{TO WHICH IS APPENDED}} }{.65}
}
}

{\fontsize{11}{11} \selectfont
\makebox[.61\textwidth][s]{\textls[20]{\textbf{TWO LIVES OF ST EDMUND}, viz.}}
}

\makebox[.57\textwidth][s]{\textls[70]{\textbf{ST ABBO OF FLEURY,} and}}

\makebox[.53\textwidth][s]{\textls[80]{\textbf{\AE{}LFRIC OF EYNSHAM}}},

{\fontsize{8}{11} \selectfont
\makebox[.53\textwidth]{
\ruleline{ \textls[80]{\textbf{IN THEIR ORIGINAL}} }{.5}
}
}

{\fontsize{10}{10} \selectfont
\makebox[.49\textwidth][s]{\textls[20]{\textbf{LATIN AND ANGLO \hspace{-1pt}SAXON}}},
}	

\vspace{-0.1cm}

{\fontsize{8}{8} \selectfont
\makebox[.45\textwidth][s]{\textls[20]{with \textbf{ADJACENT TRANSLATIONS}}},
}

\vspace{-0.15cm}

{\fontsize{8}{8} \selectfont
\makebox[.41\textwidth][s]{\textls[40]{\textbf{EXPLANATORY NOTES,} etc.;}}
}

\vfill


{\fontsize{9}{9} \selectfont
R.\ Creswell,
}

\vspace{-0.15cm}

{\fontsize{11}{11} \selectfont
\textls[200]{\emph{OXFORD:}}
}

\vspace{-0.15cm}

{\fontsize{9}{9} \selectfont
An.do. MMXXI.
}


\end{center}
\end{titlepage}































\begin{titlepage}
\begin{center}
{\bfseries \lsstyle
{\Large CRONICA}

\vspace{0cm}

{\color{BrickRed} \Huge JOCELINI \\ \vspace*{0.5cm} DE BRAKELONDA,}

\vspace{0.4cm}

{\Large DE REBUS GESTIS SAMSONIS}

\vspace{0.2cm}

{\Large ABBATIS MONASTERII}

\vspace{0.4cm}

\includegraphics[scale=0.37]{fig/abbey.png}

\vspace{0.4cm}

{\color{BrickRed} \Huge SANCTI \AE{}DMUNDI.}

\vspace{0.2cm}

}
\end{center}
\end{titlepage}


\thispagestyle{empty}
\begin{center}

{\setlength{\parskip}{3mm}

Cronica de rebus gestis Samsonis abbatis

(Harl.\ MS.\ 1005 ff.\ 127r--170v)

Jocelin of Brakelond

}

\end{center}

\vfill

{\setlength{\parskip}{1mm} \small

\emph{Frontispiece---}

G.\ F.\ Sargent and J.\ C.\ Varrall in \emph{The Book of Shakespeare Gems: In a Series of Landscape Illustrations of the Most Interesting Localities of Shakespeare's Dramas}, Bohn, London, \oldstylenums{1846}.

\vspace{0.4cm}

\emph{Latin text---}

Ed.\ T.\ Arnold: \emph{Memorials of St.\ Edmund's Abbey}, Vol.\ \oldstylenums{1}, Rerum Britannicarum Medii \AE{}vi Scriptores, The Lords Commissioners of Her Majesty's Treasury, Under the Direction of the Master of the Rolls, London, \oldstylenums{1890}.


\vspace{0.4cm}

\emph{English translation, introduction and footnotes---}

Trans.\ ed.\ L.\ C.\ Jane, intr.\ Abbot Gasquet: \emph{The Chronicle of Jocelin of Brakelond, Monk of St.\ Edmundsbury: A Picture of Monastic and Social Life in the XII{\tiny TH} Century}, Chatto \& Windus, London, \oldstylenums{1907}.

}


\newpage

\thispagestyle{empty}
\begin{center}

\hspace{0pt}
\vfill

{\setstretch{1.1}

\parbox{5.5cm}{
\makebox[0pt][r]{\emph{``}}{\emph{Fly, noble English, you are bought and sold;\\
Unthread the rude eye of rebellion\\
And welcome home again discarded faith.\\
Seek out King John and fall before his feet;\\
For if the French be lords of this loud day,\\
He means to recompense the pains you take\\
By cutting off your heads: thus hath he sworn\\
And I with him, and many moe with me,\\
Upon the altar at Saint Edmundsbury;\\
Even on that altar, where we swore to you\\
Dear amity and everlasting love.''}
}
}
}

\vfill
\hspace{0pt}

\end{center}


\cleardoublepage

\begin{center}

\hspace{0cm}\vspace{1.0cm}

\parbox{8cm}{

\begin{center}
\textls[100]{EDITORIAL REMARKS.}
\end{center}

\vspace{.25cm}
{\setstretch{1.05}
The paragraphs, headings, and numbered footnotes are those of L.\ C.\ Jane's translation, as it appeared in the \oldstylenums{1907} edition of Chatto \& Windus. Where not completely redundant, T.\ Arnold's (Rolls ed., \oldstylenums{1890}) marginalia and notes have been reproduced as obelus footnotes in the Latin text. Minor printing errors have been corrected throughout.
}
}

\end{center}


\begin{center}

\hspace{0cm}\vspace{1.0cm}

\parbox{8cm}{

\begin{center}
\textls[100]{NOTE CONCERNING CURRENCY.}
\end{center}

\vspace{.25cm}
{\setstretch{1.05}
The \emph{mark} is an archaic northern European weight unit, typically divided into eight \emph{ounces} (of some specification), and used particularly for the measurement of gold and silver. In Norman England it was a unit of account worth \oldstylenums{160} pennies (\oldstylenums{13}s.\ \oldstylenums{4}d. or two-thirds of a pound), \emph{i.e.}, one mark weight of silver, provided that the pennies were not debased. The English pennies of the late twelfth and thirteenth centuries contained approximately \oldstylenums{1}.\hspace{1pt}\oldstylenums{4} grams of fine silver, and were worth about a half day's wage.
}
}

\end{center}


%\cleardoublepage
%\thispagestyle{empty}
%\includegraphics[scale=1,angle=90]{map/export1.pdf}



\cleardoublepage

\begin{center}


\hspace{0cm}\vspace{1.4cm}

\parbox{8cm}{
{\scshape
A veritable monk of Bury St.\ Edmunds is worth attending to, if by chance made visible and audible. Here he is; and in his hand a magical speculum, much gone to rust, indeed, yet in fragments still clear; wherein the marvellous image of his existence does still shadow itself, though f\vphantom iitfully, and as with an intermittent light.\\
}

\hspace{0pt}\hfill Carlyle --- \emph{Past and Present}.
}

\end{center}

\vspace{1cm}

{\setstretch{1.05}

\lettrine[lines=4]{\color{BrickRed}F}{ew} medi\ae{}val documents have exercised a greater fascination over men's minds in these latter days than ``The Chronicle of Jocelin of Brakelond.'' More than sixty years ago the publication of the Latin text of this history, by the Camden Society, attracted the attention of the great Thomas Carlyle, and furnished him with material for sketching his picture of ``The Ancient Monk,'' which occupied the entire second book of his \emph{Past and Present}. Although the modern sage in his own rugged way affected no little contempt for what he called this ``extremely foreign book,'' and for ``the monk-Latin'' in which it was written, it is evident that Jocelin's simple story of the wise, firm, yet withal gentle rule of a medi\ae{}val abbot over a great English monastery cast a spell over him, the influence of which can be detected in every page of his delightful and almost surprisingly sympathetic account of Abbot Samson and of Edmundsbury.

In this case the \emph{Past}, as Carlyle read it in the ``Chronicle,'' was so entirely different from the \emph{Present}, as he knew it in his day, that the wonder is not that he was fascinated by it, but that he was able with its help to paint so true and living a picture and to fashion so fitting a frame in which to set it. For to him, without doubt, the story dealt with what he regarded as ``vanished existences''---``ideas, life-furniture, whole workings and ways,'' which were not only \emph{Past}, but gone beyond recall, and ``covered deeper than Pompeii with the lava-ashes and inarticulate wreck of seven hundred years!''

And indeed it cannot be denied that the ideals and aspirations, as revealed to us in the history of Abbot Samson and, so far as we know, in the life story of his biographer Jocelin, are of a higher and almost a different order to those of our modern world. To men of their calling in those far-off times, the natural and the supernatural were united and intermingled in the simplest and most ordinary way. Their very notions of the unseen world are almost sufficient to take away the breath of those whose lots have been cast in this more material and prosaic age of doubts and disbeliefs. To Samson, and Jocelin, and their fellow-monks at Edmundsbury in the twelfth century, heaven, as a great writer has said of earlier English monasticism, was hardly even ``next door.'' The future life was merely the present continued, and each man went forth to his task as it came and laboured at it day by day, not with any idea of finishing it, but only of carrying on for the span of his allotted existence. They built, and planted, and wrote till the end came, and then they went to heaven and others stepped into their places and took up the common work. It was indeed a ``simple life:'' it was almost Arcadian in its picturesque simplicity, and, as Cardinal Newman says of the same life in the days of our Venerable Bede, it reminds us of those times in the dayspring of the world, when Adam delved and Abel watched the flocks, and Noah tended his vines, and angels visited them.

This living belief in the nearness and all-importance of the supernatural is the key-note of Jocelin's charming story of a few brief years in the long history of an old English abbey, a new translation of which is here given to the public. As a story, however, Brakelond's ``Chronicle'' is not wholly, nor indeed mostly, either mysterious or incredible: there are troubles, and trials, and difficulties enough recounted by the writer; and at every turn we may see evidences of human nature and even of human struggles and passions, which are sufficient, and as some may perhaps think, more than sufficient, to show us that it is a history of men, and not of angels, which the old monk is setting forth so naturally and so truthfully. At any rate, there is quite sufficient of the human element in the narrative to give most of us a human interest in the story.

And this itself is proof that Jocelin is a true chronicler of what really took place, and no mere romancer tempted to edit or suppress entirely what might not be unto ``edification.'' He manifests no desire to make himself or his brethren appear other than what they were in reality---that is, thorough Englishmen, with strong wills and human passions, which, though these same passions might occasionally appear to gain the mastery, they were at all times endeavouring to subdue unto God's service by the help of His grace and through the broad-minded provisions of St.\ Benedict's Rule. The actors who appear in this living drama, though they are for the most part monks, are obviously men, natural and human enough in all their works and words; but these men are at the same time also monks, endeavouring to raise their minds and hearts to supernatural ideals, and striving to attain to that personal communion with God which is the aim and object of all true religion and of all religious observance and practice. This is ``another world truly,'' writes Carlyle, ``and this present poor distressed world might get some profit by looking wisely into it, instead of foolishly. But at lowest, O dilettante friend, let us know always that it \emph{was} a world, and not a void infinite of grey haze with phantasms swimming in it. These old St.\ Edmundsbury walls, I say, were not peopled with phantasms, but with men of flesh and blood, made altogether as we are. Had thou and I then been, who knows but we ourselves had taken refuge from an evil Time and fled to dwell here, and meditate on an Eternity, in such fashion as we could? Alas, how like an old osseous fragment, a broken blackened shinbone of the old dead Ages, this black ruin looks out, not yet covered by the soil; still indicating what a once gigantic Life lies buried there! It is dead now, and dumb; but was alive once and spake. For twenty generations, here was the earthly arena where painful living men worked out their life-wrestle,---looked at by Earth, by Heaven and Hell. Bells tolled to prayers; and men of many humours, various thoughts, chanted Vespers and Matins;---and round the little islet of their life rolled for ever (as round ours still rolls, though we are blind and deaf) the illimitable Ocean, tinting all things with \emph{its} eternal hues and reflexes, making strange prophetic music! How silent now!''

\textbf{The Author.}---Jocelin de Brakelond, the writer of the Chronicle called by his name, was a monk of Edmundsbury. The date of his birth is uncertain, but as he became a novice in that abbey in \oldstylenums{1173}, we may suppose that he was born not later than \oldstylenums{1156}. It has been conjectured that he was a native of Bury St.\ Edmunds, and that his name Brakelond was derived from that of an ancient street of the city, in accordance with the common practice of calling monks by the name of the place from which they came to religion. Little more is known about him than he tells us incidentally in the course of his narrative, but one of his contemporaries in the monastery speaks of him as ``a man of excellent religious observance, as well as a power both in word and work''---\emph{eximiae religionis, potens sermone et opere}. Carlyle sees him in his writing as a man of a ``patient, peaceable, loving, clear-smiling nature.'' A ``wise simplicity,'' he adds, ``is in him; much natural sense; a veracity that goes deeper than words.'' What more can we desire in a writer, especially when we may add that he shows himself to have been a cultured man, acquainted with the ancient authors, quoting Virgil and Horace and Ovid? His knowledge of the Bible is naturally extensive, and, as was common in those days, his very phraseology is obviously founded upon the sacred text. He once likewise cites, with acknowledgment, a short passage from the more modern Ralph de Diceto's \emph{Imagines Historiarum}. Our latter-day philosopher praises him also because he shows himself to have ``a pleasant wit; and to love a timely joke, though in a mild subdued manner; very amiable to see.''

In \textsc{a.d}.\ \oldstylenums{1173}, as just noted, Jocelin entered the community and passed under the care of Samson of Tottington, who subsequently became abbot, but who was then Master of novices. The then abbot, Hugh, was old, and although a high standard of the religious exercises and of the monastic life inside the cloister was maintained, the temporalities were in a sad state, and year by year tended to get from bad to worse, so that Jocelin's early experiences of monastic life were connected with anxieties about the load of debt to money-lenders under which Edmundsbury groaned.

He tells us that he had himself seen bonds for repayment made out to the Jews, under which, for failure to meet the sums falling due, the original loan had grown in eight years from £\oldstylenums{100} to £\oldstylenums{800}. No wonder that the youthful religious questioned his Master of novices as to why some remedy was not found by those in authority for a state of things which meant temporal ruin and disgrace for the community of Edmundsbury.

In \oldstylenums{1180} Abbot Hugh met with an accident and died. After a period of a year and three months the former Master of novices, Samson, then the provident Sacrist, was chosen in his place. It was during this period of vacancy that, in recording something which happened in the monastery, Jocelin incidentally makes mention of another literary work of his own, namely, the \emph{Book of the Miracles of St.\ Robert}, a boy supposed to have been martyred by the Jews in \oldstylenums{1181}, who was entombed in the church at Edmundsbury.

On the election of Samson, Jocelin was appointed his chaplain, and this brought him into the closest connection with the abbot for six years. In \oldstylenums{1198} and \oldstylenums{1200} he was Guest-master, and in \oldstylenums{1212} he held the office of Almoner. In all these offices the future chronicler had exceptional means of acquiring information, and these he utilised in writing the story of Abbot Samson's administration, which is introduced by a vivid sketch of the temporal disorder of the house in the closing years of Abbot Hugh. His Chronicle covers the period of the history of Edmundsbury from \oldstylenums{1173} to \oldstylenums{1190}, and, as he says in the beginning, ``he took care to write only what he himself saw and heard.'' The date of his death is uncertain.

\textbf{The ``Chronicle.''}---The Latin text of \emph{Cronica Joceline} is found complete only in one manuscript---Harl.\ MS.\ \oldstylenums{1005}---in the British Museum. It was printed for the first time by the Camden Society in \oldstylenums{1840} under the editorship of I.\ G.\ Gage Rokewood, who supplied a valuable Introduction and notes, of which subsequent editors have availed themselves. The text was likewise printed in Mr.\ Thomas Arnold's \emph{Memorials of St. Edmund's Abbey} (Rolls Series) I., pp. \oldstylenums{209}--\oldstylenums{336}.

In \oldstylenums{1844}, under the title \emph{Monastic and Social Life in the Twelfth Century, as exemplified in the Chronicle of Jocelin of Brakelond, A.D. \oldstylenums{1173}--\oldstylenums{1202}}, the work was translated by Thomas Edlyne Tomlins. Carlyle's work, \emph{Past and Present}, published in \oldstylenums{1843} had already drawn attention to the ``Chronicle of Jocelin,'' and another edition of Mr. Tomlins' work was called for in \oldstylenums{1849}. This translation has since appeared at least once, but for the present edition a new English version has been carefully prepared from the original Latin text of the Chronicle.

\textbf{Abbot Samson.}---The central figure and, as we may say, ``the hero'' of Jocelin's story is, of course, Abbot Samson. He was born in \oldstylenums{1135} at Tottington, near Thetford, in Norfolk. His father appears to have died when Samson was young, and a pretty legend of a boyish dream in which St.\ Edmund extended his protection to the child against the assaults of the devil, and the recognition of the place seen in the dream as the gate of the monastery of St.\ Edmundsbury, when his mother had taken him with her on a pilgrimage to the shrine of the saint, led to his taking refuge in the cloister. He had received his early instruction from a schoolmaster named William of Diss, and he attained the degree of Master of Arts in the University of Paris. In this place we are not concerned with the events of his life: these may be read for the most part in the Chronicle of Jocelin of Brakelond. What alone seems to be called for in this brief Introduction is some account of his person and character as it is manifested in the scattered evidences of his acts.

If we want a picture of the man let us take Carlyle's, who sketches ``the substantial figure of a man with eminent nose, bushy brows, and clear-flashing eyes, his russet beard growing daily greyer,'' and his hair which, before his elevation to the abbot's chair, had been black, becoming daily more and more silvered with his many cares. We know something of the task that was before him when he gathered up the reins of office, and we may be sure he knew more. But as we see him in the pages of Jocelin, he was not the man to flinch from his duty, or to seek to let difficulties mend themselves by pretending that he did not see them. From the time that he walked barefooted into his church to be installed in the abbatial chair, he let all see that he was abbot and had come to rule. He had set his whole strength to accomplish a great task and his shoulders to sustain an almost overwhelming burden, when in the hour of his election he walked to the altar singing the \emph{Miserere mei} with his brethren. ``His head was held erect,'' says the faithful Jocelin, ``and his face showed no change,'' a portent which called from the king the remark: ``This abbot-elect seems to think himself capable of governing an abbey.''

``It is beautiful''---writes Carlyle in a philosophical appreciation of the principles of monastic government---``it is beautiful how the chrysalis governing-soul, shaking off its dusty slough and prison, starts forth winged, a true royal soul! Our new abbot has a right honest, unconscious feeling, without insolence as without fear or flutter, of what he is and what others are. A courage to quell the proudest, an honest pity to encourage the humblest. Withal there is a noble reticence in this Lord Abbot: much vain unreason he hears; lays up without response. He is not there to expect reason and nobleness of others, he is there to give them of his own reason and nobleness. Is he not their servant, who can suffer from them and for them; bear the burden their poor spindle-limbs totter and stagger under; and in virtue of \emph{being} their servant, govern them, lead them out of weakness to strength, out of defeat into victory?''

Abbot Samson ruled over his house for thirty years, and when in \oldstylenums{1212}, ten years after the end of Jocelin's Chronicle, he died, he was followed to the grave by a sorrowing community whose unstinted reverence and affection he had won. An unknown monk of Edmundsbury, the author of another Chronicle of the house, thus wrote of him: ``On the \oldstylenums{30}th December, at St.\ Edmund's, died Samson, of pious memory, the venerable abbot of that place; after he had prosperously ruled the abbey committed to him for thirty years and had freed it from a load of debt, had enriched it with privileges, liberties, possessions and spacious buildings and had restored the worship of the church both internally and externally, in the most ample manner. Then bidding his last farewell to his sons, by whom the blessed man deserved to be blest for evermore, whilst they were all standing by and gazing with awe at a death which was a cause for admiration, not for regret, in the fourth year of the interdict he rested in peace."

The first business to which Abbot Samson applied himself after his election was the task of understanding and grappling with the deplorable financial state of his house. He insisted upon the immediate production of every claim against the monastery, and by personally visiting each of its many manors he gained a correct knowledge of its resources. Within twelve months he had formed his plans and had quieted every creditor: within twelve years the entire debt had been paid off, and he could turn his attention to building and adorning the house of St.\ Edmund. It is impossible to read the pages of Jocelin without seeing that the ruling idea of the abbot's life was his devotion to his great patron, St.\ Edmund. He was the servant, after God, of the saint, his representative and the upholder of his honour and privileges, the champion of his rights, the guardian of his property. Inspired by this thought he worked to make Edmundsbury worthy of its patron, and in his success he saw the result of the saint's intercession and protection.

``Apart from this special devotion to St.\ Edmund, it is easy to see,'' writes Mr.\ Thomas Arnold, ``that Samson was an earnestly religious man, and not a Christian by halves. After the news had come of the capture of Jerusalem by the Saracens, Samson took the loss of the Holy Places so much to heart, that from that time he wore undergarments of hair-cloth and abstained from the use of meat.''

He was, too, a thorough Englishman, and read admirably---\emph{elegantissime}---the Bible in English---\emph{scripturam anglice scriptam}---and ``he was wont to preach to the people in English---but in the dialect of Norfolk where he had been born and bred.'' On one occasion he gives as a reason, and as some may think, a somewhat strange reason, for appointing a monk to an office, that ``he did not know French.'' He was no doubt anxious to secure that St.\ Edmundsbury should be truly national, with its roots deep in the soil of his country, to teach it to build up its own traditions, and to let people see that it was a great \emph{English} house.

But Samson's work was not accomplished without grave anxiety, none the less because it was unseen by others. Though he walked upright with a smiling face, and had ever the courage to battle for the rights of his house when there was need, in a way that might make people regard him as a man of iron nerve possessed of a soul that never felt any trouble, nevertheless in the first fourteen years of his administration his black hair was blanched as white as snow, and Jocelin speaks of hearing his beloved master walking about when all were in bed and relieving his pent-up feelings with sighs and groans. Once the chronicler took courage to tell his master that he had thus heard him in his night vigil, and to this the abbot replied: ``'Tis no wonder: you (as my chaplain) share in the streets of my office, in the meat and drink, in the journeys and the like, but you little think what I have to do to provide for my house and family, or of the many and difficult matters of my pastoral office, which are always pressing upon me: these are the things which make my soul anxious and cause me to sigh.''

And so when Abbot Samson came to die, the thin veil which to him and his monks of Edmundsbury alone hid the world to come from their vision was parted, and the supernatural life eternal was revealed to him in the most natural of ways. He passed from labour for God and St.\ Edmundsbury, to rest in God and with his loved patron, carrying with him the full sheaves of his good works. Carlyle has only partially caught the idea when he writes: ``Genuine work alone, what thou workest faithfully, that is eternal.'' ``Yes,'' he concludes, ``a noble Abbot Samson resigns himself to oblivion; feels it no hardship, but a comfort; counts it as a still resting-place, for much sick fret, and fever, and stupidity, which in the night-watches often made his strong heart sigh.''
}
}


\begin{flushright}
\parbox{6cm}{
\begin{center}
\textsc{Francis Aidan Gasquet},\\
\vspace{0.1cm}
\emph{Abbot-President of the English Benedictines}.
\end{center}
}
\end{flushright}

\cleardoublepage
\footnotelayout{m}
\emergencystretch 1em
\pagenumbering{arabic}




\begin{paracol}{2}

\lettrine[lines=4]{\color{BrickRed}Q}{uod} \begin{otherlanguage}{latin}vidi et audivi scribere curavi, qu\ae{}dam mala interserens ad cautelam, qu\ae{}dam bona ad usum,\linebreak qu\ae{}~\hspace{1.6cm}contigerunt in ecclesia sancti \AE{}dmundi in diebus nostris, ab anno quo Flandrenses capti sunt extra villam, quo habitum religionis suscepi, quo anno Hugo prior depositus est, et R.\ prior substitutus.
\end{otherlanguage}

\switchcolumn

\lettrine[lines=4]{\color{BrickRed}I}{ have} undertaken to write of those things which I have seen and heard, and which have occurred in the church of Saint Edmund, from the year in which the Flemings were taken\footnote{The allusion is to the battle of Fornham, November, \oldstylenums{1173}. In this year the quarrel between Henry II. and his sons, culminated in a general rising both in Normandy and in England. Of the leaders of the rebellion in England, Robert de Bellemont, earl of Leicester, was the chief. Having gathered a force of mercenaries in the Low Countries, he landed at Walton, which he failed to take. After joining hands with Hugh Bigod, earl of Norfolk, at Framingham, and capturing Haughley, he attempted to force his way to his own estates. Meanwhile, the justiciar Richard de Lucy and Humphrey Bohun hastened south from their campaign against the Scots, and having been reinforced by the local levies, they succeeded in intercepting Leicester at Fornham St.\ Geneveve, on the river Lark, four miles south from Bury St.\ Edmund's. The rebels were easily defeated, and Leicester taken prisoner; of his mercenaries only a few escaped. An account of the battle, not very accurate, from the point of view of the St.\ Edmund's monks, is to be found in Appendix E of the first volume of the \emph{Memorials of St.\ Edmund's Abbey}, (Rolls Series). The escape of those mercenaries who did escape is attributed to the intervention of St.\ Edmund and St.\ Thomas.} without the town, in which year also I assumed the religious habit, and in which Prior Hugh was deposed and Robert made Prior in his room. And I have related the evil as a warning, and the good for an example.

%\switchcolumn*

\end{paracol}
\begin{paracol}{2}

\begin{otherlanguage}{latin}
\blockhead[\textsc{a.d}.\ \oldstylenums{1173}.]{How Abbot Hugh ruled the Church of St.\ Edmund.}{4}{-0.45cm}
Tunc temporis senuit Hugo abbas, et aliquantulum caligaverunt oculi ejus; homo pius et benignus, monachus religiosus et bonus, sed nec bonus nec providus in s\ae{}cularibus exercitiis: qui nimis confidebat suis et nimis eis credebat, de alieno potius quam de proprio pendens consilio. Ordo quidem et religio fervebant in claustro, et ea qu\ae{} ad ordinem spectant; sed exteriora male tractabantur, dum quisque, serviens sub domino simplice et jam senescente, fecit quod voluit, non quod decuit. Dabantur vill\ae{} abbatis et omnes hundredi ad firmam; nemora destruebantur; domus maneriorum minabantur ruinam; omnia de die in diem in deteriorem statum vertebantur. Unicum erat refugium et consolationis remedium abbati, denarios appruntare; ut saltem sic honorem domus su\ae{} posset sustentare. Non erat terminus Pasch\ae{} nec sancti Michaelis octo annis ante obitum ejus, quin centum libr\ae{} vel ducent\ae{} ad minus crescerent in debitum; semper renovabantur cart\ae{}, et usura qu\ae{} excrevit vertebatur in katallum.

\end{otherlanguage}

\switchcolumn

In those days Abbot Hugh\footnote{Hugh, prior of Westminster, was elected abbot in \oldstylenums{1157}, in succession to abbot Ording. According to Gervase (I., 163), he received his benediction from Theobald, archbishop of Canterbury, to whom he made profession of canonical obedience. According to the \emph{Chronica Buriensis} (\emph{Mem}.\ III., \oldstylenums{6}) he was confirmed by the bishop of Winchester. In any case, abbot Hugh, as it related in the text of the \emph{Chronicle of Jocelin} (p.\ \oldstylenums{6}), was freed from all obedience by pope Alexander III.} grew old, and his eyes were dim.\footnote{Gen.\ xxvii., \oldstylenums{1}.} He was a good and kindly man, a godfearing and pious monk, but in temporal matters he was unskilful and improvident. He relied too much on his own intimates and believed too readily in them, rather trusting to a stranger's advice than using his own judgment. It is true that discipline and the service of God, and all that pertained to the rule, flourished greatly within the cloister, but without the walls all-things were mismanaged. For every man, seeing that he served a simple and ageing lord, did not that which was right, but that which was pleasing in his own eyes. The townships and all the hundreds of the abbot were given to firm; the woods were destroyed, and the houses on the manors were on the verge of ruin; from day to day all things grew worse. The abbot's sole resource and means of relief was in borrowing money, that so it might at least be possible to maintain the dignity of his house. For eight years before his death, there was never an Easter or Michaelmas which did not see at least one or two hundred pounds added to the debt. The bonds were ever renewed, and the growing interest was converted into principal.

\switchcolumn*

\begin{otherlanguage}{latin}
Descendebat h\ae{}c infirmitas a capite in membra, a pr\ae{}lato in subjectos. Unde contigit quod quilibet obedientiarius haberet sigillum proprium, et debito se obligaret tam Judeis quam Christianis pro voluntate sua. S\ae{}pe capp\ae{} seric\ae{}, et ampull\ae{} aure\ae{}, et alia ornamenta ecclesi\ae{} impignorabantur, inconsulto conventu. Vidi cartam fieri Willelmo filio Isabel mille librarum et xl.; sed nec causam nec originem scivi. Vidi et aliam cartam fieri Isaac, filio Raby Joce, cccc.\ librarum, sed nescio quare. Vidi et tertiam cartam fieri Benedicto Judeo de Norwico, octies c.\ librarum et quater viginti; et h\ae{}c fuit origo et causa hujus debiti.
\end{otherlanguage}

\switchcolumn

This disease spread from the head to the members, from the ruler to his subjects. So it came to pass that if any official had a seal of his own, he also bound himself in debt as he listed, both to Jews and Christians. Silken caps, and golden vessels, and the other ornaments of the church, were often placed in pledge\footnote{This was illegal. Rokewode (\emph{Chron.\ Joc}., pp.\ \oldstylenums{106}--\oldstylenums{7}) gives instances of fines inflicted on Jews for taking church property in pawn, from the Pipe Rolls of Norfolk and Suffolk.} without the assent of the monastery. I have seen a bond made to William Fitzlsabel for a thousand and two score pounds, but know not the why nor wherefore. And I have seen another bond to Isaac, son of Rabbi Joce, for four hundred pounds, but know not wherefore it was made. I have seen also a third bond to Benedict, the Jew of Norwich, for eight hundred and fourscore pounds, and this was the origin and cause of that debt.

\switchcolumn*

\begin{otherlanguage}{latin}
Destructa fuit camera nostra, et recepit eam Willelmus\engnotetext{William Wardell (\emph{Mem}.\ II., \oldstylenums{291}). His incompetence is mentioned in the \emph{Gesta Sacristurum (ibid.)}, and is described in the text.} sacrista volens vel nolens, ut eam instauraret; et occulte appruntavit a Benedicto Judeo xl.\ marcas ad usuram, et ei fecit cartam signatam quodam sigillo quod solebat pendere ad feretrum sancti \AE{}dmundi, unde gild\ae{} et fraternationes solebant sigillari, quod postea sed tarde fractum est, jubente conventu. Cum autem crevisset debitum illud usque ad c.\ libras, venit Judeus portans literas domini regis\engnotetext{The Jews were legally the king's chattels, and debts due to them were due to the king. Accordingly, when debtors failed to pay, the Jews were able to invoke the royal authority to enforce payment.} de debito sacrist\ae{}; et tunc demum patuit quod latuit abbatem et conventum. Iratus autem abbas voluit deponere sacristam, pr\ae{}tendens privilegium domini pap\ae{}, ut posset deponere Willelmum sacristam suum, quando vellet. Venit autem aliquis ad abbatem, et, loquens pro sacrista, ita circumvenit abbatem, quod passus est cartam fieri Benedicto Judeo cccc.\ librarum, reddendarum in fine iiij$^\text{or}$ annorum, scilicet pro c.\ libris qu\ae{} jam excreverant in usuram, et aliis c.\ libris quas idem Judeus commodavit sacrist\ae{} ad opus abbatis. Et sacrista suscepit omne debitum illud reddendum in pleno capitulo, et facta est carta sigillo conventus signata, abbate dissimulante et sigillum suum non apponente, tanquam illud debitum non pertineret ad illum. In fine vero quatuor annorum non erat unde illud debitum posset reddi; et facta est nova carta octies c.\ librarum et quater viginti librarum, reddendarum ad terminos statutos, annis singulis quater xx.\ librarum.
\end{otherlanguage}

\switchcolumn

\setcounter{engnote}{5}

Our buttery was destroyed, and the sacristan William\engnotenum{} received it to restore whether he would or no. He secretly borrowed forty marks at interest from Benedict the Jew, and made him a bond, scaled with a certain seal which was wont to hang at the shrine of St.\ Edmund. With this the gilds and brotherhoods used to be sealed; afterwards, but in no great haste, it was destroyed by order of the monastery. Now when that debt increased to one hundred pounds, the Jew came, bearing letters of the lord king\engnotenum{} concerning the sacristan's debt, and then at last that which had been hidden from the abbot and the monks appeared. So the abbot in anger would have deposed the sacristan, alleging a privilege of the lord pope that enabled him to remove William his sacristan when he would. However, there came one to the abbot, who pleaded for the sacristan, and so won over the abbot that he suffered a bond to be made to Benedict the Jew for four hundred pounds, payable at the end of four years, that is, a bond for the hundred pounds to which the interest had increased, and for another hundred pounds which the same Jew had lent to the sacristan for the use of the abbot. And in full chapter the sacristan obtained that all this debt should be paid, and a bond was made and sealed with the seal of the monastery. For the abbot pretended that the debt was no concern of his, and did not affix his seal. However, at the end of the four years there was nothing wherewith the debt might be discharged, and a new bond was made for eight hundred and fourscore pounds, which was to be repaid at stated times, every year fourscore pounds.

\switchcolumn*

\begin{otherlanguage}{latin}
Habuit et idem Judeus plures alias cartas de minoribus debitis, et aliquam cartam qu\ae{} erat xiiij.\ annorum, ita quod summa debiti illius Judei erat mille et cc.\ librarum, pr\ae{}ter usuram qu\ae{} excreverat.
\end{otherlanguage}

\switchcolumn

And the same Jew had many other bonds for smaller debts, and one bond which was for fourteen years, so that the sum of the debt owing to that Jew was a thousand and two hundred pounds, over and above the amount by which usury had increased it.

\switchcolumn*

\begin{otherlanguage}{latin}
Veniensque R.\ elemosinarius domini regis significavit domino abbati rumorem talem venisse ad regem de tantis debitis. Et inito consilio cum priore et paucis aliis, ductus est elemosinarius in capitulum; nobisque assidentibus et tacentibus, dixit abbas: ``Ecce elemosinarius regis, dominus et amicus noster et vester, qui, ductus amore Dei et sancti \AE{}dmundi, nobis ostendit dominum regem quoddam sinistrum audisse de nobis et vobis, et res ecclesi\ae{} male tractari et interius et exterius. Et ideo volo et pr\ae{}cipio in vi obedienti\ae{}, ut dicatis et cognoscatis palam qualiter res se habeant.'' Surgens ergo prior et loquens, quasi unus pro omnibus, dixit ecclesiam in bono statu esse, et ordinem bene et religiose observari interius, et exteriora bene et discrete tractari, debito tamen aliquantulo obligatos nos esse sicut c\ae{}teros vicinos nostros, nec esse aliquod debitum quod nos graveret. Audiens hoc, elemosinarius dixit se valde l\ae{}tum esse ex hoc quod audierat testimonium conventus, id est, prioris sic loquentis. H\ae{}c eadem verba respondit prior alia vice, et magister Galfridus de Constantino, loquentes et excusantes abbatem, quando Ricardus archiepiscopus jure legati\ae{} venit in capitulum nostrum, antequam talem exemptionem haberemus sicut nunc habemus.
\end{otherlanguage}

\switchcolumn

Then came the almoner of the lord king and told the lord abbot that many rumours concerning these great debts had come to the king. And when counsel had been taken with the prior and a few others, the almoner was brought into the chapter. Then, when we were seated and were silent, the abbot said: ``Behold the almoner of the king, our lord and friend and yours, who, moved by love of God and Saint Edmund, has shown to us that the lord king has heard some evil report of us and you, and that the affairs of the church are ill-­managed within and without the walls. And therefore I will, and command you upon your vow of obedience, that you say and make known openly how our affairs stand.'' So the prior arose, and speaking as it were one for all, said that the church was in good order, and that the rule was well and strictly kept within, and matters outside the walls carefully and discreetly managed; and that though we, like others round us, were slightly involved in debt, there was no debt which might give us cause for anxiety. When he heard this, the almoner said that he rejoiced greatly to hear this witness of the monastery, by which he meant these words of the prior. And the prior, and Master Geoffrey of Coutances, answered in these same words on another occasion, when they spoke in defence of the abbot at the time when Archbishop Richard, by virtue of his legatine power, came into our chapter, in the days before we possessed that exemption which we now enjoy.

\switchcolumn*

\begin{otherlanguage}{latin}
Ego vero tunc temporis novicius, data opportunitate, magistrum meum super his conveni, qui me docebat ordinem et cujus custodi\ae{} deputatus fui, scilicet, magistrum Sampsonem, postea abbatem. ``Quid est,'' inquam, ``quod audio? Utquid taces qui talia vides et audis, tu qui claustralis es, nec obedientias cupis, et Deum times magis quam hominem?'' At ille respondens, ait: ``Fili mi, puer noviter combustus timet ignem; ita est de me et pluribus aliis. Hugo prior noviter depositus est de prioratu suo, et in exilium missus; Dionisius et H.\ et R.\ de Hingham de exilio nuper domum redierunt. Ego similiter incarceratus\engnotetext{Arnold (\textit{Mem}., I., xliv., note \oldstylenums{3}; \oldstylenums{212}), gives reasons for supposing that this alludes to a second imprisonment of Samson, distinct from that which he suffered on his return from Rome (see text, p.\ \oldstylenums{77}). The passage would appear to refer to a recent event, which the imprisonment after his Roman journey was not.} fui, et postea apud Acram missus, quia locuti sumus pro communi bono ecclesi\ae{} nostr\ae{} contra voluntatem abbatis. H\ae{}c est hora tenebrarum; h\ae{}c est hora qua adulatores dominantur et eis creditur: confortata est potentia eorum, nec possumus ad eam. Dissimulanda sunt ista pro tempore: videat Dominus et judicet.''\engnotetext{In all probability this means Castle Acre, where was a famous Cluniac priory, founded by William de Warenne, as a cell to St.\ Pancras, Lewes. Acre, however, might mean either Castle Acre of West Acre, at both of which places were priories.}\engnotetext{Ps.\ cxxxviii., \oldstylenums{6} (Vulgate).}\engnotetext{Ex.\ v., \oldstylenums{21}.}
\end{otherlanguage}

\switchcolumn

Now I was then in my novitiate, and on a convenient occasion talked of these things to my master, who was teaching me the Rule, and in whose care I was placed; he was Master Samson, who was afterwards abbot. ``What is this,'' I said, ``that I hear? And why do you keep silence when you see and hear such things­­ you, who are a cloistered monk, and desire not offices, and fear God rather than man?'' But he answered and said, ``My son, the newly burnt child feareth the fire, and so is it with me and with many another. Prior Hugh has been lately deposed and sent into exile; Dennis, and Hugo, and Roger de Hingham have but lately returned to the house from exile. I was in like manner imprisoned,\engnotenum{} and afterwards was sent to Acre,\engnotenum{} for that we spoke to the common good of our church against the will of the abbot. This is the hour of darkness; this is the hour in the which flatterers triumph and are believed; their might is increased,\engnotenum{} nor can we prevail against them. These things must be endured for a while; the Lord see and judge!''\engnotenum{}

\switchcolumn*

\begin{otherlanguage}{latin}
\blockhead[\textsc{a.d}.\ \oldstylenums{1176}.]{How the monastery was freed from legatine visitation.}{4}{-0.65cm}
Venit rumor ad abbatem H.\ quod R.\ archiepiscopus Cantuariensis\engnotetext{Richard of Dover, a Norman, prior of Dover, was archbishop from \oldstylenums{1173} to \oldstylenums{1184}. He was elected at the end of the three years vacancy which followed on the murder of Becket (Gervase, I., \oldstylenums{244}). As to the question of the legatine authority over the abbey, Rokewode (p.\ \oldstylenums{107}--\oldstylenums{8}) collects details. He points out that abbot Hugh appears to have obtained first a special exemption from Alexander III.\ from all authority other than that of the pope or a legate \emph{a latere}; and afterwards a further exemption from the authority of archbishop Richard.} vellet venire ad scrutinium faciendum in ecclesia nostra, auctoritate legati\ae{} su\ae{}; et, accepto consilio, misit abbas Romam et impetravit exemptionem a potestate pr\ae{}dicti legati. Redeunte nuntio ad nos de Roma, non erat unde solvi poterat quod ipse promiserat domino pap\ae{} et cardinalibus, nisi, ex circumstantiis, crux qu\ae{} erat super magnum altare, et Mariola, et Johannes, quas imagines Stigandus archiepiscopus magno pondere auri et argenti ornaverat, et sancto \AE{}dmundo dederat. Dixerunt etiam quidam ex nostris qui abbatem familiarius diligebant, quod ipsum feretrum sancti \AE{}dmundi deberet excrustari propter talem libertatem, non advertentes magnum periculum posse nasci de tali libertate; quod, si forte fuerit aliquis abbas noster qui res ecclesi\ae{} voluerit dilapidare et conventum suum male tractare, non erit persona cui conventus possit conqueri de injuriis abbatis, qui nec episcopum, nec archiepiscopum, nec legatum timebit, et impunitas ausum pr\ae{}bebit delinquendi.

\end{otherlanguage}

\switchcolumn

There came a rumour to Abbot Hugh that Richard, Archbishop of Canter-bury,\engnotenum{} purposed to come and to hold a visitation of our church by virtue of his legatine authority. And having taken advice, the abbot sent to Rome and obtained exemption from the power of the said legate. But when the messenger returned to us from Rome, there was not found means of paying that which he had promised to the lord pope and to the cardinals, unless in the circumstances use might be made of the cross which was above the high altar, and of a Mary, and a John, which images Archbishop Stigand had adorned with much weight of gold and silver, and had given to the blessed Edmund. Then some among our number, who were very intimate with the abbot, said that the very shrine of Saint Edmund itself ought to be stripped in order to win so notable a privilege. But they considered not the great danger that might ensue from so great liberty. For if by chance we should have an abbot who wished to waste the goods of the church and evilly entreat his monastery, then there would be no one to whom the monastery might make complaint of the evil deeds of the abbot, who would fear neither bishop, nor archbishop, nor legate, and whose impunity would give him boldness in wrongdoing.

\switchcolumn*

\begin{otherlanguage}{latin}
\blockhead{Concerning Master Dennis the cellarer.}{3}{-0.55cm}
In diebus illis celerarius, sicut ceteri officiales, appruntavit denarios a Jurneto Judeo, inconsulto conventu, super cartam supradicto sigillo signatam. Cum autem excrevit\footnote[\textdagger]{Excrevisset.} debitum usque ad sexaginta libras, summonitus est conventus ad solvendum debitum celerarii. Depositus est celerarius; licet allegaret gravamen suum, dicens quod susceperat tribus annis hospites omnes in domo hospitum ad pr\ae{}ceptum abbatis sive abbas fuerit pr\ae{}sens sive absens, quos debeat suscipere abbas secundum consuetudinem abbati\ae{}.

\end{otherlanguage}

\switchcolumn

Now in those days the cellarer, like the rest of the officers of the monastery, borrowed money from Jurnet the Jew, without the knowledge of the monastery, on a bond sealed with the seal mentioned above. But when the debt had grown to three score pounds, the monastery was called upon to discharge the debt of the cellarer. He was deposed, though he defended himself by saying that for three years he, by command of the abbot, had received all guests in the guest­house, whether the abbot were at home or no, whom the abbot ought to have received according to the constitution of the house.

\switchcolumn*

\begin{otherlanguage}{latin}
Substitutus est magister Dionisius, qui per providentiam suam et cautelam minoravit debitum lx.\ librarum usque ad xxx.\ libras; de quo debito reddidimus xxx$^\text{ta}$ marcas, quas Benedictus de Blakeham dedit conventui pro maneriis Neutone et Wepstede tenendis: sed carta Judei usque hodie remansit apud Judeum, in qua continentur xxvi.\ libr\ae{} de katallo et de debito celerarii.
\end{otherlanguage}

\switchcolumn

In his stead Master Dennis was appointed, and by his economy and care reduced that debt of sixty pounds to thirty. Towards the extinction of that debt we paid the thirty marks which Benedict de Blakeham gave to the monastery for the manors of Newton and Whepstead. But the Jew's bond remains with the Jew to this day, and in it twenty-six pounds are written down as principal and for the debt of the cellarer.

\switchcolumn*

\begin{otherlanguage}{latin}
Tertio die postquam magister Dionisius fuit celerarius, ducti sunt tres mihtes cum armigeris suis usque in domum hospitum, ut ibi reficerentur, abbate domi existente et in thalamo suo residente. Quod cum audisset magnanimus ille \AE{}acides, nolens pendere in bailiva sua,\footnote[\textdagger]{The owner of a bailiwick makes other people pay toll who enter it, but does not pay himself.} sicut ceteri, surrexit et accepit claves cellarii, et ducens secum milites illos usque in aulam abbatis, veniensque ad abbatem, dixit: ``Domine, bene novistis quod consuetudo abbati\ae{} est, ut milites et laici recipiantur in curia vestra, si abbas domi fuerit; nec volo nec possum recipere hospites qui ad vos pertinent. Alioquin, accipite claves cellarii vestri, et alium constituite celerarium pro beneplacito vestro.'' Audiens hoc abbas, volens vel nolens recepit illos milites, et semper postea milites et laicos recepit secundum antiquam consuetudinem; et adhuc recipiuntur, abbate domi existente.
\end{otherlanguage}

\switchcolumn

On the third day after Master Dennis was made cellarer, three knights with their squires were brought into the guest­house to be entertained there, the abbot being at home and sitting in his chamber. Now when that great­hearted Achilles heard this, not wishing to fail in his office as did the others, he arose and took the keys of the cellar, and bearing the knights with him to the hall of the abbot, came himself into the abbot's presence. And he said to him, ``Lord, you know well that the custom of the abbey is that knights and laymen be received in your hall, if the abbot be at home. I neither wish, nor am I able, to receive guests whose entertainment is your care. But if it be otherwise, take the keys of your cellar, and appoint another cellarer at your pleasure.'' When the abbot heard this, he received those knights perforce and ever after he received knights and laymen in accordance with ancient custom. And they are still so received when the abbot is at home.

\switchcolumn*

\begin{otherlanguage}{latin}
\blockhead{How Abbot Hugh strove to win the favour of Master Samson.}{4}{-0.45cm}
Volens aliquando abbas Hugo magistrum Sampsonem conciliare sibi in gratiam, subsacristam eum constituit; qui s\ae{}pius accusatus, s\ae{}pius de officio in officium est translatus; quandoque factus est magister hospitum, quandoque pitentiarius,\engnotetext{The official of the monastery who had charge of the distribution of the pittances to the monks, that is, additional allowances of food or drink, the result of some benefaction.} quandoque tertius prior, et iterum subsacrista; et multi ei adversabantur qui postea ei adulabantur. Ille vero aliter agens quam ceteri officiales, nunquam ad adulandum flecti potuit; unde dicebat abbas suis familiaribus, se nunquam vidisse talem hominem, quem non posset converti ad suam voluntatem, pr\ae{}ter Sampsonem subsacristam.

\end{otherlanguage}

\switchcolumn

At one time Abbot Hugh desired to win the favour of Master Samson, and made him his subsacristan. He was often accused, often transferred from one office to another. For he was made guest-master, and then pittance-­master,\engnotenum{} then third prior and finally again subsacristan. Then many strove against him who afterwards were his flatterers. But Samson did not bear himself as did the other officials, nor could he ever be brought to flatter. Wherefore the abbot said to his intimates that never had he seen a man whom he could not bend to his will, save Samson the subsacristan.

\newpage

\switchcolumn*


\begin{otherlanguage}{latin}
\blockhead[\textsc{a.d}.\ \oldstylenums{1180} Sep.\ \oldstylenums{9}.]{How Abbot Hugh came by his death.}{4}{-0.1cm}
Venit abbati Hugoni in mentem, anno vicesimo tertio abbati\ae{} su\ae{},\engnotetext{There is an account of this event in the \emph{Chronica Buriensis} (\emph{Mem}. III., \oldstylenums{6}).} adire sanctum Thomam orandi gratia; arreptoque in itinere, in crastino nativitatis sanct\ae{} Mari\ae{} prope Rouecestriam miserabiliter cecidit, ita quod patella tibi\ae{} de proprio loco exivit et resedit in poplite. Occurrerunt medici, et eum multis modis cruciabant, sed non sanabant. Reportatus est ad nos in feretro equitario, et devote susceptus, sicut decuit. Quid multa? conputruit crux\footnote[\textdagger]{lege \emph{crus}, Roke.} ejus, et ascendit dolor usque ad cor, et ex dolore arripuit eum febris tertiana, et in quarta accessione expiravit; et animam reddidit Deo in crastino sancti Bricii.

\end{otherlanguage}

\switchcolumn

In the twenty-­third year of his being abbot,\engnotenum{} it came into the mind of Abbot Hugh to journey to the shrine of the blessed Thomas to pray there. And when he was almost at his journey's end, and was near unto Rochester on the morrow of the Nativity of the Blessed Mary, he most unhappily fell, so that his knee-­pan was put out and lodged in the ham of his leg. Physicians hastened to him, and put him to pain in many ways, but they healed him not. So he was borne back to us in a horse-­litter, and received with great concern, as was fitting. To put it shortly, his leg mortified and the sickness spread to his heart. Pain brought on a tertian fever, and in the fourth fit he died, rendering his soul to God on the morrow of Saint Brice's day.

\switchcolumn*

\begin{otherlanguage}{latin}
Antequam mortuus esset, distracta fuerunt omnia a servientibus suis, ita quod nichil omnino in domibus abbatis remanserat, nisi tripodes et mens\ae{} qu\ae{} asportari non poterant. Vix abbati remanserant coopertorium suum et du\ae{} stragul\ae{} qu\ae{} veteres erant et fract\ae{}, quas aliquis apposuerat qui integras abstulerat. Non erat aliquid ad pretium unius denarii quod possit distribui pauperibus pro anima ejus. Sacrista dicit non pertinere ad eum ut hoc faceret, dicens se expensas abbati et famili\ae{} su\ae{} invenisse per mensem integrum; quia nec firmarii, qui villas tenebant, volebant aliquid dare ante tempus constitutum, nec creditores volebant aliquid commodare, videntes eum infirmum usque ad mortem. Quinquaginta tamen solidos invenit firmarius de Palegrava ad distribuendum pauperibus; hac ratione, quia firmam de Palegrava intravit illa die. Sed illi quinquaginta solidi erant postea redditi iterum bailivis regis, firmam integram exigentibus ad opus regis.
\end{otherlanguage}

\switchcolumn

Ere he was dead, all things were thrown into disorder by his servants, so that in the abbot's houses there was nothing at all left, except stools and tables which could not be carried away. There hardly remained to the abbot a coverlet and quilts which were old and torn, and which someone who had taken away those which were sound, had left in their place. There was not even some thing of a penny's value which might be given to the poor for the good of his soul. The sacristan said that it was not his affair to do this, declaring that he had found the money for the expenses of the abbot and his household for a full month, since neither would those who farmed the townships pay anything before the appointed time, nor would the creditors give any grace, as they saw the abbot to be sick unto death. However the tenant of Palegrave found fifty shillings for distribution to the poor, because he entered upon his tenancy of Palegrave on that day. But those fifty shillings were afterwards again paid to the king's officers, who exacted the full rent for the use of the king.

\switchcolumn*

\begin{otherlanguage}{latin}
\blockhead{How the death of Abbot Hugh was told to the king, and of those things which the servants of the king did.}{4}{-0.65cm}
Sepulto abbate Hugone, decretum est in capitulo, ut aliquis nunciaret Ranulfo de Glanvill,\footnote[\textdagger]{Ranulf de Glaville, author of the famous treatise \emph{De legibus et consuetudinibus regni Angli\ae{}}, which explains the forms of procedure observed in the \emph{curia regis}, took the cross in the last year of Henry II.'s reign, and died at the siege of Acre in \oldstylenums{1190}.} justiciario Angli\ae{}, mortem abbatis. Magister Sampson et Magister R.\ Ruffus, monachi nostri, cito transfretaverunt\footnote[\ddag]{``King Henry II. kept Christmas at Le Mans.'' (Roke.)}\engnotetext{Justiciar from \oldstylenums{1180} to \oldstylenums{1189}. He was deprived at the accession of Richard, and died on the crusade at the siege of Acre (\oldstylenums{1190}), whither he had preceded Richard I. The news of the vacancy was sent to the justiciar, owing to the absence of Henry II.\ in Maine.} nunciantes hoc idem domino regi, et impetraverunt literas, ut res et redditus conventus, qui separati sunt a rebus et redditibus abbatis, essent integr\ae{} in manu prioris et conventus, et reliqua pars abbati\ae{} esset in manu regis. Data est custodia abbati\ae{}\engnotetext{Rokewode (pp.\ \oldstylenums{109}--\oldstylenums{11}) gives the accounts of the wardship from the Pipe Roll of Norfolk and Suffolk.} Roberto de Cokefeld, et R.\ de Flamville senescallo, qui statim omnes famulos abbatis et parentes ejus, quibus abbas aliquid donaverat, postquam infirmus fuerat, vel qui aliquid de rebus abbatis abstulerant, posuerunt per vadium et plegios, et etiam capellanum abbatis, monachum nostrum, quem prior plegiavit; et intrantes vestiarium nostrum omnia ornamenta ecclesi\ae{} in chirographo subscribi fecerunt.\footnote[\textdagger]{The accounts of the wardens during the vacancy of the abbotship are still extant among the Pipe Rolls, and Mr.\ Rokewode has deduced from them that ``the rental of the abbot of St.\ Edmund's for year \oldstylenums{1181}, that is to say, from Mich.\ \oldstylenums{1180} to Mich.\ \oldstylenums{1181}, according to the ancient assise, and exclusive of the sustenance of the monks, who had their own portion of lands, was 326l.\ 12s.\ 4d.''}

\end{otherlanguage}

\switchcolumn

When Abbot Hugh had been laid to rest, it was decreed in the chapter that one should tell the death of the abbot to Ranulf de Glanvill,\engnotenum{} Justiciar of England. Master Samson and Master Robert Ruffus hastened across the sea, bearing this same news to the lord King, and obtained from him letters directing that the possessions and revenues of the monastery, which were distinct from those of the abbot, should remain entirely in the hands of the prior and of the monastery, and that the rest of the abbey's property should be in the hands of the King. The wardship of the abbey\engnotenum{} was given to Robert de Cokefield and to Robert de Flamvill the seneschal, who at once placed under surety and pledges those of the servants and relatives of the abbot to whom the abbot had given anything after he fell ill, or who had taken anything from the property of the abbot. And they also treated the chaplain of the abbot in the same way, for whom the prior became surety. And entering our vestry, they made a double inventory of all the ornaments of the church.

\switchcolumn*

\begin{otherlanguage}{latin}
\blockhead{How the prior ruled the monastery, while there was no abott.}{4}{-0.45cm}
Vacante abbatia, prior\engnotetext{The prior was Robert, who was appointed on the deposition of prior Hugh (\oldstylenums{1173}). He held office until his death, which took place about \oldstylenums{1200} (see text, p.\ \oldstylenums{194}).} super omnia studuit ad pacem conservandam in conventu, et ad honorem ecclesi\ae{} conservandum in hospitibus suscipiendis, neminem volens turbare, neminem ad iracundiam provocare, ut omnes et omnia in pace posset conservare; dissimulans tamen qu\ae{}dam corrigenda de obedientiariis nostris, et maxime de sacrista, tanquam non curaret quid ipse ageret de sacristia, qui, tempore quo abbatia vacavit, nec debitum aliquod adquietavit, nec aliquid \ae{}dificavit; sed oblationes et obventiones stulte distrahebantur. Unde prior, qui caput conventus erat, pluribus videbatur vituperandus, et remissus dicebatur. Et hoc memorabant fratres nostri inter se, quando perventum fuit ad faciendam electionem abbatis.

\end{otherlanguage}

\switchcolumn

There being no abbot, the prior\engnotenum{} took care, above all things, to preserve peace in the monastery and to maintain the repute of the house in the matter of receiving guests. He wished to disturb no one, to provoke no one to anger, that he might keep all men and all things in quiet. But he overlooked some acts of our officials which should have been corrected; and especially in the case of the sacristan, as if he cared not how that office was performed. Now the sacristan, while the abbey was vacant, neither paid any debt nor erected any building, but the offerings and accidental receipts were foolishly wasted. Wherefore the prior, who was head of the monastery, seemed to many to be blameworthy, and was called slack. And our brothers reminded each other of this when the time came for electing an abbot.

\switchcolumn*

\begin{otherlanguage}{latin}
\blockhead{How the cellarer and the sacristan behaved during the vacancy.}{4}{-0.45cm}
Celerarius noster omnes hospites, cujuscumque conditionis essent, suscepit ad expensas conventus. Willelmus sacrista ex sua parte dabat et expendebat; homo benignus, dans danda et non danda, oculos omnium exc\ae{}cans muneribus.\engnotetext{Deut.\ xvi., \oldstylenums{19}.}

\end{otherlanguage}

\switchcolumn

Our cellarer entertained all guests of whatever condition, at the expense of the monastery. William the sacristan, for his part gave and spent. Kind man! he spent indiscriminately, and blinded the eyes of all with gifts.\engnotenum{}

\switchcolumn*

\begin{otherlanguage}{latin}
\blockhead{Concerning the conduct of Samson the subsacristan during the vacancy.}{4}{-0.45cm}
Samson subsacrista, magister super operarios, nichil fractum, nichil rimatum, nichil fissum, nichil inemendatum reliquit pro posse suo; unde conventum et maxime claustrales sibi conciliavit in gratiam. In diebus illis chorus noster fuit erectus, Samsone procurante, historias pictur\ae{} ordinante, et versus elegiacos dictante. Attractum fecit magnum de lapidibus et sabulo ad magnam turrim ecclesi\ae{} construendam.\engnotetext{The tower in the centre of the west front. It was begun probably either by abbot Baldwin (\oldstylenums{1065}--\oldstylenums{97}), or by abbot Robert I.\ (\oldstylenums{1100}--\oldstylenums{02}). It was blown down in \oldstylenums{1210}, before the death of Samson (\emph{Ann.\ St.\ Edmund, Mem}.\ II., \oldstylenums{18}).} Et interrogatus unde denarios haberet ad hoc faciendum, respondit quosdam burgenses dedisse ei occulte pecuniam ad turrim \ae{}dificandam et perficiendam. Dicebant tamen quidam fratres nostri, quod Warinus monachus noster, custos feretri, et Samson subsacrista communi consilio surripuerunt, quasi furtive, portionem aliquam de oblationibus feretri, ut eam in usus necessarios ecclesi\ae{}, et nominatim ad \ae{}dificationem turris, expenderent; hac ratione ducti, quia videbant quod oblationes in usus extraordinarios expendebantur ab aliis, qui, ut verius dicam, eas furabantur. Et ut tam felicis furti sui suspicionem tollerent pr\ae{}nominati duo viri, truncum quendam fecerunt, concavum, et perforatum in medio vel in summo, et obseratum sera ferrea; et erigi fecerunt in magna ecclesia, juxta ostium extra chorum in communi transitu vulgi, ut ibi ponerent homines elemosinam suam ad \ae{}dificationem turris.

\end{otherlanguage}

\switchcolumn

Samson the subsacristan, who was master over the workmen, did his utmost that nothing which was broken, and no chink or crack, should remain unrepaired. In this way he won the favour of the monastery, and more especially of the cloistered monks. At that time, and under Samson's direction, was our choir built. He determined the subjects, the paintings, and composed elegiac verses for them. He made a great store of stone and sand for building the great tower of the church.\engnotenum{} And when he was asked where he found the money for this work, he answered that some of the townsfolk had given him money secretly for the building and completing of the tower. But some of our brothers said that Warin our monk and custodian of the shrine, had agreed to take, or as it were to steal, some part of the offerings of the shrine, and to spend it for the necessary purposes of the church, especially for the building of the tower. They were led to this opinion by the fact that they saw the strange uses to which these offerings were put by others, who, to speak the truth, did steal them. And in order to remove from themselves the suspicion of so happy a theft, Samson and Warin made a hollow chest, in the middle of the cover of which there was a hole, and which was secured with an iron bar. This chest they caused to be placed in the great church near the door outside the choir, where all the people passed by, that men might place therein gifts for the building of the tower.

\switchcolumn*

\begin{otherlanguage}{latin}
\blockhead{How the enemies of Samson prevailed against him, but only for a time.}{4}{-0.45cm}
Willelmus vero sacrista socium suum Samsonem suspectum habuit, et multi alii qui partem ejusdem Willelmi fovebant, tam Christiani quam Jud\ae{}i. Judei, inquam, quibus sacrista pater et patronus dicebatur; de cujus protectione gaudebant, et liberum ingressum et egressum habebant, et passim ibant per monasterium, vagantes per altaria et circa feretrum, dum missarum celebrarentur sollemnia: et denarii eorum in thesauro nostro sub custodia sacrist\ae{} reponebantur, et, quod absurdius est, uxores eorum cum parvis suis in pitanceria nostra tempore werr\ae{} hospitabantur.

\end{otherlanguage}

\switchcolumn

But William the sacristan mistrusted his colleague Samson, as did many others, both Christians and Jews, who favoured the opinion of the same William. The Jews, I say, to whom the sacristan was said to be a father and a patron. And they did rejoice in his protection, having freedom to enter and to leave the monastery, and wandering all over it. For they walked by the altars and round the shrine while high mass was being celebrated; their money was lodged in our treasury under the care of the sacristan; and, a thing still more foolish, their wives and little ones were entertained in our pittancy during time of war.

\switchcolumn*

\begin{otherlanguage}{latin}
Accepto itaque consilio qualiter irruerent in Samsonem inimici vel adversarii ejus, convenerunt Robertum de Cokefeld et socium ejus, qui custodes erant abbati\ae{}, et induxerunt eos ad hoc, quod illi prohibuerant ex parte regis, ne aliquis aliquod opus vel aliquod \ae{}dificium faceret, quamdiu abbatia vacaret; sed potius denarii ex oblationibus colligerentur et conservarentur ad faciendam solutionem alicujus debiti. Et sic illusus est Samson\engnotetext{Jud.\ xvi., \oldstylenums{17}; xvi., \oldstylenums{19}.}, et recessit ab eo fortitudo ejus; nec de c\ae{}tero aliquid operari potuit, sicut voluit. Potuerunt quidem adversarii ejus rem differre, sed non auferre; quia resumptis viribus suis, et subversis duobus columnis,\engnotetext{Jud.\ xvi., \oldstylenums{29}.} id est, remotis duobus custodibus abbati\ae{} quibus aliorum malitia innitebatur, dedit ei Dominus, processu temporis, potestatem perficiendi votum suum ut pr\ae{}dictam turrim \ae{}dificaret, et pro desiderio suo consummaret. Et factum est ac si ei divinitus diceretur: ``Euge, serve bone et fidelis,\engnotetext{Matt.\ xxv., \oldstylenums{21}.} quia super pauca fuisti fidelis, super multa, \&c.''
\end{otherlanguage}

\switchcolumn

Therefore, having taken counsel together how they might attack Samson, his enemies and adversaries went to Robert de Cokefield and to his colleague, who had the wardship of the abbey, and persuaded them to forbid in the name of the king that any one should do any work or build anything while the abbacy was vacant, but rather should the money from he offerings be collected and saved for the payment of some part of the debt. Thus was Samson mocked, and his strength went from him,\engnotenum{} and he could not from that time do any work as he desired. But though his enemies could delay his work, they could not finally interrupt it. For he regained his strength and overthrew the two middle pillars,\engnotenum{} that is, he removed the two wardens in whom the malice of the others trusted. And afterwards in course of time, the Lord gave him power to perform his vow that he would build the said tower, and to finish it according to his wish. And so it came to pass as though a voice from Heaven had said to him, ``Well done, thou good and faithful servant;\engnotenum{} thou hast been faithful over a few things; I will make thee ruler over many things.''

\switchcolumn*

\begin{otherlanguage}{latin}
\blockhead{How the monks disputed among themselves which of them should be abbot.}{4}{-0.45cm}
Vacante abbatia, s\ae{}pe, sicut decuit, rogavimus Dominum et sanctum martyrem \AE{}dmundum, ut nobis et ecclesi\ae{} nostr\ae{} congruum darent pastorem, singulis ebdomadibus ter cantantes vii. psalmos p\oe{}nitentiales prostrati in choro, post exitum in capitulo: et erant aliqui, quibus si constaret quis futurus esset abbas, non ita devote orassent. De eligendo abbate, si rex nobis liberam concederet electionem, diversi diversis modis loquebantur, quidam publice, quidam occulte; et ``quot homines tot sententi\ae{}.''\footnote[\textdagger]{Ter.\ Phorm.\ \oldstylenums{2}, \oldstylenums{3}, \oldstylenums{14}.}

\end{otherlanguage}

\switchcolumn

The abbacy being vacant, we often, as was right, made supplication unto the Lord and to the blessed martyr Edmund that they would give us and our church a fit pastor. Three times in each week, after leaving the chapter, did we prostrate ourselves in the choir and sing seven penitential psalms. And there were some who would not have been so earnest in their prayers if they had known who was to become abbot. As to the choice of an abbot, if the king should grant us free election, there was much difference of opinion, some of it openly expressed, some of it privately; and every man had his own ideas.

\switchcolumn*

\begin{otherlanguage}{latin}
Dixit quidam de quodam: ``Ille, ille frater, bonus monachus est, probabilis persona; multum scit de ordine et consuetudinibus ecclesi\ae{}: licet non sit tam perfectus philosophus sicut quidam alii, bene potest esse abbas. Abbas Ordingus\footnote[\ddag]{This was the abbot to whom Geoffrey de Fontibus addressed his work ``De Infantia S.\ Eadm.''; see above, p.\ \oldstylenums{93}.}\engnotetext{Ording de Stowe was elected in \oldstylenums{1138}, when abbot Anselm was elected to the bishopric of London. Anselm was driven from his see soon afterwards on the ground that the election had taken place without the assent of the dean of London, and resumed his abbacy. On his death in \oldstylenums{1148}, Ording was again elected, and held the abbacy to his death in \oldstylenums{1156}. (\emph{Chron.\ Bur., Mem}.\ III., \oldstylenums{5}--\oldstylenums{6}.)} homo illiteratus fuit, et tamen fuit bonus abbas et sapienter domum istam rexit: legitur etiam in fabulis, melius fuit ranis eligere truncum in regem, super quem confidere possent, quam serpentem, qui venenose sibilaret, et post sibilum subjectas devoraret.'' Respondit alter: ``Quomodo potest hoc fieri? quomodo potest facere sermonem in capitulo, vel ad populum diebus festivis, homo qui literas non novit? quomodo habebit scientiam ligandi et solvendi,\engnotetext{Cp.\ Matt.\ xvi., \oldstylenums{19}.} qui scripturas non intelligit? cum sit ars artium, scientia scientiarum, regimen animarum. Absit ut statua muta erigatur in ecclesia sancti \AE{}dmundi, ubi multi literati viri et industrii esse dinoscuntur.''
\end{otherlanguage}

\switchcolumn

One said of a certain brother, ``He, that brother, is a good monk, a likely person. He knows much of the rule and of the customs of the church. It is true that he is not so profoundly wise as are some others, but he is quite capable of being abbot. Abbot Ording\engnotenum{} was illiterate, and yet he was a good abbot and ruled this house wisely; and one reads in the fable that the frogs did better to elect a log to be their king than a serpent, who hissed venomously, and when he had hissed, devoured his subjects.'' Another answered, ``How could this thing be? How could one who does not know letters preach in the chapter, or to the people on feast days? How could one who does not know the scriptures have the knowledge of binding and loosing?\engnotenum{} For the rule of souls is the art of arts, the highest form of knowledge. God forbid that a dumb idol be set up in the church of Saint Edmund, where many men are to be found who are learned and industrious.''

\switchcolumn*

\begin{otherlanguage}{latin}
Item dixit alius de alio: ``Ille frater vir literatus est, eloquens et providus; rigidus in ordine; multum dilexit conventum, et multa mala pertulit pro bonis ecclesi\ae{}: dignus est ut fiat abbas.'' Respondit alter: ``A bonis clericis libera nos, Domine: ut a baratoribus de Norfolchia\engnotetext{The allusion is to Samson, who was a native of Norfolk.} nos conservare digneris, te rogamus audi nos.''\footnote[\textdagger]{The words of supplication are taken from the Litany of the Saints.}
\end{otherlanguage}

\switchcolumn

Again, one said of another, ``That brother is a literate man, eloquent and prudent, and strict in his observance of the rule. He loves the monastery greatly, and has suffered many ills for the good of the church. He is worthy to be made abbot.'' Another answered, ``From good clerks deliver us, oh Lord! That it may please Thee to preserve us from the cheats of Norfolk;\engnotenum{} we beseech Thee to hear us!''

\switchcolumn*

\begin{otherlanguage}{latin}
Item dixit quidam de quodam: ``Ille frater bonus husebondus est: quod probatur ex warda sua, et ex obedientiis quas bene servavit, et \ae{}dificiis et emendationibus quas fecit. Multum potest laborare et domum defendere, et est aliquantulum clericus, quamvis nimi\ae{} liter\ae{} non faciant eum insanire:\engnotetext{Cp.\ Acts xxvi., \oldstylenums{24}.} ille dignus est abbatia.'' Respondit alter: ``Nolit Deus ut homo, qui non potest legere, nec cantare, nec divina officia celebrare, homo improbus et injustus, et excoriator pauperum hominum, fiat abbas.''
\end{otherlanguage}

\switchcolumn

And again, one said of one, ``That brother is a good husbandman; this is proved by the state of his office, and from the posts in which he has served well, and from the buildings and repairs which he has effected. He is well able to work and to defend the house, and he is something of a scholar, though too much learning has not made him mad.\engnotenum{} He is worthy of the abbacy.'' Another answered, ``God forbid that a man who can neither read nor sing, nor celebrate the holy office, a man who is dishonest and unjust, and who evil intreats the poor men, should be made abbot.''

\switchcolumn*

\begin{otherlanguage}{latin}
Item dixit aliquis de aliquo: ``Ille frater homo benignus est, affabilis et amabilis, pacificus et compositus, largus et liberalis, vir literatus et eloquens, et satis idonea persona in vultu et in gestu, et a multis dilectus intus et extra; et talis homo ad magnum honorem ecclesi\ae{} posset fieri abbas, si Deus vellet.'' Respondit alter: ``Non honor esset sed onus de homine qui nimis delicatus est in cibo et potu; qui virtutem reputat multum dormire; qui multum scit expendere et parum adquirere; qui stertit quando ceteri vigilant; qui semper vult esse in abundantia, nec curat de debitis qu\ae{} crescunt de die in diem, nec de expensis unde adquietari possint; solicitudinem et laborem odio habens, nihil curans, dummodo unus dies vadat et alter veniat; homo adulatores et mendaces diligens et fovens; homo alius in verbo in alius in opere. A tali pr\ae{}lato defendat nos Dominus.''
\end{otherlanguage}

\switchcolumn

Again, one said of another, ``That brother is a kindly man, friendly and amiable, peaceful and calm, generous and liberal, a learned and eloquent man, and proper enough in face and gait. He is beloved of many within and without the walls, and such an one might become abbot to the great honour of the church, if God wills.'' Another answered, ``It is no credit, but rather a disgrace, in a man to be too particular as to what he eats and drinks, to think it a virtue to sleep much, to know well how to spend and to know little how to gain, to snore while others keep vigil, to wish ever to have abundance, and not to trouble when debts daily increase, or when money spent brings no return; to be one who hates anxiety and toil, caring nothing while one day passes and another dawns; to be one who loves and cherishes flatterers and liars; to be one man in word and another in deed. From such a prelate the Lord defend us.''

\switchcolumn*

\begin{otherlanguage}{latin}
Item dixit quidam de socio suo: ``Ille vir fere sapientior est omnibus nobis, et in s\ae{}cularibus et in ecclesiasticis; vir magni consilii, rigidus in ordine, literatus et eloquens et personalis statur\ae{}: talis pr\ae{}latus decet ecclesiam nostram.'' Respondit alter: ``Verum est, si esset rat\ae{} et probat\ae{} opinionis. Fama ejus laborat, qu\ae{} forte mentitur, forte non mentitur; et licet ille homo sapiens sit, humilis in capitulo, devotus in psalmis, rigidus in claustro, dum claustrale\footnote[\textdagger]{Claustralis? (Roke.)} est, ex consuetudine tamen habet: quod si preest in obedientia aliqua, nimis indignans est, monachos parvipendens, s\ae{}culares homines familiarius diligens, et, si iratus fuerit, vix aliquid verbum ultro alicui fratri respondere, nec etiam interroganti.''
\end{otherlanguage}

\switchcolumn

And again, one said of his friend, ``That man is almost wiser than all of us, and that both in secular and in ecclesiastical matters. He is a man skilled in counsel, strict in the rule, learned and eloquent, and noble in stature; such a prelate would become our church.'' Another answered, ``That would be true if he were a man of good and approved repute. But his character has been questioned, perhaps falsely, perhaps rightly. And though the man is wise, humble in the chapter, devoted to the singing of psalms, strict in his conduct in the cloister while he is a cloistered monk, this is only from force of habit. For if he have authority in any office, he is too scornful, holding monks of no account, and being on familiar terms with secular men, and if he be angry, he will scarce say a word willingly to any brother, even in answer to a question.''

\switchcolumn*

\begin{otherlanguage}{latin}
Audivi scilicet alium fratrem reprobatum a quibusdam, quia impeditioris lingu\ae{} fuerat; de quo dicebatur quod habebat pastum vel draschium in ore suo, cum loqui deberet. Et ego quidem, tunc teraporis juvenis,\engnotetext{I.\ Cor.\ xiii., \oldstylenums{11}.} sapiebam ut juvenis, loquebar ut juvenis, et dixi quod non consentirem alicui ut fieret abbas, nisi sciret aliquid de dialectica, et sciret discernere verum a falso. Item dixit quidam, qui sibi videbatur sapiens: ``Stultum et idiotam pastorem tribuat nobis omnipotens Dominus, ut necesse sit ei se adjuvare de nobis.'' Audivi scilicet quendam virum, industrium, et literatum, et nobilitate generis splendidum, reprobatum esse a quibusdam prioribus nostris hac causa, quia novicius erat. Novicii dicebant de prioribus suis, quod senes valitudinarii erant et ad abbatiam regendam minus idonei. Et ita multi multa loquebantur, et unusquisque abundabat in suo sensu.\engnotetext{Rom.\ xiv., \oldstylenums{5}.}
\end{otherlanguage}

\switchcolumn

I heard in truth another brother abused by some because he had an impediment in his speech, and it was said of him that he had pastry or draff in his mouth when he should have spoken. And I myself, as I was then young, understood as a child,\engnotenum{} spake as a child; and I said that I would not consent that any one should be made abbot unless he knew something of dialectic, and knew how to distinguish the true from the false. One, moreover, who was wise in his own eyes, said, ``May Almighty God give us a foolish and stupid pastor, that he may be driven to use our help.'' And I heard, forsooth, that one man who was industrious, learned, and pre-eminent for his high birth, was abused by some of the older men because he was a novice. The novices said of their elders that they were invalid old men, and little capable of ruling an abbey. And so many men said many things, and every man was fully persuaded in his own mind.\engnotenum{}

\switchcolumn*

\begin{otherlanguage}{latin}
\blockhead{How Samson the subsacristan bore himself while others discussed the vacancy.}{4}{-0.65cm}
Vidi Samsonem subsacristam assidentem quidem his conventiculis tempore minutionis\engnotetext{The monks practised blood-letting five times a year,---in September, before Advent and Lent, and after Easter and Pentecost, under the rule of St.\ Victor. An account of the manner in which it was practised at St.\ Edmund's is to be found in the \emph{Liber Albus}. (Rokewode, p.\ \oldstylenums{113}, \emph{Mem}., I., \oldstylenums{221}, note.)} (quo tempore claustrales solent alternatim secreta cordis revelare, et adinvicem conferre); vidi eum assidentem et subridentem et tacentem, et singulorum verba notantem, et aliquam ex pr\ae{}scriptis sententiis in fine xx.\ annorum memorantem.

\end{otherlanguage}

\switchcolumn

Then I saw Samson the subsacristan sitting by, for the time of this little council was a season of blood-letting,\engnotenum{} when the cloistered monks were wont to reveal the secrets of their hearts in turn, and to discuss matters one with another. I saw him sitting by and laughing to himself, while he kept silence and marked that which each one said, so that at the end of twenty years lie was able to remember some part of the various opinions which I have set forth above.

\switchcolumn*

\begin{otherlanguage}{latin}
\blockhead{How the author spoke his mind too hastily.}{3}{-0.45cm}
Quo audiente, solebam respondere ita judicantibus, dicens, quod, si debemus expectare ad eligendum abbatem donec inveniamus aliquem qui sine omni reprehensione et macula fuerit, nunquam talem inveniemus, quia nemo sine crimine vivit, et nichil omni parte beatum.\footnote[\textdagger]{Hor.\ Od.\ ii.\ \oldstylenums{16}.} Quodam tempore non potui cohibere spiritum meum quin pr\ae{}cipitarem sententiam meam, putans me loqui fidis auribus, et dixi quendam indignum abbatia, qui me multum dilexerat prius, et multa bona contulerat; et alium dignum duxi, et nominavi aliquem, quem minus diligebam. Loquebar secundum conscientiam meam, considerans potius communem utilitatem ecclesi\ae{} quam meam promotionem; et verum dixi; quod sequentia probaverunt. Et ecce unus ex filiis Belial dictum meum revelavit benefactori meo et amico; ob quam causam, usque ad hodiernum diem nunquam postea nec prece nec pretio potui recuperare gratiam ejus ad plenum. Quod dixi, dixi;\\
\vspace{-0.75cm}
\begin{verse}
\footnotesize
Et semel emissum volat irrevocabile verbum.\footnote[\ddag]{Hor.\ Ep.\ i.\ \oldstylenums{18}, \oldstylenums{71}.}
\end{verse}

\end{otherlanguage}

\switchcolumn

And when I heard these things, I was wont to answer to those who so judged, and to say that if we had to wait to choose an abbot until we found one without spot or flaw, we should never find such an one, since there is none living without fault, and nothing altogether good. At one time I could not refrain my spirit, but put forward my own opinion only too readily, thinking that I was speaking to faithful ears. And I said that one was not worthy of the abbacy who had before loved me dearly and done much good to me. And I put forward another as worthy and named him, a man whom I loved but little. I spoke according to my conscience, thinking rather of the general well­being of the church than of my own promotion; and I spoke the truth, as subsequent events proved. And behold, one of the sons of Belial revealed my saying to my benefactor and friend, wherefore to this very day I have never been able, by prayer or present, to regain his full favour. What I have said, I have said. And the word once uttered flies beyond recall.

\switchcolumn*

\begin{otherlanguage}{latin}
Unum restat; quod caveam mihi de c\ae{}tero, et, si tamdiu vixero ut videam abbatiam vacare, videbo quid, cui, et quando loquar de tali materia, ne vel Deum offendam mentiendo, vel hominem importune loquendo. Ad consilium meum tum erit, si duravero, ut aliquem eligamus non multum bonum monachum, non multum sapientem clericum, nec nimis idiotam, nec nimis dissolutum; ne, si nimis sapiat, de se et de proprio sensu nimis confidat, et alios vilipendat; vel si nimis brutescat, in opprobrium aliis fiat. Scio quis dixerit: ``medio tutissimus ibis;''\footnote[\textdagger]{Ov.\ Met.\ ii.\ \oldstylenums{137}.} et illud, ``medium tenuere beati.'' Vel forte, sanius consilium erit omnino tacere, ut dicam in corde meo: ``Qui potest capere, capiat.''\engnotetext{Matt.\ xix., \oldstylenums{12}.}
\end{otherlanguage}

\switchcolumn

One thing remains for me,---that I take care henceforth; and if I should live long enough to see the abbey once more vacant, I will see what, and to whom, and when I speak of so weighty a matter, that I offend not God by lying or man by hasty talk. Then it will be my care, if I live, that we elect one who is neither a very good monk, nor a very wise clerk, nor too foolish, nor too dissolute; lest, if he know too much, he have also too much confidence in himself and in his own opinion, and hold others of small account; or if he be too foolish, he be abused by others. I know that one has said, ``You will walk most safely in the middle,'' and that ``Blessed are those who steer a middle course.'' And perchance it is wiser counsel to be silent altogether, so that I say in my heart, ``He that is able to receive it, let him receive it.''\engnotenum{}

\switchcolumn*

\begin{otherlanguage}{latin}
\blockhead[\textsc{a.d}.\ \oldstylenums{1181}.]{How the archbishop of Norway dwelt in the abbot's lodgings while the abbacy was vacant.}{5}{-0.6cm}
Vacante abbatia perhendinavit Augustinus archiepiscopus Norweie\footnote[\ddag]{In \oldstylenums{1180}, Eystein (Augustinus), archbishop of Trondheim, refusing to crown Sverrir, a successful rebel, who had defeated Magnus, king of Norway, was driven into excile, and came to England (William de Newburgh, iii.\ \oldstylenums{6}). ``It appears from the accounts of the wardens of the abbey . . . that the archbishop remained in the monastery from the vigil of St.\ Lawrence, \oldstylenums{9} Aug.\ \oldstylenums{1181}, until about the time of the election of abbot Samson, in February following, and that the corrodies allowed him amounted altogether to 94l.\ 10s.'' (Roke.) This Eystein wrote a remarkable bigoraphy of St.\ Olaf, recentedly edited from the unique MS.\ by my friend, the late Rev.\ F.\ Metcalfe, who was so long and so favourably known as a successful student of Scandinavian antiquity.} apud nos in domibus abbatis, habens per pr\ae{}ceptum regis singulis diebus x.\ solidos de denariis abbati\ae{}; qui multum valuit nobis ad habendam liberam electionem nostram, testimonium perhibens de bono,\footnote[\dag]{Joh.\ xviii.\ \oldstylenums{23}.} et publice protestans coram rege quod viderat et audierat. \engnotetext{Eystein, archbishop of Trondheim, was banished from Norway for political reasons and came to England in \oldstylenums{1180}. Rokewode (p.\ \oldstylenums{113}) collects from the accounts of the wardship of the abbey that he received £\oldstylenums{94} \oldstylenums{10}s.}

\end{otherlanguage}

\switchcolumn

While the abbacy was vacant, Augustine, Archbishop of Norway,\engnotenum{} dwelt with us in the abbot's lodgings, and by command of the king received ten shillings every day from the revenues of the abbot. He assisted us greatly to gain freedom of election, bearing witness of the good, and publicly declaring in the presence of the king that which he had seen and heard.

\switchcolumn*

\begin{otherlanguage}{latin}
\blockhead[\textsc{a.d}.\ \oldstylenums{1181}, June \oldstylenums{10}.]{Of the martyrdom of Saint Robert.}{4}{-0.45cm}
Eodem tempore fuit sanctus puer Robertas martirizatus,\engnotetext{Bale states that there was an account of the martyrdom of this child by the Jews, written by Jocelin. The work, however, is not known to exist at the present day. Gervase (I., \oldstylenums{296}) relates the event in somewhat similar terms to those in the text: ``In this year, at Eastertide, a certain boy, Robert by name, was martyred by the Jews at St.\ Edmund's, and he was afterwards honourably buried in the church of St.\ Edmund, and became famous, as common report goes, for many miracles.'' There is an account also in \emph{Chron.\ Bur}.\ (\emph{Mem}.\ III., \oldstylenums{6}) where the date (June \oldstylenums{10}th) is given.} et in ecclesia nostra sepultus, et fiebant prodigia et signa multa in plebe,\engnotetext{Acts v., \oldstylenums{12}.} sicut alibi scripsimus.

\end{otherlanguage}

\switchcolumn

In those days was the holy child Robert martyred,\engnotenum{} and was buried in our church. And many signs and wonders\engnotenum{} were wrought among the people, as we have related in another place.

\switchcolumn*

\begin{otherlanguage}{latin}
\blockhead[\textsc{a.d}.\ \oldstylenums{1182}, February.]{How thirteen men were chosen, by command of the king, to elect an abbot in the presence of the king.}{5}{-0.6cm}
Post mortem Hugonis abbatis, peracto anno cum tribus mensibus, pr\ae{}cepit dominus rex per literas suas, ut prior noster et duodecim de conventu, in quorum ore universitatis concordaret sententia, apparerent die statuto coram eo ad eligendum abbatem.

\end{otherlanguage}

\switchcolumn

Now when a year and three months had passed since the death of Abbot Hugh, the lord king commanded by his letters that our prior and twelve members of the monastery, by whose lips the opinion of the whole community might be expressed unanimously, should appear before him on a stated day to elect an abbot.

\switchcolumn*

\begin{otherlanguage}{latin}
In crastino post susceptionem literarum, convenimus in capitulo de tanto tractaturi negotio. In primis lect\ae{} sunt liter\ae{} domini regis in conventu; postea rogavimus et oneravimus priorem in periculo anim\ae{} su\ae{}, ut xii.\ secundum conscientiam suam nominaret secum ducendos, de quorum vita et moribus constaret eos a recto nolle deviare.
\end{otherlanguage}

\switchcolumn

On the morrow after receiving the letters, we gathered in the chapter for the purpose of performing so important a task. And first the letters of the lord king were read in the assembly of the monastery; then we offered prayers, and bound the prior on the peril of his soul that he should conscientiously nominate to go with him twelve men, from whose life and manners he knew well that they would not stray from the right path.

\switchcolumn*

\begin{otherlanguage}{latin}
Qui petitis annuens, dictante Spiritu Sancto, sex ex una parte chori et sex ex altera nominavit, et sine contradictione nobis satisfecit. A dextro choro fuerunt Galfridus de Fordham, Benedictus, magister Dionisius, magister Samson subsacrista, Hugo tertius prior, et magister Hermerus, tunc temporis novicius: a sinistro, Willelmus sacrista, Andreas, Petrus de Broc, Rogerus celerarius, magister Ambrosius, magister Walterus medicus.
\end{otherlanguage}

\switchcolumn

Then he, by inspiration of the Holy Spirit, gave ear to these prayers, and named six from one side of the choir and six from the other, and he gave us satisfaction without any dispute arising. From the right­hand side of the choir he named Geoffrey de Fordham, Benedict, Master Dennis, Master Samson the subsacristan, Hugh the third prior, and Master Hermer, who was then a novice. From the left side he named William the sacristan, Andrew, Peter de Broc, Roger the cellarer, Master Ambrose, and Master Walter the physician.

\switchcolumn*

\begin{otherlanguage}{latin}
Unus autem dixit: ``Quid fiet si isti tredecim non possunt coram rege concordare in abbate eligendo?'' Respondit quidam: ``Quia hoc erit nobis et ecclesi\ae{} nostr\ae{} in opprobrium\engnotetext{Jer.\ xxiii., \oldstylenums{40}.} sempiternum.'' Voluerunt ideo plures ut electio fieret domi antequam c\ae{}teri recederent, ut per hanc providentiam non fieret dissensio coram rege; sed illud nobis videbatur stultum et dissonum facere sine regis assensu, quia nondum constabat nobis posse impetrare a domino rege ut liberam electionem haberemus.
\end{otherlanguage}

\switchcolumn

But one said, ``What shall be done if these thirteen cannot agree on the choice of an abbot in the presence of the king?'' One answered, ``That would be a perpetual shame\engnotenum{} to us and our church.'' For that cause many wished that the election might be made at home before the departure of the rest, so that by this means there might be no dissension in the presence of the king. But it seemed to us foolish and unbecoming to do this without the royal assent, since as yet we had no certain knowledge that we should obtain freedom of election from the lord king.

\switchcolumn*

\begin{otherlanguage}{latin}
\blockhead{How Samson suggested that the monastery should appoint men to make a secret choice of an abbot, and how this was done.}{5}{-0.6cm}
Samson subsacrista in spiritu\engnotetext{I.\ Cor.\ xii., \oldstylenums{3}.} loquens, ``Fiat,'' inquit, ``media via, ut hinc et inde periculum evitetur. Eligantur quatuor confessores de conventu, et duo ex senioribus prioribus de conventu, bon\ae{} opinionis, qui, visis sacrosanctis, tactis evangeliis, inter se eligant tres viros de conventu, ad hoc magis idoneos juxta regulam sancti Benedicti, et eorum nomina in scriptum redigant, et scriptum sub sigillo includant, et sic inclusum committatur nobis ituris ad curiam; et cum venerimus coram rege, et constiterit nobis de libera electione habenda, tunc demum frangatur sigillum, et sic certi erimus qui tres nominandi erunt coram rege. Et constiterit nobis, si dominus rex noluerit concedere nobis unum de nostris, reportetur sigillum integrum et sex juratoribus tradatur, ita quod secretum illorum imperpetuum celetur in periculum animarum suarum.''

\end{otherlanguage}

\switchcolumn

Then Samson the subsacristan, speaking by the Spirit of God,\engnotenum{} said, ``Let a middle course be taken, that so danger may be avoided on either side. Let four confessors be chosen from the monastery, and two of the older priors, men of repute, and let them look upon and take in their hands the most holy gospels, and choose among themselves three members of the monastery, men especially fitted according to the rule of the holy Benedict for this purpose. Then let them write down the names and seal that which is written, and let them give the writing thus secured to us on the eve of our departure for the court. And when we are come to the court, if it shall be determined that we have free election, then and not till then let the seal be broken, and so shall we know certainly the three who must be named in the presence of the king. Let it also be resolved that if the lord king will not grant us one of our own number, then the writing shall be brought back, with the seal unbroken, and delivered to the six sworn men, that so their secret may remain a secret for ever on the peril of their souls.''

\switchcolumn*

\begin{otherlanguage}{latin}
Huic consilio omnes adquievimus, et nominati sunt quatuor confessores; scilicet Eustachius, Gilbertus de Alueth, Hugo tertius prior, Antonius, et alii duo senes, Turstanus et Rualdus. Quo facto, exivimus cantantes, ``Verba mea,''\engnotetext{Ps.\ v., Vulgate.} et remanserunt pr\ae{}dicti sex habentes regulam sancti Benedicti pr\ae{} manibus, et negotium sicut pr\ae{}finitum fuerat impleverunt.
\end{otherlanguage}

\switchcolumn

In this council we all agreed, and the four confessors were nominated, to wit, Eustace, Gilbert de Alueth, Hugh the third prior, and Anthony, with two other old men, Thurstan and Rualdus. And when this had been done, we went out chanting the ``Verba mea,''\engnotenum{} while the said six remained behind with the rule of St.\ Benedict in their hands, and completed the matter as had been ordained.

\switchcolumn*

\begin{otherlanguage}{latin}
Dum illi sex hoc tractabant, nos de diversis eligendis diversa putabamus, habentes tamen omnes quasi pro certo Samsonem esse unum ex tribus, attendentes labores ejus et pericula mortis versus Romam\engnotetext{For Samson's own account of this, see text, p.\ \oldstylenums{77} ff.} pro bonis ecclesi\ae{} nostr\ae{}, et qualiter tractus et compeditus et incarceratus erat ab H.\ abbate, loquens pro communi utilitate; qui nec sic flecti potuit ad adulandum, licet cogi potuit ad tacendum
\end{otherlanguage}

\switchcolumn

While these six men performed their work, we had various opinions as to the choice of different men, but all considered it to be certain that Samson would be one of them. For they called to mind his labours and the danger of death, which he had endured in his journey to Rome\engnotenum{} for the good of our church, and how he had been ill treated, and bound, and put in prison by Abbot Hugh, because he spoke to the common advantage. And they considered he was a man who could not be brought to flatter, though he might be driven to keep silence.

\switchcolumn*

\begin{otherlanguage}{latin}
Facta autem mora, vocatus conventus rediit in capitulum. Et dixerunt senes se fecisse secundum quod pr\ae{}ceptum eis fuerat.
\end{otherlanguage}

\switchcolumn

So, after some delay, the whole monastery was summoned to return to the chapter. And the old men said that they had done as had been commanded them.

\switchcolumn*

\begin{otherlanguage}{latin}
\blockhead{How, on the advice of Samson, it was decided what should be done if the king would not grant freedom of election.}{5}{-0.6cm}
Tunc prior qu\ae{}sivit, quid fuerit si dominus rex nollet aliquem ex illis tribus in scriptis recipere; et responsum est, quod quemcunque vellet dominus rex suscipere, susciperetur, dum modo esset processus ecclesi\ae{} nostr\ae{}. Adjectum est etiam quod, si illi tredecim fratres viderent aliquid in alio scripto quod emendari deberet, secundum Deum de communi assensu vel consilio emendarent.

\end{otherlanguage}

\switchcolumn

Then the prior asked what and if the lord king would not accept any one of the three whose names were written down. And it was answered that, since whomsoever the king wished to be received, must be received, there was but one course open to our church. It was added also that if those thirteen brothers should see in any writing aught that should be altered, they should make the alteration, according to God, unanimously and after consultation.

\switchcolumn*

\begin{otherlanguage}{latin}
Samson subsacrista sedens ad pedes prioris dixit: ``Ecclesi\ae{} expedire si omnes juraremus in verbo veritatis, ut super quemcunque sors electionis caderet conventum rationabiliter tractaret, nec capitales obedientiales mutaret sine assensu conventus, nec sacristam gravaret, nec aliquem monacaret sine voluntate conventus;'' et hoc ipsum concessimus, omnes dextras erigentes in signum concessionis. Provisum est quod, si dominus rex vellet aliquem extraneum abbatem facere, non reciperetur a tredecim nisi per consilium fratrum domi remanentium.
\end{otherlanguage}

\switchcolumn

Samson the subsacristan, sitting at the feet of the prior, said, ``It would be for the good of the church were all to swear on the word of truth that on whomsoever the lot of election shall fall, that man shall treat the monastery reasonably, and not change the chief officials without the assent of the house, nor burden the sacristan, nor make any one a monk without the consent of the monastery.'' And we agreed on this matter, all raising their right hands in token of their approval. And it was provided that if the lord king willed that some stranger should be made abbot, the thirteen should not accept this man save with the advice of the brothers who remained at home.

\switchcolumn*

\begin{otherlanguage}{latin}
\blockhead{How the chosen thirteen journeyed to the king.}{3}{-0.55cm}
In crastino igitur iter arripuerunt illi tredecim versus curiam. Postremus omnium fuit Samson provisor expensarum, quia subsacrista erat, circa collum scrinium portans, quo liter\ae{} conventus continebantur, quasi omnium minister solus, et, sine armigero, froggum suum in ulnis bajulans, curiam exivit, socios sequens a longe.

\end{otherlanguage}

\switchcolumn

On the morrow, therefore, these thirteen set out for the court. Last of all was Samson, who had charge of the expenses of the journey as being subsacristan. And he bore a letter­-case round his neck, in which were contained the letters of the monastery, as if he were only servant of them all. So, with no attendant, and with his frock borne in his arms, he went out of the court, and followed far behind his comrades.

\switchcolumn*

\begin{otherlanguage}{latin}
In itinere versus curiam convenientibus fratribus in unum, dixit Samson bonum esse ut jurarent omnes, ut quicunque fieret abbas, redderet ecclesias de dominiis conventus in usum hospitalitatis; quod omnes concesserunt pr\ae{}ter priorem, qui dixit: ``Satis juravimus; tantum potestis gravare abbatem, quod ego non curabo abbatiam.'' Et hac occasione non juraverunt; et hoc bene actum est, quia si hoc esset juratum, non esset observatum.
\end{otherlanguage}

\switchcolumn

On their journey to the court, the brothers gathered together, and Samson said that it would be well if all were to swear that whoever might be made abbot, should restore the churches on the demesne lands of the monastery to the exercise of hospitality. To this all agreed save the prior, and he said, ``We have sworn enough; you will so limit the power of the abbot, that I would not care to be abbot at all.'' And for this reason, they did not swear; and it was well that they did not do, for had this oath been taken it would not have been observed.

\switchcolumn*

\begin{otherlanguage}{latin}
\blockhead{Of the dreams which the brothers dreamed concerning the election of a new abbot.}{4}{-0.65cm}
Eodem die quo tredecim recesserunt, sedentibus nobis in claustro, dixit Willelmus de Hastinga unus ex fratribus nostris: ``Scio quod habebimus abbatem unum de nostris;'' et interrogatus quomodo hoc sciret, respondit, se vidisse in somnis prophetam albis indutum stantem pr\ae{} foribus monasterii, et se qu\ae{}sisse in nomine Domini utrum haberemus abbatem aliquem de nostris. Et respondit propheta: ``Habebitis unum de vestris, sed s\ae{}viet inter vos ut lupus.'' Cujus somnii significatio secuta in parte, quia futurus abbas studuit magis timeri quam amari, sicut plures dicebant.

\end{otherlanguage}

\switchcolumn

Then on the day on which the thirteen departed, while we were sitting in the cloister, William de Hastings, one of our brothers, said, ``I know that we shall have one of our own number as abbot.'' And when he was asked how he knew this, he answered that he had beheld in dreams a prophet, clothed in white, standing before the gates of the monastery. Him he had asked in the name of the Lord whether we should have one of ourselves as abbot. And the prophet answered, ``Ye shall have one of your own number, but he shall raven as a wolf among you.'' And this dream was partly fulfilled, since he that became abbot strove rather to be feared than loved, as many were wont to say.

\switchcolumn*

\begin{otherlanguage}{latin}
Assedit et alius frater, \AE{}dmundus nomine, asserens quod Samson futurus esset abbas, et narrans visionem quam proxima nocte viderat. Dixit se vidisse in somnis R.\ celerarium et H.\ tertium priorem stantes ante altare, et Samsonem in medio, eminentem ab humeris supra, pallio circumdatum longo et talari, ligato in humeris ejus, et stantem quasi pugilem ad duellum faciendum. Et surrexit sanctus Eadmundus de feretro, sicut ei somnianti visum fuerat, et quasi languidus pedes et tibias nudas exposuit, et accedente quodam et volente operire pedes sancti, dixit sanctus: ``Noli accedere: Ecce! ille velabit mihi pedes,'' pr\ae{}tendens digitum versus Samsonem. H\ae{}c est interpretatio somnii:---Per hoc quod pugil videbatur, significatur quod futurus abbas semper in labore existens, quandoque movens controversiam contra archiepiscopum Cantuariensem de placitis coron\ae{}, quandoque contra milites sancti Eadmundi pro scutagiis integre reddendis, quandoque cum burgensibus pro purpresturis in foro, quandoque cum sochemannis pro sectis hundredorum; quasi pugil volens pugnando superare adversarios, ut jura et libertates ecclesi\ae{} su\ae{} posset revocare. Velavit autem pedes sancti martyris, quando turres ecclesi\ae{} a centum annis inceptas perfecte consummavit.
\end{otherlanguage}

\switchcolumn

Another brother also, Edmund by name, was sitting by, and declared that Samson would be abbot, relating a dream which he had seen on the previous night. For he said that he had seen in dreams Roger the cellarer and Hugh the third prior standing before the altar, and Samson in their midst, head and shoulders taller than they, and wearing a long and flowing cloak, fastened at his shoulders, and he was standing as it were like a champion about to engage in a duel. Then the holy Edmund arose from his shrine ­as it seemed to the brother in his dream ­and showed his feet and legs bare, as though sickness was upon him. Then when one rose and would have covered the feet of the saint, the saint said, ``Come not near. Lo! he shall cover my feet,'' and pointed his finger towards Samson. This is the interpretation of the dream: In that a champion was seen, this signified that he who was to become abbot would be constant in labour, alike when disputing with the archbishop of Canterbury about the pleas of the crown, and when striving with the knights of St.\ Edmund for the full payment of scutages, or with the burghers about encroachments on the market, or with the sokemen for the suits of hundreds; and that he was as it were a champion anxious to overcome his enemies by fighting, that so far as in him lay he might recover the rights and liberties of his church. Moreover, he covered the feet of the holy martyr, when he completed fully the towers of the church, which had been begun a hundred years before.

\switchcolumn*

\begin{otherlanguage}{latin}
Hujusmodi somnia somniabant fratres nostri, qu\ae{} statim divulgabantur, primo per claustrum, postea per curiam, ita quod ante vesperam publice dicebatur in plebe, ille et ille et ille electi sunt, et unus eorum erit abbas.
\end{otherlanguage}

\switchcolumn

Such dreams did our brothers dream, and at once told them first of all in the cloister and then in the court. And so it came to pass that before vespers the people openly said that this and this and this man was elected, and that one of them would be abbot.

\switchcolumn*

\begin{otherlanguage}{latin}
\blockhead[\oldstylenums{1182}, February \oldstylenums{21}.]{How the thirteen came to the king and showed to him the names of those whom the confessors had selected.}{5}{-0.6cm}
Prior autem et xii.\ cum eo post labores et dilationes multas tandem steterunt coram rege apud Waltham,\engnotetext{Bishop's Waltham, in Hampshire.} manerium Wintoniensis episcopi, secunda Dominica quadragesim\ae{}. Quos dominus rex benigne suscepit, et asserens se velle secundum Deum agere et ad honorem ecclesi\ae{} nostr\ae{}, pr\ae{}cepit fratribus per internuncios, scilicet, Ricardum episcopum Wintoniensem\engnotetext{Richard Toclive, archdeacon of Poitiers (\oldstylenums{1162}--\oldstylenums{73}), elected bishop of Winchester in \oldstylenums{1173}; died in \oldstylenums{1188}.} et G.\ cancellarium,\engnotetext{Son of Henry II. by some woman of low birth. He became bishop-elect of Lincoln in \oldstylenums{1173}, but resigned his see and was made chancellor in \oldstylenums{1182}. In \oldstylenums{1189} he secured the archbishopric of York by forgery, as his enemies asserted, and certainly by bribery. From that time his life was one long quarrel with Richard I. and John, and with the chapter of York and Hubert Walter, both as dean of York and as archbishop. Eventually, he fled into exile as the result of his refusal to meet the financial demands of John, and died in Normandy in \oldstylenums{1212}.} postea archiepiscopum Eboracensem, ut nominarent tres de conventu nostro.

\end{otherlanguage}

\switchcolumn

So the prior and the twelve with him, after many labours stood at last in the presence of the king at Waltham,\engnotenum{} a manor of the bishop of Winchester, on the second Sunday in Lent. And the lord king received them graciously, and declared that he wished to act according to the will of God and for the honour of our church. Then he gave command to the brothers through his proctors, Richard bishop of Winchester\engnotenum{} and Geoffrey the chancellor,\engnotenum{} who was afterwards archbishop of York, that they should nominate three members of our monastery.

\switchcolumn*

\begin{otherlanguage}{latin}
Prior vero et fratres se divertentes, quasi inde collocuturi, extraxerunt sigillum et fregerunt et invenerunt h\ae{}c nomina sub tali ordine scripta,---Samson subsacrista, R.\ celerarius, Hugo tertius prior. Erubuerunt inde fratres qui majoris dignitatis erant. Mirabantur etiam omnes eundem Hugonem esse electorem et electum. Quia tamen rem mutare non poterant, ordinem nominum de communi consilio mutaverunt, pronuntiando primum H.\ quia tertius prior erat; secundo R.\ celerarium; tertio Samsonem, facientes verbo tenus novissimum primum et primum novissimum.\engnotetext{Matt.\ xix., \oldstylenums{30}.}
\end{otherlanguage}

\switchcolumn

Then the prior and the brothers withdrew themselves, as it were to discuss this matter, and drew forth the seal and broke it, and found these names written down in this order,---­Samson the sub-sacristan, Roger the cellarer, and Hugh the third prior. And at this the brothers who were of greater dignity blushed. Moreover all marvelled that the same Hugh should be both an elector and one of the elected. But because they could not alter the thing they unanimously changed the order of the names, naming Hugh, because he was third prior, first, and Roger the cellarer next, and Samson third. Thus, as far as words went, they made the last first, and the first last.\engnotenum{}

\switchcolumn*

\begin{otherlanguage}{latin}
Rex vero, primo qu\ae{}rens an nati essent in sua terra, et in cujus dominio, dixit se non nosse eos, mandans ut cum illis tribus alios tres nominarent de conventu.
\end{otherlanguage}

\switchcolumn

But the king, having first asked whether they were born in his land, and in whose lordship, said that he did not know them, and commanded that they should name three other members of the monastery with them.

\switchcolumn*

\begin{otherlanguage}{latin}
\blockhead{How the thirteen, by command of the king, chose three other names from the monastery, and three strangers.}{5}{-0.6cm}
Quo concesso, dixit W.\ sacrista: ``Prior noster debet nominari, quia caput nostrum est:'' quod cito concessum est. Dixit prior: ``W.\ sacrista bonus vir est.'' Similiter dictum est de Dionisio, et concessum est. Quibus nominatis coram rege sine omni mora, mirabatur rex, dicens: ``Cito fecerunt isti. Deus est cum eis.''

\end{otherlanguage}

\switchcolumn

And when this had been granted, William the sacristan said, ``Our prior ought to be nominated, for he is our head,'' and this was readily agreed. Then the prior said, ``William the sacristan is a good man.'' The same was said of Dennis, and was agreed. And when these were named in the presence of the king without any delay, the king marvelled, saying, ``These men act quickly. God is with them.''

\switchcolumn*

\begin{otherlanguage}{latin}
Postea mandavit rex ut propter honorem regni sui nominarent tres personas de aliis domibus. Quo audito, timebant fratres suspicantes dolum. Tandem consilium inierunt ut nominarent tres, sed sub conditione, scilicet, ut nullum reciperent nisi per consilium conventus qui domi fuit. Et nominaverunt tres, magistrum Nicholaum de Waringeford, postea ad horam abbatem de Malmsberi;\footnote[\textdagger]{``This Nicholas, a monk of St.\ Alban's, prior of Wallingford, succeeded Osbert Foliot as abbot of Malmesbury about the year \oldstylenums{1183}, and was deposed in \oldstylenums{1187}.'' So Mr.\ Rokewode, from the Monasticon, ``Ad horam'' must therefore mean ``for a season.''}\engnotetext{A monk of St.\ Alban's, prior of Wallingford. He became abbot of Malmesbury about \oldstylenums{1163}, but was deposed in \oldstylenums{1187}. (Rokewode, p.\ \oldstylenums{114}.)} et Bertrandum priorem Sanct\ae{} Fidis, postea abbatem de Certeseia;\footnote[\ddag]{``Bertrand [Bertan] succeeded Aymer, abbot of Chertsey'' (Roke.); his successor, Martin, was elected in \oldstylenums{1197}.} et dominum H.\ de Sancto Neoto, monachum de Becco,\engnotetext{There was a Herbert, prior of St.\ Neot's in \oldstylenums{1159} and in \oldstylenums{1173}. (Rokewode, p.\ \oldstylenums{114}.)} virum admodum religiosum et in temporalibus et spiritualibus admodum circumspectum.
\end{otherlanguage}

\switchcolumn

And after that the king commanded that for the honour of the kingdom, they should nominate three persons from other houses. When they heard this the brothers feared, for they suspected a fraud. Yet did they agree to nominate three, but under conditions, namely, that they would receive no one save with the assent of the members of the monastery who were at home. And they named three, master Nicholas de Waringford,\engnotenum{} who was afterwards for awhile abbot of Malmesbury, and Bertrand, prior of St.\ Faith's, who was afterwards abbot of Chertsey, and lord H.\ de St.\ Neots,\engnotenum{} a monk of Bec, a most pious man, and in both secular and spiritual matters very prudent.

\switchcolumn*

\begin{otherlanguage}{latin}
\blockhead{How the list of names was reduced from nine to two.}{3}{-0.5cm}
Quo facto mandavit rex, gratias agens, ut tres removerentur de novem, et statim remoti sunt alieni tres, scilicet prior Sanct\ae{} Fidei, postea Certeseiensis abbas, et Nicholaus monachus Sancti Albani, postea abbas Malmsberiensis, et prior Sancti Neoti.

\end{otherlanguage}

\switchcolumn

When this had been done, the king sent them thanks and commanded that three of the nine should be removed, and the three strangers were at once removed, that is, the prior of St.\ Faith's, who was afterwards abbot of Chertsey, and Nicholas, the monk of St.\ Alban's, who was afterwards abbot of Malmesbury, and the prior of St.\ Neot's.

\switchcolumn*

\begin{otherlanguage}{latin}
Willelmus sacrista sponte cessit; remoti sunt duo ex quinque per pr\ae{}ceptum regis; et postea unus ex tribus, et remanserunt tum duo, scilicet, prior et Samson.
\end{otherlanguage}

\switchcolumn

William the sacristan of his accord withdrew, two of the five were removed by the order of the king, and finally one of the last three, so that there remained then two, namely, the prior and Samson.

\switchcolumn*

\begin{otherlanguage}{latin}
\blockhead{How Samson was elected abbot.}{3}{-0.5cm}
Tunc tandem vocati sunt ad consilium fratrum pr\ae{}nominati internuntii domini regis. Et loquens Dionisius, unus pro omnibus, c\oe{}pit commendare personas prioris et Samsonis, dicens utrumque eorum literatum, utrumque bonum, utrumque laudabilem vit\ae{} et integr\ae{} opinionis, sed semper in angulo sui sermonis Samsonem protulit, multiplicans verba in laudem ejus, dicens eum esse virum rigidum in conversatione, severum in corrigendis excessibus, et aptum ad labores, et in s\ae{}cularibus curis prudentem, et in diversis officiis probatum.

\end{otherlanguage}

\switchcolumn

Then at last the above-­mentioned proctors of the lord king were summoned to the council of the brothers. And Dennis, speaking as one for all, began to commend the persons of the prior and Samson. He said that they were both learned men, both good, both praiseworthy in their lives and of unblemished reputation. But ever at the climax of his speech he put forward Samson, multiplying words in his praise, saying that he was a man strict in his conduct, stern in correcting faults, apt for labour, prudent in temporal matters, and proved in divers offices.

\switchcolumn*

\begin{otherlanguage}{latin}
Respondit Wintoniensis: ``Bene intelligimus quod vultis dicere; ex verbis vestris conjicimus quod prior vester vobis videtur aliquantulum remissus, et illum qui Samson dicitur vultis habere.'' Respondit Dionisius: ``Uterque bonus est, sed meliorem vellemus habere si Deus vellet.'' Cui episcopus: ``Duorum bonorum magis bonum eligendum est: dicite aperte, vultis habere Samsonem?'' Et responsum est pr\ae{}cise a pluribus et a majori parte, ``Volumus Samsonem,'' nullo reclamante, quibusdam tamen tacentibus ex industria, nec hunc nec illum offendere volentibus.
\end{otherlanguage}

\switchcolumn

Then the bishop of Winchester answered, ``We know well what you would say, from your words we gather that your prior has appeared to you to be somewhat slack, and that you wish to have him who is called Samson.'' Dennis answered, ``Both of them are good men, but we desire to have the better, if God wills.'' Thereupon the bishop said, ``Of two good things, the greater good should be selected. Say openly, do you desire to have Samson?'' And many, and they a majority, answered plainly, ``We wish to have Samson,'' and none spoke against him. Some, however, were silent from caution, wishing to offend neither candidate.

\switchcolumn*

\begin{otherlanguage}{latin}
Nominato Samsone, coram domino rege, et habito brevi consilio cum suis, vocati sunt omnes, et dixit rex: ``Vos pr\ae{}sentastis mihi Samsonem: non novi eum: si pr\ae{}sentaretis mihi priorem vestrum, illum reciperem quem vidi et agnovi; sed modo faciam quod vultis. Cavete vobis; per veros oculos Dei, si male feceritis, ego me capiam ad vos.''
\end{otherlanguage}

\switchcolumn

Then Samson was nominated in the presence of the lord king, and when the king had consulted with his men for a while, all were summoned. And the king said, ``You have presented to me Samson. I know him not. If you had presented your prior to me, I would have accepted him, for I have seen and know him. But I will only do what you will. Take heed to yourselves; by the true eyes of God, if you do ill, I will exact a recompense at your hands.''

\switchcolumn*

\begin{otherlanguage}{latin}
Et interrogavit priorem si assentiret, et hoc vellet; qui respondit se hoc velle, et Samsonem multo majore dignum honore. Electus igitur, ad pedes regis procidens et deosculans, festinanter surrexit et festinanter ad altare tetendit, cantando: ``Miserere mei Deus,'' cum fratribus, erecto capite, vultu non mutato.
\end{otherlanguage}

\switchcolumn

Then he asked the prior if he assented to the choice and wished it, and the prior answered that he did will it, and that Samson was worthy of much greater honour. Therefore he was elected, and fell at the king's feet and embraced them. Then he arose quickly and hastened to the altar, with his head erect and without changing his expression, chanting the ``Miserere mei, Deus''\engnotetext{Ps.\ \oldstylenums{50}, Vulgate.}\engnotenum{} with the brothers.

\switchcolumn*

\begin{otherlanguage}{latin}
Quod cum rex vidisset, dixit astantibus: ``Per oculos Dei,\footnote[\dag]{These strange oaths were constantly in the mouths of the Anglo-Norman and Angevin kings. William Rufus used to swear by ``the holy Face of Lucca.''} iste electus videtur sibi dignus abbati\ae{} custodiend\ae{}.''
\end{otherlanguage}

\switchcolumn

And when the king saw this, he said to those that stood by, ``By the eyes of God, this elect thinks that he is worthy to rule the abbey.''

\switchcolumn*

\begin{otherlanguage}{latin}
\blockhead[\textsc{a.d}.\ \oldstylenums{1182}, February \oldstylenums{28}.]{How the news of the election came to the monastery and how Samson was blessed.}{5}{-0.6cm}
Hujus electionis rumor cum ad conventum perveniret, omnes claustrales vel fere omnes, et quosdam obedientiales, sed paucos, l\ae{}tificavit: ``Bene,'' multi dicebant, ``quia bene est.'' Alii dicebant quod ``non; imo, omnes seducti sumus.''

\end{otherlanguage}

\switchcolumn

The news of this election came to the monastery, and all the cloistered monks or almost all of them were rejoiced, and also some of the officials, but few. ``It is well,'' said many, ``because it is well.'' Others said that this was not so, ``Of a truth, we have all been bewitched.''

\switchcolumn*

\begin{otherlanguage}{latin}
Electus, antequam rediret ad nos, benedictionem\engnotetext{At Merewell, near Newport, I.W. He received his benediction from Richard of Winchester and Augustine, bishop of Waterford, on February \oldstylenums{28}th (\emph{Chron.\ Bur., Mem}.\ III., \oldstylenums{7}, and \emph{Ann.\ St.\ Ed., Mem}.\ II., \oldstylenums{5}). From the same sources we learn that Samson was received at St.\ Edmund's on Palm Sunday, March \oldstylenums{21}st.} suam accepit a domino Wintoniensi, qui in eadem hora mitram capiti abbatis imponens et annulum digito, ait: ``H\ae{}c est dignitas abbatum sancti Eadmundi: diu est ex quo scivi hoc.''
\end{otherlanguage}

\switchcolumn

Before he returned to us, the elect received his benediction\engnotenum{} from the lord of Winchester, who in the same hour in the which he placed the mitre on the abbot's head and the ring on his finger, said, ``This man is worthy of the abbacy of St.\ Edmund, and for a long while have I known it.''

\switchcolumn*

\begin{otherlanguage}{latin}
Abbas itaque tres monachos secum retinens, alios domum pr\ae{}misit, nuncians adventum suum Dominica Palmarum, quibusdam commendans curam ad providenda necessaria in die festi sui.
\end{otherlanguage}

\switchcolumn

Therefore the abbot retained with him three monks, and allowed the rest to return home. And he announced that he would himself come on Palm Sunday, and charged certain men with the care of providing those things which might be necessary for his feast.

\switchcolumn*

\begin{otherlanguage}{latin}
\blockhead[\textsc{a.d}.\ \oldstylenums{1182}, March \oldstylenums{21}.]{How Samson, having been made abbot, returned and was received at the monastery.}{5}{-0.6cm}
Redeunti multitudo novorum parentum occurrit, volentium ei servire; qui omnibus respondit, se esse contentum servientibus prioris, nec alios posse retinere, donec inde consuluisset conventum suum. Unum tamen militem retinuit eloquentem et juris peritum, non tantum consideratione proximitatis, sed ratione utilitatis, causis quidem s\ae{}cularibus assuetum; quem suscepit in novitate sua quasi coadjutorem in mundanis controversiis, quia novus abbas erat, et rudis in talibus, sicut ipsemet protestatus est: quia nunquam ante susceptam abbatiam loco interfuit ubi datum esset vadium et plegium.

\end{otherlanguage}

\switchcolumn

On his homeward way a multitude of new relations met him, desiring to serve him. But he answered all of them that he was content with the servants of the prior, and that he was unable to maintain others until he had consulted the monastery on the matter. But one knight he did retain, a man who was eloquent and skilled in the law. This he did not only from consideration of their relationship, but from arguments of utility also, as he was indeed used to secular affairs. He received him as a novice and as his assessor in temporal disputes. For he was a new abbot and unskilful in such matters, as he himself protested, since until he received the abbacy he had never held any office in which surety and pledge was given.

\switchcolumn*

\begin{otherlanguage}{latin}
Cum debito honore et etiam processione receptus est a conventu suo, Dominica Palmarum.
\end{otherlanguage}

\switchcolumn

On Palm Sunday he was received with due honour and with ceremony also by his monastery.

\switchcolumn*

\begin{otherlanguage}{latin}
Susceptus est autem dominus abbas hoc modo: proxima nocte jacuerat apud Chenteford, et accepta temporis opportunitate, ivimus contra eum solempniter, post exitum de capitulo, usque ad portam cimiterii, sonantibus campanis in choro et extra. Ipse vero multitudine hominum constipatus, videns conventum, descendit de equo extra limen port\ae{}, et faciens se discalciari, intra portam nudipes susceptus est, priore et sacrista hinc et inde ducentibus eum. Nos vero cantavimus responsoria, ``Benedictus Dominus,''\engnotetext{Response at matins after the second lesson on Trinity Sunday. (Rokewode, p.\ \oldstylenums{18}.)} de Trinitate, et post, ``Martiri adhuc,''\engnotetext{A response following the sixth lesson at matins on St.\ Edmund's day. Rokewode (p.\ \oldstylenums{115}) gives the musical notes which accompany it from a life of the saint of the time of abbot Anselm. Rokewode seems to suggest that this was one of the antiphons composed in honour of St.\ Edmund by Warner, abbot of Rebaix, for which see Hermannus, \emph{De Mir.\ St.\ Eadmundi} (\emph{Mem}.\ I., \oldstylenums{69}--\oldstylenums{70}).} de sancto Eadmundo; ducentes abbatem usque ad magnum altare.
\end{otherlanguage}

\switchcolumn

Now the lord abbot was thus received. The night before he had lain at Kentford, and at the proper moment we went to meet him in the solemn procession, after leaving the chapter, as far as the gate of the graveyard, while bells were rung in the choir and outside it. But he was surrounded by a multitude of men, and when he saw the monastery, dismounted from his horse without the threshold of the gate, and causing his sandals to be removed, was received within the door barefooted, the prior and the sacristan supporting him on either side. And we chanted the responses ``Benedictus Dominus''\engnotenum{} from the service for Trinity Sunday, and afterwards the ``Martiri adhuc''\engnotenum{} from that for St.\ Edmund, and conducted the abbot as far as the high altar.

\switchcolumn*

\begin{otherlanguage}{latin}
Quibus peractis, siluerunt et organa et campan\ae{}, et dicta oratione a priore, ``Omnipotens sempiterne Deus, miserere huic,''\footnote[\textdagger]{Collect. Miss\ae{} Votiv\ae{} lxxi.\ in Sacrament. Gregor.---Liturgia Romana Vetus. Muratori, tom.\ ii., p.\ \oldstylenums{90}. (Rokewode.)}\engnotetext{Liturgia Romana Vetus. (Rokewode, p.\ \oldstylenums{18}, note \oldstylenums{4}.)} \&c. super abbatem prostratum, et facta oblatione ab abbate, et deosculato feretro, rediit in chorum, et ibi recepit eum Samson cantor per manum, et duxit usque ad sedem abbatis ad occidentalem partem, ubi, eo stante, in directum incepit cantor: ``Te Deum laudamus;'' quod dum decantabatur, deosculatur a priore et a toto conventu per ordinem.
\end{otherlanguage}

\switchcolumn

And when this had been done, the organs and bells were silenced, and the prior said the prayer ``Omnipotens sempiterne Deus, miserere huic,''\engnotenum{} over the prostrate abbot. Then the abbot made oblation and kissed the shrine, and returned to the choir. There Samson the precentor took him by the hand and led him to the abbot's chair on the western side of the choir, and while he stood there the precentor at once began ``Te Deum laudamus,'' and while it was being chanted, the abbot was embraced by the prior and by the whole monastery.

\switchcolumn*

\begin{otherlanguage}{latin}
Quibus expletis, ivit abbas in capitulum, sequente conventu et multis aliis. Dicto autem ``Benedicite,'' in primis gratias egit conventui quod eum, ut aiebat, minimum eorum, non suis meritis, sed sola Dei voluntate, in dominum et pastorem elegerunt. Rogansque breviter ut orarent pro eo, convertit sermonem ad clericos et milites, rogans, ut eum consulerent ad solicitudinem commissi regiminis.
\end{otherlanguage}

\switchcolumn

And so, these ceremonies being completed, the abbot entered the chapter, the whole monastery and many others following. He said many times ``Benedicite,'' and then he first returned thanks to the monastery that they had chosen him, the least of them all, as he said, not for his own merits but only by the will of God, to be their lord and pastor. And asking in a few words that they would pray for him, he addressed the clerks and knights, and asked them to advise him for the good of the monastery.

\switchcolumn*

\begin{otherlanguage}{latin}
Et respondens Wimerus\engnotetext{Rokewode (p.\ \oldstylenums{116}) collects from the Pipe Rolls of Norfolk and Suffolk that Wimer was a sheriff of Norfolk and Suffolk in conjunction with Bartholomew Glanvill, and William Bardolf from the \oldstylenums{16}th to the \oldstylenums{22}nd years of Henry II.; and sheriff alone from then to the \oldstylenums{34}th year of Henry II.} vicecomes pro omnibus, dixit: ``Et nos parati sumus vobis consistere in consilio et auxilio omnibus modis, sicut caro domino quem Dominus vocavit ad honorem suum et ad honorem sancti martyris Eadmundi.''
\end{otherlanguage}

\switchcolumn

Then Wimer, the sheriff,\engnotenum{} answered for them all, and said, ``We also are ready to be with you in counsel and in helping you in every way, as with a dear lord whom the Lord has called for His honour, and for the honour of the holy martyr Edmund.''

\switchcolumn*

\begin{otherlanguage}{latin}
Et deinde extract\ae{} sunt cart\ae{} regis, et lect\ae{} in audientia de donatione abbati\ae{}. Facta autem oratione ab ipso abbate, ut ei Deus consuleret secundum gratiam suam, et responso ``amen'' ab omnibus, ivit in talamum suum, diem festivum agens cum plus quam mille comedentibus, in gaudio magno.
\end{otherlanguage}

\switchcolumn

Afterwards the charters of the king concerning the donation of the abbacy were brought forth, and were read in the hearing of all. The abbot himself also prayed that God would guide him according to His grace, and all answered ``Amen.'' Then he went into his own chamber, and celebrated his day of festival with more than a thousand guests and with great joy.

\switchcolumn*

\begin{otherlanguage}{latin}
\blockhead{How abbot Samson began to rule the monastery.}{3}{-0.55cm}
Quando h\ae{}c fiebant, eram capellanus pr-ioris, et infra quatuor menses capellanus abbatis factus, plurima notans et memori\ae{} commendans. In crastino ergo festi sui, convocavit priorem et alios quosdam paucos, quasi consilium ab aliis qu\ae{}rens: ipse enim sciebat quid esset facturus.

\end{otherlanguage}

\switchcolumn

In those days I was prior's chaplain, and within four months was made chaplain to the abbot. And I noted many things and committed them to memory. So, on the morrow of his feast, the abbot assembled the prior and some few others together, as if to seek advice from others, but he himself knew what he would do.

\switchcolumn*

\begin{otherlanguage}{latin}
Dixit novum sigillum\footnote[\textdagger]{As the frontispiece to his edition of the Chronicle, Mr.\ Gage Rokewode gives an engraving of an impression from this seal, answering to the description in the text, which is still attached to a document among the archives of Christ Church, Canterbury.}\engnotetext{A reproduction of this seal appears as a frontispiece to Rokewode's edition.} esse faciendum, et cum mitra esse pingendum, licet pr\ae{}decessores sui tale non haberent; sigillo autem prioris nostri hucusque usus fuerat, singulis literis in fine subscribens, quod proprium sigillum non habuit, unde et sigillo prioris oportuit uti ad tempus. Postea disponens domui su\ae{},\engnotetext{John vi., \oldstylenums{6}.} diversos famulos diversis officiis deputavit, dicens se pr\ae{}cogitasse viginti sex equos in curia sua habendos; et ad plus asserens, ``puerum prius repere, postea firmius stare et ire;'' hoc super omnia famulis pr\ae{}cipiens, ut caverent ne in novitate sua possit infamari avaritia cibi vel potus, sed hospitalitatem domus solicite procurarent.
\end{otherlanguage}

\switchcolumn

He said that a new seal\engnotenum{} must be made and adorned with a mitred effigy of himself, though his predecessors had not had such a seal. For a time, however, he used the seal of our prior, writing at the end of all letters that he did so for the time being because he had no seal of his own. And afterwards he ordered his household,\engnotenum{} and transferred various officials to other offices, saying that he proposed to maintain twenty-six horses in his court, and many times he declared that ``a child must first crawl, and afterwards he may stand upright and walk.'' And he laid this especial command upon his servants, that they should take care that he might not be laid open to the charge of not providing enough food and drink, but that they should assiduously provide for the maintenance of the hospitality of the house.

\switchcolumn*

\begin{otherlanguage}{latin}
In his et in omnibus rebus agendis et constituendis de Dei auxilio et proprio sensu plenius confidens, inglorium duxit de alieno pendere consilio, tanquam ipse sibi sufficeret. Mirabantur monachi, indignabantur milites; damnantes eum arrogantia, et quodammodo infamantes eum apud curiam regis, et dicentes quia nollet operari secundum consilium suorum liberorum hominum. Ipse magnates abbati\ae{}, tam laicos quam literatos, sine quorum consilio et auxilio abbatia videbatur non posse regi, omnes a privato suo elongavit consilio; et hac occasione Ranulfus de Glanvill, justiciarius Angli\ae{}, primo eum suspectum hahebat, et minus propitius ei erat quam deceret, donec ei certis indiciis constaret abbatem tam in interioribus quam exterioribus negotiis provide et prudenter agere.
\end{otherlanguage}

\switchcolumn

In these matters, and in all the things which he did and determined, he trusted fully in the help of God and his own good sense, holding it to be shameful to rely upon the counsel of another, and thinking he was sufficient unto himself. The monks marvelled and the knights were angered; they blamed his pride, and often defamed him at the court of the king, saying that he would not act in accordance with the advice of his freemen. He himself put away from his privy council all the great men of the abbey, both lay and literate, men without whose advice and assistance it seemed impossible that the abbey could be ruled. For this reason Ranulf de Glanvill, justiciar of England, was at first offended with him, and was less well-disposed towards him than was expedient, until he knew well from definite proofs that the abbot acted providently and prudently, both in domestic and in external affairs.

\switchcolumn*

\begin{otherlanguage}{latin}
\blockhead[\textsc{a.d}.\ \oldstylenums{1182} April 1.]{How the abbot met the demand of Thomas de Hastings that his nephew should be steward.}{5}{-0.6cm}
Facta summonitione generali, conveniunt omnes barones et milites et liberi homines ut facerent homagium quarto die Pasch\ae{}: et ecce! Thomas de Hastingo, cum magna multitudine militum, ducens Henricum nepotem\engnotetext{Rokewode (p.\ \oldstylenums{116} ff) prints the charters upon which Henry de Hastings claimed to be hereditary steward of the liberty of St.\ Edmund's.} suum nondum militem, clamans senescaldiam\footnote[\textdagger]{A seneschal (\emph{senscalcus}, \emph{senescallus}, \emph{senescaldus}) originally meant the servant placed over the household, as ``marshal'' (\emph{mariscalcus}) meant the servant placed over the horses. The origin of the first part of the word is unknown; probably it is some Teutonic word like \emph{sin}, meaning ``old.'' In the twelfth century the seneschal generally administered justice for his lord, and this was the case at Bury, as Samson's remarks on Gilbert's appointment show.} cum consuetudinibus suis, sicut carta ejus loquitur. Quibus abbas statim respondit: ``Henrico non nego jus suum, nex negare volo. Si sciret in propria persona mihi servire, concederem ei et decem hominibus et octo equis necessaria in mea curia, sicut carta ejus loquitur; si pr\ae{}sentetis mihi senescaldum, vicarium ejus, qui sciat et possit senescaldiam regere, recipiam eum in tali statu, sicut pr\ae{}decessor meus eum habuit die quo fuit vivus et mortuus, scilicet cum iiij$^\text{or}$ equis et pertinentiis. Quod si nolueritis, pono loquelam coram rege vel coram capitali justicia.''

\end{otherlanguage}

\switchcolumn

A general summons was sent out, and on the fourth day of Easter all the barons, and knights, and freemen came to do homage. And, lo! Thomas de Hastings came also with a great multitude of knights, and brought with him Henry his nephew,\engnotenum{} who was not yet a knight, and for whom he demanded the office of steward with the customary dues thereof, as his charter provided. And to this demand the abbot at once answered, ``I neither deny, nor wish to deny, his right to Henry. If he were able to serve me in his own person, I would grant him the means of supporting ten men and eight horses in my court. And if you will present a steward to me, who knows how to fulfil the office of steward and is able to do so, I will receive him on the same terms as my predecessor had his steward on the day whereon he was alive and dead, that is, I will allow him four horses with the things needful for them. But if you will not agree to this, I will make complaint before the king and before the chief justiciar.''

\switchcolumn*

\begin{otherlanguage}{latin}
Quo dicto, cepit res dilationem: postea pr\ae{}sentatus est ei quidam senescaldus simplex et idiota, Gilbertus nomine, quem antequam suscepisset, dixit familiaribus suis: ``Si defectus fuerit de justitia regis servanda per inscientiam senescaldi, ipse erit in misericordia regis et non ego, quia senescaldiam vendicat sibi jure h\ae{}reditario: et ideo ad pr\ae{}sens malo istum recipere, quam alium magis argutum ad me decipiendum. Ego mihi ero senescaldus cum auxilio Dei.''
\end{otherlanguage}

\switchcolumn

When the abbot had so spoken, the matter was postponed. But afterwards a certain simple and foolish steward, by name Gilbert, was presented to him, and before he received him, the abbot said to his intimates, ``If the ignorance of the steward leads to the ill-rendering of the justice of the king, then it will be he who will be responsible to the king and not I, for he gained the stewardship by hereditary right. For the time, therefore, I would rather accept him than another, even more incompetent, to my loss. By God's help, I will be my own steward.''

\switchcolumn*

\begin{otherlanguage}{latin}
\blockhead{How the abbot dealt with the lands of his house.}{3}{-0.55cm}
Post homagia suscepta, petivit abbas auxilium a militibus, qui promiserunt ab unoquoque xx.\ solidos; sed in instanti inierunt consilium, et retraxerunt duodecim libras de duodecim militibus, dicentes, quod illi xii.\ debent adjuvare alios xl.\ et ad wardas faciendas et ad scutagia, similiter et ad auxilium abbatis. Quod cum abbas audisset, iratus est, et dixit familiaribus suis, quod, si posset vivere, redderet eis vicem pro vice et gravamen pro gravamine.\engnotetext{For the way in which this threat was carried out, see text, p.\ \oldstylenums{104}.}

\end{otherlanguage}

\switchcolumn

When homage had been received, the abbot demanded an aid from the knights, and they promised twenty shillings from each fee of a knight. But they at once took counsel, and reduced the aid by twelve pounds from twelve knights, alleging that these twelve ought to assist the other forty to keep ward, and to make scutages, and also in assisting the abbey. When the abbot heard this, he was wroth, and said to his friends that should his life be spared, he would repay them like for like, and injury for injury.\engnotenum{}

\switchcolumn*

\begin{otherlanguage}{latin}
Post h\ae{}c, per unumquodque manerium abbati\ae{} fecit abbas inquiri annuos census liberorum hominum, et nomina rusticorum, et eorum tenementa, et singulorum servitia, et in scriptum omnia redigi. Aulas autem veteres et domos confractas, per quas milvi et cornices volabant, reformavit; capellas novas \ae{}dificavit, et talamos et solia pluribus locis, ubi nunquam fuerunt \ae{}dificia, nisi horrea solummodo.
\end{otherlanguage}

\switchcolumn

After this, the abbot caused inquest to be made in every manor belonging to the abbacy as to the annual revenues of the free men, and the names of the villeins, and their holdings, and the services due from each, and caused all these details to be written down. Then he restored the old halls and ruined houses, through which kites and crows flew; he built new chapels, and rooms and seats in many places where there had never been buildings, save perhaps barns.

\switchcolumn*

\begin{otherlanguage}{latin}
Plures etiam parcos fecit, quos bestiis replevit, venatorem cum canibus habens; et, superveniente aliquo hospite magni nominis, sedebat cum monachis suis in aliquo saltu nemoris, et videbat aliquando canes currere; sed de venatione nunquam vidi eum gustare.
\end{otherlanguage}

\switchcolumn

He also made many parks, which he filled with beasts, and had a huntsman and dogs. And whenever any important guest arrived, he used to sit with his monks in some retired grove, and watch the coursing for a while; but I never saw him interested in hunting.

\switchcolumn*

\begin{otherlanguage}{latin}
Plura etiam assartavit et in agriculturam reduxit; in omnibus utilitati abbati\ae{} prospiciens: sed utinam super maneriis conventus commendandis consimili studio vigilaret! Maneria tamen nostra de Bradefeld et Rutham recepit ad tempus in manu sua, implens defectus firmarum per expensam xl.\ librarum, qu\ae{} postea resignavit nobis, audito quod murmur erat in conventu ex hoc quod maneria nostra tenuit in manu sua.
\end{otherlanguage}

\switchcolumn

He made many clearings and brought land into cultivation, in everything regarding the advantage of the abbacy. But would that he had watched with equal care over the grants of the manors of the monastery. For he received our manors of Bradfield and Rougham for a while into his own hand, making good the loss of rent by the expenditure of forty pounds, which he afterwards handed over to us when he heard that the monastery murmured because he held our manors in his own hands.


\switchcolumn*

\begin{otherlanguage}{latin}
Eisdem etiam maneriis et omnibus aliis regendis, tam monachos quam laicos sapientiores prioribus custodibus constituit, qui et nobis et terris nostris consultius providerent.
\end{otherlanguage}

\switchcolumn

For the management of the same manors and for the management of all other affairs, he appointed monks and laymen who were wiser than those who had previously held the posts, and who made careful provision for us and our lands.

\switchcolumn*

\begin{otherlanguage}{latin}
Octo etiam hundredos\footnote[\textdagger]{``Tenuit'' desiderari videtur. (Roke.)} in manu sua, et post mortem Roberti de Cokefeld recepit hundredum de Cosford, quos omnes servientibus suis de mensa sua custodiendos tradidit; qu\ae{} majoris qu\ae{}stionis erant ad se referens, et qu\ae{} minoris per alios terminans, et singula ad suum commodum retorquens.
\end{otherlanguage}

\switchcolumn

Then he received eight hundreds into his own hands, and when Robert de Cokefield died, he took the hundred of Cosford. All these he handed over to the care of the servants of his own table. Matters of greater moment were kept for his own decision, and those which were of less import were decided by his agents; all things he turned to his advantage.

\switchcolumn*

\begin{otherlanguage}{latin}
Facta, eo jubente, descriptio generalis per hundredos de letis et sectis, de hidagiis et fodercorn, de gallinis reddendis, et aliis consuetudinibus et redditibus et exitibus, qui in magna parte semper celati fuerant per firmarios, et omnia redegit in scriptum, ita quod, infra iiij$^\text{or}$ annos ab electione sua, non erat qui posset eum decipere de redditibus abbati\ae{} ad valentiam unius denarii, cum de abbatia custodienda nullum scriptum a predecessoribus suis recepisset, nisi schedulam parvam, qua continebantur nomina militum Sancti Eadmundi et nomina maneriorum, et qu\ae{} firma quam firmam sequi deberet. Hunc autem librum vocavit Kalendarium\engnotetext{Rokewode (p.\ \oldstylenums{121}) mentions a transcript of the Calendar which was in a copy of the \emph{Liber de Consuetudinibus S.\ Edmundi} in his possession.} suum, quo etiam inscribebantur singula debita qu\ae{} adquietaverat; quem librum fere quotidie inspexit, tanquam ibi consideraret vultum probitatis su\ae{} in speculo.
\end{otherlanguage}

\switchcolumn

By his command, a general account was drawn up for every hundred of the leets and suits, of the hidages and customary supplies of fodder, of the hens which ought to be paid to him, and of all the other customary dues, revenues, and expenses, which the tenants had always concealed to a great extent. All these things he reduced to writing, so that within four years of his election, no one could deceive him as to the resources of the abbey even to a penny's value, whereas he had received nothing in writing from his predecessors concerning the management of the abbey, except a little schedule containing the names of the knights of St.\ Edmund and the names of the manors, and the rent which attached to each farm. Now he called this book of his his Calendar,\engnotenum{} in the which also were written down all the debts which he had paid. And he consulted this book almost daily, as if in it he saw the image of his probity as in a glass.

\switchcolumn*

\begin{otherlanguage}{latin}
\blockhead{Of that which was done at the abbot's first chapter.}{3}{-0.55cm}
Prima die qua tenuit capitulum, confirmavit nobis novo sigillo suo lx.\ solidos de Suthreia, quos pr\ae{}decessores sui injuste receperant primo ab Eadmundo, aureo monacho dicto, ut posset tenere eandem villam ad firmam omnibus diebus vit\ae{} su\ae{}.

\end{otherlanguage}

\switchcolumn

On the first day on the which he held a chapter, he confirmed to us under his new seal the sixty shillings for Southrey, which his predecessors had in the first instance unjustly received from Edmund, called the golden monk, that the same might hold the said township to farm all the days of his life.

\switchcolumn*

\begin{otherlanguage}{latin}
Et proposuit edictum ut nullus de c\ae{}tero ornamenta ecclesi\ae{} invadiaret sine assensu conventus, sicut solebat fieri, nec aliqua carta sigillaretur sigilio conventus nisi in capitulo coram conventu;
\end{otherlanguage}

\switchcolumn

And he proposed an edict that no one should pledge the ornaments of the church henceforth without the assent of the monastery, as had been done formerly. He proposed also that no charter should be sealed with the seal of the monastery save in the chapter and in the presence of the whole community.

\switchcolumn*

\begin{otherlanguage}{latin}
et fecit Hugonem subsacristam, statuens ut Willelmus sacrista nichil omnino ageret de sacristia, nec in receptis nec in expensis, nisi per assensum ejus. Posth\ae{}c, sed non eodem die, antiquos custodes oblationum transtulit ad alia officia. Postremo ipsum W.\ deposuit; unde quidam diligentes Willelmum dicebant: ``Ecce abbas! ecce lupus de quo somniatum est! ecce qualiter s\ae{}vit:''
\end{otherlanguage}

\switchcolumn

Then he made Hugh sub-sacristan, ordaining that William the sacristan should do nothing in the office of sacristan, either as to receipts or as to expenses, save by his assent. Afterwards, but on the same day, he removed the former custodians of the oblations to other offices. And last of all he deposed William himself, whereupon certain who loved William said, ``See the abbot! See the wolf of whom one dreamed! See how he ravens!''

\switchcolumn*

\begin{otherlanguage}{latin}
\blockhead{How certain men wished to conspire against the abbot.}{3}{-0.55cm}
et voluerunt facere quidam conspirationem contra abbatem. Quod cum abbati revelatum esset, volens nec omnino tacere nc conventum turbare, intravit capitulum in crastino, extrahens sacculum plenum cartis cancellatis adhuc sigillis pendentibus, scilicet, pr\ae{}decessoris sui, et partim prioris, partim sacrist\ae{}, partim camerarii, et aliorum officialium, quarum summa erat trium millium librarum et lii., et una marca, de pura sorte, pr\ae{}ter usuram qu\ae{} excreverat, cujus magnitudo nunquam sciri poterat: de quibus omnibus pacem fecerat infra annum post electionem suam, et infra xii.\ annos omnia adquietavit.

\end{otherlanguage}

\switchcolumn

Then some wished to conspire against the abbot. And when this was revealed to the abbot, as he wished neither to keep silence altogether nor to disturb the monastery, he entered the chapter on the morrow. And there he drew forth a small bag full of cancelled bonds, to which the seals were still attached, some of which were those of his predecessor, some of the prior, some of the sacristan, some of the chamberlain, and some of other officials. Of these the total was three thousand two score and twelve pounds, and one pure mark, over and above the increase due to usury, the amount of which none could know; and for all these he had made some arrangement within a year of his election, and within twelve years he had paid them in full.

\switchcolumn*

\begin{otherlanguage}{latin}
``Ecce,'' inquit, ``sapientia sacrist\ae{} nostri Willelmi! Ecce tot cart\ae{} sigillo ejus signat\ae{}, cum quibus impignoraverat cappas sericas, dalmaticas, turibula argenti, et textus aureos, sine conventu, qu\ae{} omnia adquietavi et vobis reconsignavi:'' et multa alia adjecit, ostendens quare deposuerat W.; pr\ae{}cipuam tamen causam subticuit, nolens eum scandalizare.
\end{otherlanguage}

\switchcolumn

And he said, ``Observe the wisdom of our sacristan, William! See the number of bonds marked with his seal, in which he has pledged silken caps, dalmatics, silver vases, and books of the gospels bound in gold, without the assent of the monastery. And all these things I have settled and restored to you.'' And he added many other words, showing wherefore he had deposed William. But on the principal cause he kept silence, not wishing to make him a public example.

\switchcolumn*

\begin{otherlanguage}{latin}
Et cum substituisset Samsonem cantorem, nobis omnibus placentem et omni exceptione majorem, in pace facta sunt omnia. Abbas vero domos sacrist\ae{} in cimiterio funditus pr\ae{}cepit erui, tanquam non essent dign\ae{} stare super terram, propter frequentes bibationes et qu\ae{}dam tacenda, qu\ae{} nolens et dolens viderat quando fuit subsacrista; et ita omnia complanari fecit, quod infra annum, ubi steterat nobile \ae{}dificium, vidimus fabas pullulare, et ubi jacuerant dolia vini, urticas abundare.
\end{otherlanguage}

\switchcolumn

And all things became peaceful once more when he had replaced William with the precentor Samson, who was a man pleasing to all of us, and well known to be without fault. But the abbot ordered that the houses of the sacristan in the graveyard should be utterly destroyed, as if they were unworthy to stand above ground. And for this the cause was the frequent drinking bouts and certain things which cannot be mentioned, which he had seen when he was sub-sacristan with sorrow and pain. So he caused all the buildings to be levelled with the ground, and within a year, where there had stood a noble building we saw beans growing, and where casks of wine had lain we saw nettles in abundance.

\switchcolumn*

\begin{otherlanguage}{latin}
\blockhead{How the abbot journeyed through the lands of Saint Edmund, and how he escaped death at Warkton.}{4}{-0.65cm}
Post clausum Pasch\ae{} ivit abbas per singula maneria sua et nostra, et per illa qu\ae{} confirmavimus in feudum firmariis, poscens ab omnibus et a singulis auxilium et recognitionem secundum consuetudinem regni; quotidie s\ae{}culari scientia proficiens, et ad exteriora negotia discenda et promovenda animum convertens.

\end{otherlanguage}

\switchcolumn

After the end of Easter the abbot went through all his manors and ours, and through those which we had confirmed in fee to tenants. And from all and sundry he demanded an aid and recognition, according to the custom of the realm. Daily he grew skilled in earthly learning, and turned his attention to the acquisition of knowledge of external affairs and of providing for them.

\switchcolumn*

\begin{otherlanguage}{latin}
Cum autem venisset apud Werketunam, et nocte dormisset, venit ei vox, dicens: ``Samson, surge velociter,''\engnotetext{Acts xxii., \oldstylenums{7}.} et iterum, ``Surge, nimis moraris;'' et surgens stupefactus, circumquaque respexit, et vidit lumen in domo necessaria, candelam scilicet, paratam cadere super stramen, quam Reinerus monachus ibi per incuriam reliquerat. Quam cum abbas extinxisset pergens per domum percepit ostium, quod unicum erat, ita obseratum quod aperiri non potuit nisi per clavem, et fenestras strictas, ita quod, si ignis excrevisset, ipse et omnes sui qui in solio illo dormierant extincti essent; quia non erat locus ubi exire vel quo effugere possent.
\end{otherlanguage}

\switchcolumn

But when he was come to Warkton and was at night sleeping, a voice came to him saying, ``Samson, arise up quickly,''\engnotenum{} and again, ``Rise, thou tarriest too long.'' So he arose half dazed, and looking round about him saw in a necessary place a light, a candle which Reiner the monk had left there through carelessness, and which was about to fall on the straw. And when the abbot had put it out, he went through the house and found the door---for there was but one---so fastened that it could only be opened with a key, and the windows barred. Wherefore, had the fire grown, both he and all they who were sleeping in that building would have perished. For there was no way by which they might have gone out or escaped.

\switchcolumn*

\begin{otherlanguage}{latin}
\blockhead{How the creditors of the abbey demanded payment, and how the abbot took his manors into his own hand.}{4}{-0.2cm}
Quocunque ibat abbas, tunc temporis, occurrebant tam Judei quam Christiani exigentes debita, turbantes et anxiantes eum, ita quod somnum amittebat, pallidus et macilentus effectus, et dicens, quod ``nunquam cor meum quietum erit, donec finem debiti mei sciero.''

\end{otherlanguage}

\switchcolumn

Now at that time, wheresoever the abbot went, there hastened to him both Jews and Christians demanding payment of the debts due to them. And they so disturbed the abbot, and caused him such anxiety, that he lost his sleep, and grew pale and thin. Then he said, ``There will be no rest for my soul until I shall see an end of my indebtedness.''

\switchcolumn*

\begin{otherlanguage}{latin}
Veniente festo sancti Michaelis, omnia maneria sua in manu sua recepit cum parvis admodum implementis et paucis instauramentis; Waltero do Hatfeld condonavit xix.\ libras de firmis pr\ae{}teritis, ut libere reciperet iiij$^\text{or}$ maneria qu\ae{} abbas H.\ ei confirmaverat tenenda, scilicet Haregrava et Saxham et Cheventona et Stapelford;
\end{otherlanguage}

\switchcolumn

When Michaelmas came, he took all his manors into his hand, with very few necessary implements and but little stock. He forgave Walter de Hatfield nineteen pounds of arrears of rent, in return for receiving from him the four manors in the which he had been confirmed as tenant by abbot Hugh, namely, Hargrave, Saxham, Chevington and Stapleford.

\switchcolumn*

\begin{otherlanguage}{latin}
\blockhead{How the abbot did not then take Harlow into his own hand.}{4}{-0.45cm}
Herlavam autem distulit abbas recipere hac occasione. Cum forte transitum faceremus in redeundo de Lundonia per forestam, domino abbate audiente, qu\ae{}sivi a vetula transeunte cujus hoc nemus esset, et de qua villa, et quis dominus, vel quis custos? et respondit, quia nemus erat abbatis Sancti Eadmundi, de villa de Herlava, et quod Arnaldus dictus esset custos ejus. De quo cum qu\ae{}rerem, qualiter se haberet versus homines vill\ae{}, respondit, quia demon vivus fuerat, inimicus Dei et excoriator rusticorum; sed timet modo novum abbatem Sancti Eadmundi, quem sapientem et cautelem credit esse, et ideo tractat homines pacifice. Quo audito, factus est abbas hilaris, et manerium recipere distulit ad tempus.

\end{otherlanguage}

\switchcolumn

But the abbot delayed to receive Harlow, and for this cause. It chanced that once when we were returning from London through the forest, that in the hearing of the lord abbot I asked an old woman who passed us whose this wood was, and to what township it belonged, and who was its lord or who was warden over it. And she answered that it was a wood of the abbot of St.\ Edmund, of the township of Harlow, and that one called Arnald was warden of it. And when I asked concerning him as to how he bore himself towards the men of the township, she answered that he had been a fiend incarnate, an enemy of God, and one who evil-intreated the poor. But now, she said, he fears the new abbot of St.\ Edmund, whom he believes to be wise and provident, and therefore treats his men well. And when he heard this the abbot was rejoiced, and delayed for a season to take that manor into his own hand.

\switchcolumn*

\begin{otherlanguage}{latin}
\blockhead{How the abbot managed the lands which he farmed himself.}{4}{-0.45cm}
Ex insperato venit tunc temporis rumor de morte axoris Herlewini de Rung, qu\ae{} cartam ad tenendam eandem villam in vita sua habebat; efc dixifc abbas: ``Heri dedissem lx.\ marcas ad liberandum illud manerium; modo liberavit illud Dominus.'' Cumque sine omni dilatione illuc venisset et recepisset villam in manu sua, et in crastino isset Tilleneriam, membrum illius manerii, venit quidam miles offerens xxx.\ marcas, ut posset tenere illam carucatam terr\ae{} cum pertinentiis per antiquum servitium, scilicet iiij.\ libras; quod noluit abbas; et habuit inde illo anno xxv.\ libras, secundo anno xx.\ libras.

\end{otherlanguage}

\switchcolumn

Then there came the unexpected news of the death of the wife of Herlewin of Rungton. She held a charter by which she was to have that township for her life, and the abbot said, ``Yesterday I would have given sixty marks to free that manor, now the Lord has freed it.'' Then when he had come and had received the township into his hand without any delay, and on the morrow had gone to Tilleney which was a part of that manor, there came to him a certain knight offering thirty marks that he might hold that carucate of land with what belonged to it on the same terms as before, namely, four pounds a year. This the abbot refused, and he took thence that year five and twenty pounds, and in the next year twenty pounds.

\switchcolumn*

\begin{otherlanguage}{latin}
H\ae{}c et consimilia fecerunt eum omnia tenere in manu sua; scilicet quod alibi legitur: ``Omnia Cesar erat.''\footnote[\textdagger]{Lucan Phars.\ iii., \oldstylenums{108}.} Ille vero non segniter agens, horrea et boverias \ae{}dificare c\oe{}pit in primis; ad wainandas terras super omnia excolendas sollicitus, et ad boscos custodiendos vigilans, super quibus dandis vel minuendis ipse seipsum profitebatur avarum.
\end{otherlanguage}

\switchcolumn

This and other similar events led him to keep all things in his hand; as it is written in another place, ``C\ae{}sar was everything.'' Nor was he slack, but caused first of all barns and cattlesheds to be built; he was anxious to cultivate the plough lands above all things; he was careful in maintaining the woods, and in giving and reducing these he made great profit for himself.

\switchcolumn*

\begin{otherlanguage}{latin}
Unum solum manerium de Torp carta sua confirmavit cuidam Anglico natione, gleb\ae{} ascripto, de cujus fidelitate plenius confidebat quia bonus agricola erat, et quia nesciebat loqui Gallice.
\end{otherlanguage}

\switchcolumn

The one manor of Thorp alone he confirmed under his charter to a certain Englishman, a man adscript to the soil, in whose fidelity he had the fullest confidence, because he was a good farmer and because he knew no French.

\switchcolumn*

\begin{otherlanguage}{latin}
\blockhead[\textsc{a.d}.\ \oldstylenums{1182}, November.]{How abbot Samson was made a justice, and how he bore himself in this office.}{4}{-0.65cm}
Nondum transierant vii.\ menses post electionem suam, et ecce! offerebantur ei liter\ae{} domini pap\ae{} constituentes eum judicem de causis cognoscendis, ad qu\ae{} exequenda rudis fuit et inexercitatus, licet liberalibus artibus et scripturis divinis imbutus esset, utpote vir literatus, in scholis nutritus, et rector scholarum in sua provincia notus et approbatus. Vocavit proinde duos clericos legis peritos, et sibi associavit, quorum consilio utebatur in ecclesiasticis negotiis, decretis et decretalibus epistolis operam pr\ae{}bens, cum hora dabatur; ita quod infra breve tempus, tum librorum inspectione, tum causarum exercitio, judex discretus haberetur, secundum formam juris in jure procedens. Unde quidam ait, ``Maledicta sit curia istius abbatis, ubi nec aurum nec argentum mihi prodest ad confundendum adversarium meum!''

\end{otherlanguage}

\switchcolumn

Seven months had not yet passed since his election, and, behold! letters of the lord pope were sent to him appointing him a judge for hearing causes. In the performance of this work he was rude and inexperienced, though he was skilled in the liberal arts and in the holy scriptures, as being a literate man, brought up in the schools and a ruler of scholars, and renowned and well proved in his own work. He therefore associated with himself two clerks who were learned in the the law and joined them with him, using their advice in church matters, while he spent his leisure in studying the decrees and decretal letters. And the result was that in a little while he was regarded as a discreet judge, by reason of the books which he had read and the causes which he had tried, and as one who proceeded in the cases which he tried according to the form of law. And for this cause one said, ``Cursed be the court of this abbot, where neither gold nor silver profit me to confound my enemy!''

\switchcolumn*

\begin{otherlanguage}{latin}
Processu temporis, in causis s\ae{}cularibus aliquantulum exercitatus, naturali ratione ductus, tam subtilis ingenii erat quod omnes mirabantur, et ab Osberto filio Hervei subvicecomite dicebatur: ``Iste abbas disputator est; si procedit sicut incipit, nos omnes exc\ae{}cabit quotquot sumus.'' Abbas vero in hujusmodi causis approbatus, factus est justiciarius errans, sed ab errore et devio se custodiens. Verum``summa petit livor.''\footnote[\textdagger]{Ovid.\ Rem.\ Am.\ \oldstylenums{369}.} Cum homines sui conquererentur ei in curia Sancti \AE{}dmundi, quia nolebut pr\ae{}cipitare sententiam nec credere omni spiritui,\engnotetext{I.\ John iv., \oldstylenums{1}.} sed ordine judiciario procedere, sciens quod merita causarum partium assertione panduntur, dicebatur quod nolebat facere justitiam alicui conquerenti, nisi interventu pecuni\ae{} dat\ae{} vel promiss\ae{}; et quia erat ei aspectus acutus et penetrans, et frons Catonis, raro blandiens, dicebatur magis declinare animum severitati quam benignitati; et, in misericordiis accipiendis pro aliqua forisfactura, dicebatur judicium superexaltare misericordiam,\engnotetext{Cp.\ James ii., \oldstylenums{13}.} quia, sicut visum fuit pluribus, cum perventum erat ad denarios capiendos, raro remittebat quod juste accipi potuit.
\end{otherlanguage}

\switchcolumn

In course of time, he became somewhat skilled in temporal matters, being guided by his commonsense, for his mind was so subtle that all men wondered, and Osbert FitzHerbert, the under-sheriff, used to say, ``This abbot is given to disputation; if he goes on as he has begun, he will blind us all, however many we be.'' But the abbot, being approved in these matters, was made a justice in eyre, though he kept himself from error and wandering. But ``envy seeks out the highest.'' His men complained to him in the court of St.\ Edmund since he would not give judgment hastily or believe every spirit,\engnotenum{} but proceeded in a judicial manner, knowing that the merits of the cases of suitors are made clear by discussion. It was said that he would not do justice to any complainant, unless money were given or promised; and because his aspect was acute and penetrating, and his face, like Cato's, rarely smiling, it was said that his mind lent rather to severity than to mercy. Moreover, when he took fines for any crime, it was said that judgment rejoiced against mercy,\engnotenum{} for in the opinion of many, when it came to a matter of taking money, he rarely remitted that which he might lawfully take.

\switchcolumn*

\begin{otherlanguage}{latin}
Sicut profecit sapientia, ita et providentia in rebus custodiendis et augendis et in expensis honorifice faciendis;
\end{otherlanguage}

\switchcolumn

So his wisdom increased, as well as his care in managing affairs, and in improving his state, and in spending honourably.

\switchcolumn*

\begin{otherlanguage}{latin}
\blockhead{How some men made complaint against the abbot.}{4}{-0.45cm}
sed ed hic multi detractores oblectaverunt, dicentes, quia accepit de sacristia quod voluit, propriis parcens denariis, permittens bladuin suum jacere usque ad tempus car\ae{} venditionis, et jacens ad maneria sua aliter quam pr\ae{}decessores sui, onerans celerarium hospitibus ab abbate potius suscipiendis, per quod abbas posset dici sapiens et instauratus et providus in fine anni; conventus vero et obedientiales inscii et improvidi haberentur.

\end{otherlanguage}

\switchcolumn

But hereupon many of his adversaries raised objections. For they said that he received what he would from the sacristry, and spared his own money, and allowed his corn to lie in the barns until such time as the price should be high. They said that he managed his manors in a way different from that of his predecessors; that he burdened his cellarer with guests who should rather have been received by the abbot, so that the abbot might win repute as a wise man and one who was clever and provident at the end of the year, but the monastery and its officials be thought ignorant and wasteful.

\switchcolumn*

\begin{otherlanguage}{latin}
Ad has detractiones solebam respondere, quod si de sacristia aliquid accipit, ad utilitatem ecclesi\ae{} illud convertit; et hoc nullus invidus negare potuit. Et, ut verum fatear, multo majora et plura bona fuerunt patrata ex oblationibus sacristi\ae{}, infra xv.\ annos post electionem suam, quam quadraginta annis ante.
\end{otherlanguage}

\switchcolumn

To these charges I used to answer that if he took anything from the sacristry, he employed it for the use of the church; and that no envious person could deny this. And to speak the truth, much more good and much greater good was done with the offerings of the sacristry during the fifteen years after his election than in the forty years preceding.

\switchcolumn*

\begin{otherlanguage}{latin}
Aliis objectionibus, quod abbas jacebat ad maneria sua, respondere solebam et excusabam dicens, quia abbas magis est l\ae{}tus et hilaris alibi quam domi; et hoc utique verum fuit, vel propter conqu\ae{}rentium multitudinem qui occurrebant, vel propter rumorum relatores, unde s\ae{}pius contigit quod, propter exhibitionem rigidi vultus sui, ab hospitibus suis multum perdidit favoris et grati\ae{}, licet eis in cibo et potu satisfecerit.
\end{otherlanguage}

\switchcolumn

To the others who objected that the abbot went often to his manors, I was wont to answer and to excuse him by saying that the abbot was happier and in better spirits anywhere than at home. This also was the truth, whether on account of the constant complaints which came to him, or on account of those who told him rumours concerning himself. Accordingly, it often happened that his appearance was stern, and that so he lost much favour and grace with the guests, though he satisfied them with food and drink.

\switchcolumn*

\begin{otherlanguage}{latin}
\blockhead{How the author talked with the abbot concerning the sadness of his manner.}{4}{-0.45cm}
Ego vero hoc attendens, nacta opportunitate, astans ei a secretis dixi: ``Duo sunt qu\ae{} multum miror de vobis;'' et cum qu\ae{}sisset qu\ae{} duo: ``Unum est, quod adhuc in tali statu fovetis sententiam Meludinensium,\engnotetext{Peter Abelard founded a school of dialectic at Melun.} dicentium ex falso nichil sequi, et c\ae{}tera frivola.'' Quibus cum ipse respondisset quod voluit, adjeci ego: ``Aliud nimirum est quod domi non exhibetis vultum propitium sicut alibi, nec inter fratres\footnote[\textdagger]{``Manetis'' desiderari videtur. (Roke.)} qui vos diligunt et dilexerunt et in dominum sibi elegerunt, sed raro estis inter eos, nec tunc congaudetis eis, sicut dicunt.''

\end{otherlanguage}

\switchcolumn

But I noticed this, and taking a favourable occasion, as I was with him alone, said, ``There are two things in you which make me marvel greatly.'' And when he asked what they were, I said, ``One is, that you, in the circumstances in which you are placed, favour the opinion of those of Melun\engnotenum{} who say that from a false premiss nothing can follow, and other foolish things.'' And when he answered what he would to this, I added, ``The other thing at which I marvel is that you do not show a smiling face at home as you do elsewhere, nor remain among the brothers who cherish you, and love you, and have chosen you to be their lord, but are rarely with them, nor do you then rejoice with them, so they say.''

\switchcolumn*

\begin{otherlanguage}{latin}
Quibus auditis, vultum mutavit, et demisso capite respondit: ``Stultus es et stulte loqueris. Scire deberes quod Salomon ait: Fili\ae{} tibi sunt mult\ae{}: vultum propitium ne ostendas eis.'' Ego vero tacui, de cetero ponens custodiam ori meo.
\end{otherlanguage}

\switchcolumn

When he heard this, his expression changed, and he answered, with bowed head, ``You are a fool, and speak as a fool. You should know the saying of Solomon, Hast thou many daughters; show not thyself cheerful toward them.'' Then I was silent, and from that time placed a guard on my lips.

\switchcolumn*

\begin{otherlanguage}{latin}
Alia tamen vice dixi: ``Domine, audivi te in hac nocte post matutinas vigilantem et valde suspirantem contra morem solitum.'' Qui respondit: ``Non est mirum; particeps es bonorum meorum in cibo et potu, et equitaturis, et similibus, sed parum cogitas de procuratione domus et famili\ae{}, de variis et arduis negotiis cur\ae{} pastoralis, qu\ae{} me sollicitant, qu\ae{} animum meum gementem et anxium faciunt.'' Quibus respondi, elevatis manibus ad c\oe{}lum: ``Talem anxietatem\footnote[\dag]{``Aufur'' desideratur (Roke.).} mihi, omnipotens et misericors Dominus.''
\end{otherlanguage}

\switchcolumn

Yet on another occasion I said, ``Lord, I heard you this night keeping watch after matins and breathing heavily contrary to your wont.'' And he answered, ``It is not strange. You share my good things, food, and drink, and riding, and the like. But you think little of the toil of providing for the house and household, of the many and arduous labours which are a pastor's care. These make me anxious, and cause me to groan and to be troubled in spirit.'' Thereupon I raised my hands to heaven and answered, ``From so great anxiety, almighty and merciful Lord, deliver me!''

\switchcolumn*

\begin{otherlanguage}{latin}
Audivi abbatem dicentem, quod si fuisset in eo statu quo fuit antequam monacharetur, et habuisset v.\ vel sex marcas redditus cum quibus sustentari possit in scholis, nunquam fieret monachus nec abbas. Alia vice dixit cum juramento, quod, si pr\ae{}scivisset qu\ae{} et quanta esset sollicitudo abbati\ae{} custodiend\ae{}, libentius voluisset fieri magister almarii et custos librorum, quam abbas et dominus. Illam utique obedientiam dixit pr\ae{} omnibus aliis se semper desiderasse. Et quis talia crederet? Vix ego; nec etiam ego, nisi quia, cum eo vi.\ annis existens die ac nocte, vit\ae{} scilicet meritum et sapienti\ae{} doctrinam plenius agnoscerem.
\end{otherlanguage}

\switchcolumn

I heard the abbot say that if he were in that condition in which he had been before he became a monk, and had five or six marks income wherewith he might support himself in the schools, he would never become either monk or abbot. And on another occasion, he said with an oath that had he known beforehand what care there was in ruling an abbey, and how great that care was, he would far rather have been almoner or librarian, than abbot and lord. And he declared that he had ever longed for the post of librarian above all others. Yet who would believe such things? Not I; no, not I; but that as I lived with him day and night for six years, I know fully the merit of his life and the wisdom of his mind.

\switchcolumn*

\begin{otherlanguage}{latin}
\blockhead{Concerning a dream which the abbot had when a boy.}{3}{-0.5cm}
Narravit aliquando, quod, cum esset puer ix.\ annorum, somniavit se stare pr\ae{} foribus cimiterii ecclesi\ae{} Sancti Eadmundi, et diabolum expansis ulnis velle eum capere; sed sanctus Eadmundus, prope astans, recepit eum in brachiis suis; cumque clamaret somniando, ``Sancte \AE{}dmunde, adjuva me,'' quem nunquam prius audierat nominari, expergefactus est. Mater vero ejus de tanto et tali clamore obstupuit, qu\ae{}, audito somnio, duxit eum ad sanctum Eadmundum orationis gratia; cumque venissent ad portam cimiterii, dixit, ``Mater mea, ecce locus! ecce eadem porta, quam in somnis vidi, quando diabolus volebat me accipere:'' et cognovit locum, ut aiebat, ac si prius eum carnalibus oculis vidisset.

\end{otherlanguage}

\switchcolumn

Once he told me how when he was a boy of nine years, he dreamed that he stood before the doors of the cemetery of St.\ Edmund, and that the devil wished to seize him with his outstretched arms. But the blessed Edmund, who stood near, received him into his arms; and when he had cried out in his sleep, ``St.\ Edmund, help me!'' though he had never heard the saint named before, he awoke. Then his mother was amazed at his loud cry and at his words. And when she had heard the dream, she took him to St.\ Edmund's, that he might pray there. So coming to the gate of the cemetery, he said, ``Mother mine! see this place! See, the very door which I beheld in my dreams, when the devil would have taken me.'' And he knew the place, he said, as if he had already seen it with his carnal eye.

\switchcolumn*

\begin{otherlanguage}{latin}
Abbas ipse exposuit somnium; significans per diabolum voluptatem hujus seculi qu\ae{} eum volebat allicere, sed sanctus Eadmundus eum amplexatus est, quando eum monachum ecclesi\ae{} su\ae{} fieri voluit.
\end{otherlanguage}

\switchcolumn

The abbot himself explained the dream, saying that the devil in it meant the joys of this world which would have enticed him; but that the blessed Edmund embraced him, since he would have him become a monk of his church.

\switchcolumn*

\begin{otherlanguage}{latin}
\blockhead{How the abbot restrained his temper that he might not offend.}{4}{-0.45cm}
Quodam tempore, cum nuntiatum esset ei quod quidam de conventu murmurassent de quodam facto ejus, dixit mihi assidenti: ``Deus, Deus,'' inquit ille, ``multum expedit mihi memorare somnium illud quod somniatum est de me antequam fierem abbas, scilicet quod s\ae{}virem ut lupus. Certe hoc est quod super omnia mundana timeo, ne conventus meus aliquid faciat, unde me s\ae{}vire oporteat; sed ita est, cum dicunt vel agunt aliquid contra voluntatem meam; recolo illud somnium, et licet s\ae{}viam in animo meo, occulte fremens et frendens, vim mihi facio ne s\ae{}viam verbo vel opere: et,
\vspace{-0.5cm}
\begin{verse}
\footnotesize
Strangulat inclusus dolor et cor \ae{}stuat intus.''\footnote[\textdagger]{Ovid.\ Trist.\ v.\ \oldstylenums{1}, \oldstylenums{63}.}
\end{verse}

\end{otherlanguage}

\switchcolumn

Once when it was told him that certain of those in the monastery murmured on account of some act of his, he said to me, as I was near him, ``God, God, it is most expedient that I should be mindful of that dream which was dreamed concerning me before I became abbot, to the effect that I should raven as a wolf. Of a truth, I fear this above all earthly things, that my house may do something which may make it lawful for me to raven. But so it is, that when they say or do aught against my will, I call to mind that dream, and though I raven in spirit, groaning and gnashing my teeth in secret, I put force on myself that I may raven neither in word nor in deed. And hidden grief chokes me, and my heart burns within me.''

\switchcolumn*

\begin{otherlanguage}{latin}
Cum autem esset colericus naturaliter, et facile accenderetur ad iram, iram tamen ratione dignitatis cum magna lucta animi refr\ae{}nabat s\ae{}pius. De qua etiam re aliquando se jactitabat dicens: ``Hoc et illud vidi, hoc et illud audivi, et tamen patienter sustinui.''
\end{otherlanguage}

\switchcolumn

But though he was naturally choleric and easily moved to wrath, yet from respect for his office he generally restrained his anger, albeit with much grief of mind. Of this also he often spoke, saying, ``I have seen this and that, I have heard this and that, and yet have I borne it patiently.''

\switchcolumn*

\begin{otherlanguage}{latin}
\blockhead{How the abott forbade secret accusations, and how he ordered the restoration of all private seals.}{4}{-0.65cm}
Dixit abbas aliquando, sedens in capitulo, qu\ae{}dam verba quibus videbatur efficaciter venari favorem conventus. ``Nolo,'' inquit, ``ut aliquis veniat ad me ad accusandum alium, nisi palam idem dicere voluerit; quod si aliquis aliter fecerit, nomen accusantis palam manifestabo. Volo etiam ut quilibet claustralis liberum habeat accessum ad me, ut mecum loquatur de necessitate sua quando voluerit.'' Illud autem dixit quia magnates nostri, tempore H.\ abbatis, volentes nichil agi in monasterio nisi per eos, decreverunt nullum monachum claustralem debere loqui cum abbate, nisi prius ostenderet capellano abbatis quid et de qua re vellet loqui cum abbate.

\end{otherlanguage}

\switchcolumn

Once as he sat in the chapter, the abbot spoke certain words whereby he seemed to court the favour of the monastery with success. ``I will not,'' he said, ``that any come to me to accuse other, unless he will declare the same openly. But if any desire to act otherwise, I will publicly announce the name of the accuser. And I will also that every monk shall have free access to me, that he may talk with me of his needs when he will.'' Now this he said because the chief men of our house in the days of abbot Hugh, wishing nothing to be done in the monastery save through them, decreed that no cloistered monk should speak with the abbot, unless he had first shown to the abbot's chaplain that which he desired to say to the abbot and the reason.

\switchcolumn*

\begin{otherlanguage}{latin}
Quodam die jussit in capitulo, ut quicumque  sigillum proprium haberet, ei redderet; et ita factum est, et inventa sunt triginta tria sigilla. Rationem hujus pr\ae{}cepti ipse ostendit, prohibens ne aliquis officialis appruntaret aliquod debitum ultra xx.\ solidos, sine assensu prioris et conventus, sicut solebat fieri. Priori vero et sacrist\ae{} reddidit sigilla sua, et c\ae{}tera retinuit.
\end{otherlanguage}

\switchcolumn

One day he commanded in the chapter that all who had seals of their own should restore them to him, and so it was done; and thirty and three seals were found. He himself clearly declared the reason for this command, and forbade any official to contract any debt above the sum of twenty shillings without the consent of the prior and monastery, as had been wont to be done in the past. Then he restored the prior and sacristan their seals and retained the others.

\switchcolumn*

\begin{otherlanguage}{latin}
\blockhead{Concerning further regulations which the abbot made.}{4}{-0.45cm}
Alia die, jussit sibi dari omnes claves cistarum et almariorum et hanepariorum, prohibens ne de cetero aliquis haberet cistam nec aliquid obseratum, nisi per licentiam, nec alias aliquid possideret, nisi quod regula permitteret. Cuilibet tamen nostrum generaliter dedit licentiam habendi denarios usque ad duos solidos, si forte nobis caritative darentur; ita tamen ut in pauperes parentes vel in pios usus expenderentur.

\end{otherlanguage}

\switchcolumn

At another time he ordered that all the keys of the chests, cupboards, and hampers should be given up to him, and forbade anyone henceforth to have any chest or anything locked up, save by permission, or to possess anything of any description except such things as the rule allowed. However, he gave general permission to all of us to have money to the value of two shillings, if perchance this should be given to us in charity. The leave was still conditional on the money being expended for the benefit of poor relations or in pious uses.

\switchcolumn*

\begin{otherlanguage}{latin}
Alia vice, dixit abbas se velle conservare antiquas consuetudines nostras de hospitibus suscipiendis; scilicet, quando abbas est domi, ipse recipiet omnes hospites cujuslibet conditionis, pr\ae{}ter viros religiosos, et pr\ae{}ter presbyteros s\ae{}cularis habitus, et pr\ae{}ter eorum homines, qui per eos se advocaverunt ad portam curi\ae{}; si vero abbas non fuerit domi, omnes hospites cujuslibet conditionis recipientur a celerario usque ad tredecim equos. Si vero laicus vel clericus venerit cum pluribus equis quam tredecim, recipientur a servientibus abbatis, vel intra curiam vel extra, ad expensas abbatis. Omnes viri religiosi, etiam episcopi, si ipsi forte fuerint monachi, pertinent ad celerarium et ad expensas conventus, nisi abbas voluerit eum honorare, et ad expensas suas in sua aula recipere.
\end{otherlanguage}

\switchcolumn

On another occasion the abbot said that he wished to maintain our ancient custom in the matter of the reception of guests, so that when the abbot was at home he should receive all guests of whatever condition, except religious men, and except priests of secular habit, and except their men, who should come to the doors of the court by instruction of their masters. If, however, the abbot should not be at home, all guests of whatever condition should be received by the cellarer, up to the number of thirteen horses. But if a layman or clerk should come with more than thirteen horses, they should be received by the servants of the abbot, either within or without the court, at the expense of the abbot. All religious men, even bishops, if by chance they were monks, were to be the care of the cellarer and entertained at the expense of the monastery, unless the abbot were desirous of showing them honour and of receiving them in his hall at his own expense.

\switchcolumn*

\begin{otherlanguage}{latin}
\blockhead{Concerning the appearance and private character of the abbot.}{4}{-0.45cm}
Abbas Samson mediocris erat statur\ae{}, fere omnino calvus, vultum habens nec rotundum nec oblongum, naso eminente, labiis grossis, oculis cristallinis et penetrantis intuitus, auribus clarissimi auditus, superciliis in altum crescentibus et s\ae{}pe tonsis; ex parvo frigore cito raucus; die electionis su\ae{} quadraginta et septem annos \ae{}tatis habens, et in monachatu decem et septem annos;\footnote[\textdagger]{This agrees sufficiently well with the statement of the unknown author of the MS., Harl.\ \oldstylenums{447} (not John of Taxter, with whom Mr.\ Rokewode confounds him in his note on this passage), that Samson took the habit in \oldstylenums{1166}. If so, he was not a monk at the time of his mission to Rome during the schism of Octavian; infra, p.\ 252. This point has been discussed in the Introduction.}\engnotetext{The \emph{Ann.\ St.\ Ed}.\ (\emph{Mem}.\ II., \oldstylenums{5}) bear this out, as under \oldstylenums{1166} there is an entry, ``Abbot Samson was made a monk.'' (Cp.\ \emph{Mem}.\ I., xliv.)} paucos canos habens in rufa barba, et paucissimos inter capillos nigros, et aliquantulum crispos; sed infra xiiii$^\text{or}$ annos post electionem suam totus albus efficitur sicut nix;

\end{otherlanguage}

\switchcolumn

Abbot Samson was below the average height, almost bald; his face was neither round nor oblong; his nose was prominent and his lips thick; his eyes were clear and his glance penetrating; his hearing was excellent; his eyebrows arched, and frequently shaved; and a little cold soon made him hoarse. On the day of his election he was forty-seven, and had been a monk for seventeen years.\engnotenum{} In his ruddy beard there were a few grey hairs, and still fewer in his black and curling hair. But in the course of the first fourteen years after his election all his hair became white as snow.

\switchcolumn*

\begin{otherlanguage}{latin}
homo supersobrius, nunquam desidiosus, multum valens, et volens equitare vel pedes ire, donec senectus pr\ae{}valuit, qu\ae{} talem voluntatem temperavit; qui, audito rumore de capta cruce et perditione Jerusalem,\engnotetext{Jerusalem was taken by Saladin on October \oldstylenums{2}nd, \oldstylenums{1187}.} femoralibus cilicinis c\oe{}pit uti, et cilicio loco staminis, et carnibus et carneis abstinere; carnes tamen voluit sibi anteferri sedens ad mensam, ad augmentum scilicet elemosin\ae{}. Lac dulce et mel et consimilia dulcia libentius quam c\ae{}teros cibos comedebat.
\end{otherlanguage}

\switchcolumn

He was an exceedingly temperate man; he possessed great energy and a strong constitution, and was fond both of riding and walking, until old age prevailed upon him and moderated his ardour in these respects. When he heard the news of the capture of the cross and the fall of Jerusalem,\engnotenum{} he began to wear under garments made of horse hair, and a horse-hair shirt, and gave up the use of flesh and meat. None the less, he willed that flesh should be placed before him as he sat at table, that the alms might be increased. He ate sweet milk, honey, and similar sweet things, far more readily than any other food.

\switchcolumn*

\begin{otherlanguage}{latin}
Mendaces et ebriosos et verbosos odio habuit; quia virtus sese diligit, et aspernatur contrarium. Murmuratores cibi et potus, et pr\ae{}cipue monachos murmuratores condemnans, tenorem antiquum conservans quem olim habuit dum claustralis fuit; hoc autem virtutis in se habuit quod nunquam ferculum coram eo positum voluit mutare. Quod cum ego novicius vellem probare si hoc esset verum, forte servivi in refectorio, et cogitavi penes me ut ponerem coram eo ferculum quod omnibus aliis displiceret in disco nigerrimo et fracto. Quod cum ipse vidisset, tanquam non videns erat; facta autem mora, p\oe{}nituit me hoc fecisse, et statim, arrepto disco, ferculum et discum mutavi in melius et asportavi; ille vero emendationem talem moleste tulit, iratus et turbatus.
\end{otherlanguage}

\switchcolumn

He hated liars, drunkards, and talkative persons; for virtue ever loves itself and spurns that which is contrary to it. He blamed those who grumbled about their meat and drink, and especially monks who so grumbled, and personally kept to the same manners which he had observed when he was a cloistered monk. Moreover, he had this virtue in himself that he never desired to change the dish which was placed before him. When I was a novice, I wished to prove whether this was really true, and as I happened to serve in the refectory, I thought to place before him food which would have offended any other man, in a very dirty and broken dish. But when he saw this, he was as it were blind to it. Then, as there was some delay, I repented of what I had done, and straightway seized the dish, changed the food and dish for better, and carried it to him. He, however, was angry at the change, and disturbed.

\switchcolumn*

\begin{otherlanguage}{latin}
Homo erat eloquens, Gallice et Latine, magis ratione dicendorum quam ornatui verborum innitens. Scripturam Anglice scriptam legere novit elegantissime, et Anglice sermocinari solebat populo, sed secundum linguam Norfolchi\ae{}, ubi natus et nutritus erat, unde et pulpitum jussit fieri in ecclesia et ad utilitatem audientium et ad decorem ecclesi\ae{}.
\end{otherlanguage}

\switchcolumn

He was an eloquent man, speaking both French and Latin, but rather careful of the good sense of that which he had to say than of the style of his words. He could read books written in English very well, and was wont to preach to the people in English, but in the dialect of Norfolk where he was born and bred. It was for this reason that he ordered a pulpit to be placed in the church, for the sake of those who heard him and for purposes of ornament.

\switchcolumn*

\begin{otherlanguage}{latin}
Videbatur quoque abbas activam vitam magis diligere quam contemplativam, qui bonos obedientiales magis commendavit quam bonos claustrales; et raro aliquem propter solam scientiam literarum approbavit, nisi haberet scientiam rerum s\ae{}cularium; et cum audiret forte aliquem pr\ae{}latum cedere oneri pastorali et fieri anachoritam, in hoc eum non laudavit. Homines nimis benignos laudare noluit, dicens: ``Qui omnibus placere nititur, nulli placere debet.''
\end{otherlanguage}

\switchcolumn

The abbot further appeared to prefer the active to the contemplative life, and praised good officials more than good monks. He rarely commended anyone solely on account of his knowledge of letters, unless the man happened to have knowledge of secular affairs, and if he chanced to hear of any prelate who had given up his pastoral work and become a hermit, he did not praise him for this. He would not praise men who were too kindly, saying, ``He who strives to please all men, deserves to please none.''

\switchcolumn*

\begin{otherlanguage}{latin}
\blockhead{How abbot Samson dealt with flatterers.}{3}{-0.55cm}
Primo ergo anno suscept\ae{} abbati\ae{} omnes adulatores quasi odio habuit, et maxime monachos; sed in processu temporis videbatur eos quasi libentius audire et magis familiares habere. Unde contigit quod, cum quidam frater noster, hac arte peritus, curvasset genua ante eum, et sub obtentu consilii dandi auribus ejus adulationis oleum infudisset, subrisi ego stans a longe; eo vero recedente, vocatus et interrogatus quare riserim, respondi, mundum plenum esse adulatoribus.

\end{otherlanguage}

\switchcolumn

Now in the first year of his abbacy he seemed to hate all flatterers, and especially those who were monks. But in course of time he appeared to listen to them with some willingness, and to treat them more graciously. Once a certain one of our brothers, who was skilled in this art, had bent his knees before him, and under pretence of giving him counsel, had poured the oil of flattery into his ears, while I stood at a distance and smiled. Then when the brother had gone, the abbot called me and asked me why I had been smiling, and I answered that the world was full of flatterers.

\switchcolumn*

\begin{otherlanguage}{latin}
Et abbas; ``Fili mi, diu est quod adulatores novi, et ideo non possum adulatores non audire. Multa sunt simulanda et dissimulanda, ad pacem conventus conservandam. Audiam eos loqui, sed non decipient me, si possum, sicut pr\ae{}decessorem meum, qui consilio eorum ita inconsulte credidit, quod diu ante obitum suum nichil habuit quod manducaret vel ipse vel familia sua, nisi a creditoribus mutuo acceptum; nec erat quod distribui potuit pauperibus die sepultur\ae{} ejus, nisi quinquaginta solidos, qui recepti erant a Ricardo firmario de Palegrava, hac occasione quod eadem die intravit firmam de Palegrava; quos denarios idem Ricardus alia vice reddidit bailivis regis, integram firmam exigentibus ad opus regis.'' His dictis confortatus fui.
\end{otherlanguage}

\switchcolumn

And the abbot said, ``My son, I have been flattered for a long while, and therefore I cannot attend to flattery. There must be much pretence and much concealment that the peace of the monastery may be preserved. I will hear them speak, but they will not deceive me, if I can prevent it, as they deceived my predecessor, who gave such unconsidered attention to them that for a long while before his death he had nothing wherewith to feed himself or his household, save that which he borrowed from creditors. And on the day of his burial there was nothing which could be distributed among the poor, save fifty shillings which were received from Richard the tenant of Palgrave, because on the same day he entered on the tenancy of Palgrave; and this money the same Richard afterwards paid again to the officials of the king, who exacted the full rent for the royal use.'' And with these words I was reassured.

\switchcolumn*

\begin{otherlanguage}{latin}
\blockhead{How abbot Samson managed his household.}{3}{-0.55cm}
Ille vero studuit disciplinatam domum habere, et famili\ae{} magnitudinem sed necessariam, providens sibi quod firma ebdomad\ae{}, qu\ae{} predecessori suo non sufficiebat ad expensam v.\ dierum, ei suffecit octo diebus, vel novem, vel decem, si esset ad maneria sua sine magno adventu hospitum. Singulis vero ebdomadis, computationem expens\ae{} su\ae{} domus audiebat, non per vicarium, sed in propria persona, quod antecessor ejus nunquam solebat facere.

\end{otherlanguage}

\switchcolumn

He laboured to secure a well regulated house, and a household large, but not larger than was right, and he took care that the weekly allowance which in the time of his predecessor had not been enough for five days, should last him for eight days, or nine, or ten, if he were on his manors and there were no great coming of guests. Every week, moreover, he audited the expenses of his house, not through an agent, but in person, a thing which his predecessor had never been accustomed to do.

\switchcolumn*

\begin{otherlanguage}{latin}
Septem annis primis quatuor fercula\footnote[\textdagger]{``Sumpsit'' desiderari videtur.} in domo sua, postea nisia\footnote[\ddag]{$=$ \emph{non nisi}.} tria, pr\ae{}ter xenia et pr\ae{}ter venationem de parcis suis, vel pisces de vivariis suis. Et si forte aliquem retinuit ad tempus in domo sua prece alicujus potentis vel alicujus familiaris, vel nuncios, vel cithar\ae{}dos, vel aliquem hujusmodi, nacta opportunitate transfretandi vel longe eundi, a talibus superfluis se prudenter exoneravit.
\end{otherlanguage}

\switchcolumn

For his first seven years he had four dishes in his house, afterwards only three, if one excludes presents, and game from his parks and fish from his ponds. And if he happened to keep anyone for a while in his house at the request of some great man or of one of his friends, or messengers, or minstrels, or any such person, he used to take any opportunity of crossing the sea or going a long journey, and so prudently freed himself from so great expense.

\switchcolumn*

\begin{otherlanguage}{latin}
\blockhead{How the abbot treated those monks with whom he had been intimate before he became abbot.}{4}{-0.65cm}
Monachos vero, quos socios abbas habuit ante abbatiam susceptam magis dilectos et magis familiares, raro promovit ad obedientias occasione pristin\ae{} familiaritatis, nisi essent idonei; unde quidam ex nostris, qui ei erant propitii ad eligendum eum abbatem, dixerunt eum minus quam deceret diligere eos, qui eum antequam fuerat abbas dilexerant, et eos plus ab eo amari, qui eum et aperte et occulte depravaverunt, et eum hominem iracundum, non socialem, paltenerium et baratorem de Norfolch, etiam in audientia multorum, publice nominaverunt. Verum, sicut ille pristinis amicis suis nihil amoris vel honoris indiscrete exhibuit post susceptionem abbati\ae{} sic et pluribus aliis pro meritis suis nichil rancoris vel odii exhibuit, bonum aliquando reddens pro malo, et benefaciens persequentibus eum.

\end{otherlanguage}

\switchcolumn

Those monks whom the abbot, before he acquired the abbacy, had treated as his most cherished and intimate friends, he seldom raised to official positions on the score of his former intimacy with them, unless they were fit persons. Therefore some of our number, who had favoured his election as abbot, said that he showed them less favour than was their due, who had loved him before he was abbot, and that those rather were cherished by him who had slandered him both openly and secretly, and in the hearing of many had publicly declared him to be a hot-tempered man, one who was unsociable, conceited, and a Norfolk cheat. But, just as after he received the abbacy he made no injudicious exhibition of affection or of a desire to honour his former friends, so also he did not show towards the others any of that rancour or hatred which they deserved, returning good for evil on many occasions, and doing good to those who persecuted him.

\switchcolumn*

\begin{otherlanguage}{latin}
Habuit etiam in consuetudine quiddam quod nunquam vidi hominem habere, scilicet quod multos affectuose dilexit, quibus nunquam vel raro vultum amoris exhibuit; hoc quod vulgus clamat, dicens, ``Ubi amor ibi oculus.'' Et aliud mirum fuit, quod damnum suum in temporalibus a servientibus suis scienter sustinuit, et se sustinere confessus est: sed, sicut credo, hoc fuit in causa, ut congruum tempus expectaret quo rem consultius emendaret, vel ut majus damnum dissimulando evitaret.
\end{otherlanguage}

\switchcolumn

He had also a characteristic which I have never seen in any other man, namely, that he had a strong affection for many to whom he never or seldom showed a loving face, which the common saying declares to be usual, when it says, ``Where love is, there the glance follows.'' And there was another noteworthy thing, that he wittingly suffered loss from his servants in temporal matters, and allowed that he suffered it; but, as I believe, the reason for this was that he waited for a fit season when the matter might be conveniently remedied, or that by concealing his knowledge he might avoid greater loss.

\switchcolumn*

\begin{otherlanguage}{latin}
\blockhead{How the abbot treated his relations.}{3}{-0.55cm}
Parentes suos mediocriter dilexit, nec minus vero tenere sicut alii solent; quia nullum infra tertium gradum habuit, vel habere simulavit. Sed audivi eum dicentem quod habuit parentes nobiles et generosos, quos nunquam imperpetuum ut parentes cognosceret; quia, ut aiebat, plus essent ei oneri quam honori, si hoc scirent; sed eos voluit consanguineos habere qui eum consanguineum habuerunt quando fuit pauper claustralis.

\end{otherlanguage}

\switchcolumn

For his relations he displayed moderate affection, but yet no less tender than that which others are wont to show, since he had no relatives within the third degree, or pretended that he had not. I have, however, heard him assert that he had relations who were noble and distinguished, but that he would never at any time recognise them as relations. For, as he said, they would be rather a burden than a source of advantage to him if they knew of their relationship. On the other hand, he wished to have as kin those who had claimed kinship with him when he was a poor cloistered monk.

\switchcolumn*

\begin{otherlanguage}{latin}
Quosdam eorum, (eos secundum quod sibi utiles et idoneos \ae{}stimavit) diversis officiis in domo sua, quosdam villis custodiendis, deputavit. Quos autem infideles probavit, a se elongavit, sine spe redeundi.
\end{otherlanguage}

\switchcolumn

Some of his relations, in cases where he thought them useful and capable men, he appointed to various offices in his house, and others he entrusted with the wardenship of manors. But any whom he proved to be unfaithful he drove far from him, without hope of return.

\switchcolumn*

\begin{otherlanguage}{latin}
Quendam hominem medi\ae{} manus,\footnote[\textdagger]{Of lowly station.} qui patrimonium ejus fideliter servaverat, et ei juveni devote servierat, pro caro consanguineo habuit, et filio ejus clerico primam ecclesiam in abbatia sibi commissa vacantem dedit, et c\ae{}teros filios ejus omnes promovit.
\end{otherlanguage}

\switchcolumn

He held as his dear kinsman a certain man of low birth, who had managed his inheritance faithfully and served him devotedly in his boyhood. To this man's son, who was a clerk, he gave the first church which fell vacant after his accession to the abbacy, and he also promoted all the man's other sons.

\clearpage

\switchcolumn*

\setcounter{engnote}{60}

\begin{otherlanguage}{latin}
\blockhead{How the abbot was mindful of those who had shown kindness to him in the past, and how he treated those who had been harsh.}{5}{-0.15cm}
Capellanum quendam, qui eum sustinuerat in scholis Parisius qu\ae{}stu aqu\ae{} benedict\ae{},\engnotetext{It was common practice to devote the money derived from the sale of holy water to the support of poor clerks. The synod of Exeter (\oldstylenums{1287}) definitely provided that such profits should be so spent. (\emph{Mem}.\ I., \oldstylenums{247}, note \emph{b}.)} quando pauper fuerat, mandari fecit, et ei ecclesiasticum beneficium quo sustentari possit, affectu vicario, contulit.

\end{otherlanguage}

\switchcolumn

There was a certain chaplain who had maintained him in the schools of Paris by the sale of holy water,\engnotenum{} when he was poor. This man the abbot caused to be summoned to him, and conferred on him an ecclesiastical benefice, with the position of vicar, whereby he might be supported.

\switchcolumn*

\begin{otherlanguage}{latin}
Cuidam servienti pr\ae{}decessoris sui victum et vestitum concessit omnibus diebus vit\ae{} su\ae{}, qui imposuerat ei compedes ad pr\ae{}ceptum domini sui, quando fuit positus in carcere.
\end{otherlanguage}

\switchcolumn

He granted to a certain servant of his predecessor food and clothing for all the days of his life, this man being he who had placed fetters on him at the command of his lord when he was imprisoned.

\switchcolumn*

\begin{otherlanguage}{latin}
Filio Eli\ae{},\engnotetext{As to grants of land to Elias, by abbot Hugh, see Rokewode (p.\ \oldstylenums{122}). The total amounted to one hundred and forty acres, of which sixty were in Elmswell.} pincern\ae{} Hugonis abbatis, facienti ei homagium de terra patris sui, dixit in plena curia: ``Distuli jam capere homagium tuum vij.\ annis de terra quam H.\ abbas dedit patri tuo, quia illud donum erat in detrimentum aul\ae{} de Elmeswell: modo victus sum, memor beneficii quod pater tuus mihi fecit quando in vinculis eram, quia misit mihi portionem de ipso vino quod dominus suus biberat, mandando ut confortarer in Deo.''
\end{otherlanguage}

\switchcolumn

When FitzElias,\engnotenum{} the cup-bearer of abbot Hugh, came to do him homage for his father's land, the abbot said to him in open court, ``I have delayed now for seven years to receive your homage for the land which abbot Hugh gave to your father, since that gift was to the detriment of the manor of Elmswell. Now I give way, since I am mindful of the good which your father did to me when I was in bonds. For he sent to me some of the very wine which his lord drank, with a message that I should be of good courage in God.''

\switchcolumn*

\begin{otherlanguage}{latin}
Magistro Waltero, filio magistri Willelmi de Dice,\engnotetext{This is the compiler of the passage which appears at the end of Jocelin's \emph{Chronicle} (see note to p.\ \oldstylenums{153}).} petenti caritative vicariam ecclesi\ae{} de Cheventona, respondit: ``Pater tuus magister scholarum erat; et cum pauper clericus eram, concessit mihi introitum schol\ae{} su\ae{} sine pacto et caritative, et usum discendi; et ego, causa Dei, concedo tibi quod postulas.''
\end{otherlanguage}

\switchcolumn

When master William, son of master William of Diss,\engnotenum{} asked of his grace for the vicarage of the church of Chevington, he answered, ``Your father was master of the schools, and when I was a poor clerk he allowed me to enter the school without terms and of his grace, and to have the opportunity of learning. And I, for the sake of God, grant you that which you ask.''

\switchcolumn*

\begin{otherlanguage}{latin}
Duos etiam milites de Risebi, Willelmum et Normannum, cum judicati essent forte in misericordia ejus, ita allocutus est coram omnibus; ``Cum essem monachus claustralis missus Dunelmiam pro negotiis ecclesi\ae{} nostr\ae{}, et illinc in redeundo per Risebi, vespere obscuro interceptus, petissem hospitium a domino Normanno, omnino repulsam sustinui; domum vero domini Willelmi adiens et hospitium postulans, ab eo honorifice susceptus sum: et ideo xx.\ solidos, scilicet misericordiam, sine misericordia, integram recipiam a Normanno; Willelmo autem gratias ago, et debitam miserationem xx.\ solidorum gratanter remitto.''
\end{otherlanguage}

\switchcolumn

Two knights also from Risby, William and Norman, were by chance judged to be at his mercy, and he thus addressed them in the presence of all: ``When I was a cloistered monk, I was sent to Durham on the business of our church. And as I was returning thence by way of Risby, I was overtaken by a dark night, and sought entertainment from the lord Norman, but suffered an absolute denial. Then I went to the house of lord William and prayed for lodging, and was honourably received by him. For this cause I will take twenty shillings, the full penalty, without pity, from Norman; but I give thanks to William, and gladly remit the due penalty of twenty shillings.''

\clearpage

\switchcolumn*

\begin{otherlanguage}{latin}
\blockhead{Concerning other good acts of abbot Samson.}{3}{-0.1cm}
Qu\ae{}dam juvencula virguncula, ostiatim victum qu\ae{}rens, conquesta est abbati, quod unus ex filiis Ricardi filii Drogonis eam vi oppresserat; qu\ae{} tandem, procurante abbate, pro bono pacis unam marcam accepit. Abbas autem iiij$^\text{or}$ marcas accepit ab eodem R.\ pro concessione concordi\ae{}; sed omnes illas v.\ marcas jussit dari statim cuidam mercatori, hoc pacto, ut pr\ae{}fatam pauperculam duceret in uxorem.

\end{otherlanguage}

\switchcolumn

A certain young girl, who was begging her bread from door to door, made complaint to the abbot that one of the sons of Richard FitzDrogo had assaulted her. This wrong, by the abbot's intervention, was at last settled for the sake of peace by the acceptance of one mark by the girl. The abbot further took four marks from the said Richard for leave to compound for his offence. But all these five marks he ordered to be given at once to a certain pedlar, on condition that he should marry the poor girl.

\switchcolumn*

\begin{otherlanguage}{latin}
\blockhead[Before \textsc{a.d}.\ \oldstylenums{1198}.]{}{2}{-0.7cm}
In villa Sancti \AE{}dmundi domos lapideas emit abbas, et eas scolarum regimini assignavit,\engnotetext{The date of the foundation of the school by abbot Samson is uncertain. But as there is a grant to the master in \oldstylenums{1198}, the foundation was probably made early in the abbacy of Samson. (Cp.\ Rokewode, p.\ \oldstylenums{123}.)} hac occasione, ut pauperes clerici in perpetuum ibi quieti essent de conductione domus, ad quam conducendam denarium vel obolum singuli scolares, tam impotentes quam potentes, bis in anno conferre cogebantur.

\end{otherlanguage}

\switchcolumn

In the town of St.\ Edmund's the abbot bought stone houses, and appointed them for the maintenance of the schools.\engnotenum{} His reason for so doing was that thus the poor clerks might there be for ever free from paying rent for houses. Hitherto, for the payment of the rent, all the scholars, poor and rich alike, had been compelled to contribute a penny or a halfpenny twice a year.

\switchcolumn*

\begin{otherlanguage}{latin}
\blockhead[\textsc{a.d}.\ \oldstylenums{1190}.]{How the Jews were driven from Saint Edmund's.}{4}{-0.45cm}
Recuperatio manerii de Mildenhala pro mille marculis argenti et centum, et ejectio Judeorum\engnotetext{Arnold (\emph{Mem}.\ I., \oldstylenums{249}, note) points out that, in the absence of any royal castle to which the Jews might retire for safety, their expulsion was really in their own interests. They would otherwise be always liable to massacre, and especially at this particular time, which was marked by an outburst of fanaticism against the Jews, who were massacred in many towns, especially at York. Diceto (II., \oldstylenums{75}--\oldstylenums{6}) mentions the massacre of fifty-seven Jews at St.\ Edmund's on Palm Sunday, \oldstylenums{1190}.} de villa Sancti \AE{}dmundi, et fundatio novi hospitalis de Babbewell, magn\ae{} probitatis sunt indicia.

\end{otherlanguage}

\switchcolumn

The recovery of the manor of Mildenhall for one thousand one hundred silver marks, and the expulsion of the Jews\engnotenum{} from the town of St.\ Edmund's, and the foundation of a new hospital at Babwell, were signs of great virtue.

\switchcolumn*

\begin{otherlanguage}{latin}
Dominus abbas petiit a rege literas ut Judei
ejicerentur\footnote[\textdagger]{Under the circumstances, this must have been the most humane course in the interest of the Jews themselves. All large English towns at this time were imperfectly policed, and the temper of the populace savage and uncertain. A riot having been once set on foot, the only hope of safety for the Jews was in taking refuge in some royal castle. Thus at Norwich, where there was a massacre of Jews about this time (Feb.\ \oldstylenums{6}, Diceto), those of them who were found in private houses perished; those who escaped to the castle were safe. So at Lincoln, the rumour of an intended attack having come to their ears, the Jews of the town sought shelter in the ``munitio regia'' (Newb.\ iv.\ \oldstylenums{9}) and saved their lives. At York, where the hideous tragedy of their fate was on so large a scale as to attract the notice of history, the Jews, having been admitted into the castle, would have escaped with life but for their own suspicious folly, which led them to shut the gates against the governor, when for some cause he had gone outside the walls. But at Lynn in January, and at Stamford on the \oldstylenums{7}th March, there being no royal castle at either place, the resident Jews had been killed and plundered with impunity. At Bury itself (Harl.\ \oldstylenums{1032}, sub an.\ \oldstylenums{1090}; Diceto, \oldstylenums{651}) there was a murderous outbreak on Palm Sunday (March \oldstylenums{18}) in which, according to Diceto, fifty-seven Jews perished. There was no castle at Bury; to the abbot alone could the survivors look for protection; and Samson knew that he had not sufficient force at his command to ensure it to them. To banish them therefore was the best thing that could be done.} a villa Sancti \AE{}dmundi, allegans quod
quicquid est in villa Sancti \AE{}dmundi, vel infra \emph{bannamleucam}, de jure Sancti \AE{}dmundi est; ergo, vel Judei debent esse homines Sancti \AE{}dmundi, vel de villa sunt ejiciendi. Data est ergo licentia, ut eos ejiceret, ita tamen quod haberent omnia katalla, scilicet et pretia domorum suarum et terrarum. Et cum emissi essent, et armata manu conducti ad diversa oppida, abbas jussit sollemniter excommunicari per omnes ecclesias et ad omnia altaria omnes illos, qui de cetero receptarent Judeos vel in hospitio reciperent in villa Sancti \AE{}dmundi. Quod tamen postea dispensatum est per justiciaries regis, scilicet, ut si Judei venerint ad magna placita abbatis ad exigendum debita sua a debitoribus suis, sub hac occasione poterunt duobus diebus et ij.\ noctibus hospitari in villa, tertio autem die libere discedent.
\end{otherlanguage}

\switchcolumn

The lord abbot sought letters from the king that the Jews might be expelled from the town of St.\ Edmund's, asserting that whatever is in the town of the blessed Edmund, or within the district subject to the jurisdiction of the monastery, belongs of right to the Saint, and that consequently the Jews ought either to be the men of St.\ Edmund, or else be driven from the town. Leave, therefore, was given to him to eject them, provided that they should have all their chattels, as well as the value of their houses and lands. And when they were sent forth, and under armed force were conducted to various towns, the abbot ordered that  in every church and before every altar those should be solemnly excommunicated who should henceforth receive Jews or entertain them as guests in the town of St.\ Edmund's. This provision was afterwards modified by the justices of the king, to the effect that if Jews should come to the great pleas of the abbot in order to exact debts due to them from their debtors, then for this reason they might be entertained for two days and two nights in the town, and depart in peace on the third day.

\switchcolumn*

\begin{otherlanguage}{latin}
\blockhead[\textsc{a.d}.\ \oldstylenums{1189} September.]{How the abbot secured the manor of Mildenhall, and endowed the hospital at Babwell.}{5}{-0.6cm}
Abbas optulit regi Ricardo quingentas marcas pro manerio de Mildenhala,\engnotetext{This manor was in the hands of the Crown in Domesday. It had, however, been granted to St.\ Edmund by the charter of Edward the Confessor, and was held by a certain Stigand during the life of Edward (\emph{Mem}.\ I., \oldstylenums{48} and \oldstylenums{250}, note \emph{a}; cp.\ \emph{Monasticon} III., \oldstylenums{188}). The present transaction is related by Benedictus (II., \oldstylenums{91}; cp.\ Hoveden, III., \oldstylenums{18}) ``And Samson, abbot of St.\ Edmund's, bought from King Richard the manor, which is called Mildenhall, for one thousand marks, because it was said to have belonged originally to the abbey of St.\ Edmund's.''} dicens illud manerium lx.\ librarum et decem, et pro tanto esse rollatum in magna rolla de Wincestria.\footnote[\textdagger]{Domesday Book; the returns forming the basis of which ``were transmitted to a board sitting at Winchester, by whom they were arranged in order and placed upon record'': Lingard, i.\ \oldstylenums{249}.}\engnotetext{That is, Domesday Book.} Et cum ita spem voti sui concepisset, cepit res dilationem usque in crastinum. Interim venit aliquis dicens regi, manerium illud bene valere c.\ libras. In crastino ergo abbate petitioni su\ae{} instante, dixit rex: ``Nichil est, domine abbas, quod qu\ae{}ris, vel mille marcas dabis, vel manerium non habebis.''

\end{otherlanguage}

\switchcolumn

The abbot offered king Richard five hundred marks for the manor of Mildenhall,\engnotenum{} saying that the annual value of that manor was seventy pounds, and that it had been enrolled for that amount in the great roll of Winchester.\engnotenum{} And when he thought that he would obtain his desire in this matter, the settlement of the affair was postponed to the following day. In the interval there came one to the king and told him that the manor was worth quite one hundred pounds. And so when the abbot urged his request on the morrow, the king said to him, ``My lord abbot, it is useless for you to make this petition to me. Either you shall give me a thousand marks, or you shall not have the manor.''

\switchcolumn*

\begin{otherlanguage}{latin}
Cum autem regina Ellienor secundum consuetudinem regni\footnote[\textdagger]{The ``custom'' here alluded to is described by Blackstone (\emph{Commentaries}, i.\ \oldstylenums{229}, edition of \oldstylenums{1844}) as ``an ancient perquisite called queen-gold or \emph{aurum regin\ae{}},'' due, in proportion of ten per cent., from every person making a voluntary offering to the king. ``As, if an hundred marks of silver be given to the king for liberty to take in mortmain, or to have a fair, market, park, chase, or free warren, then the queen is entitled to ten marks in silver, or (what was formerly an equivalent denomination) to one mark in gold.''} deberet accipere c.\ marcas ubi rex cepit mille,\engnotetext{This was the ``aurum regin\ae{}'' or queen-gold, a due of ten per cent., to be paid by everyone whoe made a gift to the king. (Blackstone, quoted by Arnold, \emph{Mem}.\ I., \oldstylenums{250}, note \emph{c}.)} accepit a nobis calicem magnum aureum in pretium c.\ marcarum, et eundem calicem nobis reddidit pro anima domini sui regis Henrici, qui eum primo dederat Sancto \AE{}dmundo.
\end{otherlanguage}

\switchcolumn

But queen Eleanor, who according to the custom of the realm, had the right to receive a hundred marks when the king received a thousand,\engnotenum{} took from us a great gold chalice of the value of a hundred marks, and restored this same chalice to us for the good of the soul of king Henry her lord, who had originally given it to St.\ Edmund.

\switchcolumn*

\begin{otherlanguage}{latin}
\blockhead[\textsc{a.d}.\ \oldstylenums{1193}]{}{2}{-.7cm}
Alia quoque vice, cum thesaurus ecclesi\ae{} nostr\ae{}\textsuperscript{\ddag} portaretur Lundonias ad redemptionem regis Ricardi, eadem regina\textsuperscript{*} eundem calicem adquietavit pro c.\ marcis et nobis reddidit, accipiens cartam nostram a nobis in testimonium promissionis nostr\ae{} fact\ae{} in verbo veritatis, quod calicem ilium nunquam pro aliquo casu ab ecclesia nostra alienabimus.\footnote[]{\textsuperscript{\ddag}The exaction of a heavy ransom by a German emperor (Henry VI.) from an adversary who had fallen under his power, like the substantially similar transaction in \oldstylenums{1871}, was conducted with all the forms of courtesy. The sum demanded was \oldstylenums{70,000} marks; and Richard declared (in the letter to his mother written from Haguenau on the \oldstylenums{19}th April \oldstylenums{1193}, in obedience to which St.\ Edmund's and all other monasteries in the kingdom handed over their gold and silver to royal commissioners), that the treaty of amity which he had concluded with the emperor was worth all the money, and that if he were free and at home in England, he would voluntarily pay as much or more money to obtain such a treaty! (Hoved.\ iii.\ \oldstylenums{209}).}\footnote[]{\textsuperscript{*}``The Queen's release of the golden chalice is set forth in the Registr.\ Sacr., fol.\ \oldstylenums{29} \emph{v}., and is printed in Dugdale'' (Rokewode).}\engnotetext{Richard was ransomed from Henry VI., for \oldstylenums{100,000} marks of silver ``according to the standard of Koln.'' (Hoveden, III., \oldstylenums{215}--\oldstylenums{6}. Diceto, II., \oldstylenums{110}.) The demands made on the clergy and laity may be found in Hoveden (III., \oldstylenums{208} ff). All the monasteries were obliged to hand over their gold and silver, which was placed in the hands of royal commissioners appointed to superintend the raising of the ransom. (Hoveden, III., \oldstylenums{210}.)}\engnotetext{The charter of release is printed in Dugdale's \emph{Monasticon} (Ed.\ \oldstylenums{1846}; Vol.\ III., \oldstylenums{154}).}

\end{otherlanguage}

\switchcolumn

At a later date, when the treasure of our church was carried to London for the ransom of king Richard,\engnotenum{} the queen redeemed the same chalice\engnotenum{} for a hundred marks and restored it to us, and received from us a charter in proof of our promise, made on the word of truth, that we would never for any reason alienate that chalice from our church.

\switchcolumn*

\begin{otherlanguage}{latin}
\blockhead[\textsc{a.d}.\ \oldstylenums{1198}]{}{2}{-.7cm}
Cum autem persoluta esset tanta pecunia cum magna difficultate adquisita, sedit abbas in capitulo, dicens se habere aliquam portionem de tanto qu\ae{}stu tanti manerii. Et responsum est a conventu quod hoc justum est, et ad voluntatem vestram fiat. Et dixit abbas, so posse vindicare de jure dimidiam partem, ostendens se plusquam cccc.\ marcas cum magnis laboribus expendisse, sed dixit se esse contentum quadam portione illius manerii, qu\ae{} dicitur Ikelingham; quod concessum est ei libentissime a conventu. Abbas vero hoc audiens, dixit: ``Et ego illam partem terr\ae{} recipio ad meum opus, non ut retineam in manu mea, vel ut parentibus meis donem, sed pro anima mea et pro animabus vestris communiter dono illam novo hospitali de Babbewell,\engnotetext{Traces of this are to be found in a small ruin near the railway bridge on the Thetford road. (Arnold, \emph{Mem}.\ I., \oldstylenums{252}, note \emph{a}.)} in sustentationem pauperum et usum hospitalitatis.'' Dixit, et ita factum est, et carta regis postea confirmatum.\engnotetext{On this charter, see Rokewode (p.\ \oldstylenums{124}--\oldstylenums{5}).}

\end{otherlanguage}

\switchcolumn

Now when this large sum of money had been collected with great difficulty and had been paid, the abbot, sitting in the chapter, said that he ought to share somewhat in so great an acquisition as that of so fair a manor. And the monastery answered, ``That is just. Let it be done according to your will.'' And the abbot said that he might lawfully claim half of it, and showed that he had spent more than four hundred marks with great labour, but that he would be content with one portion of that manor, a place called Icklingham, and this was most readily granted to him by the monastery. Hearing this, the abbot said, ``And I receive that portion of the land for my own purposes, not that I may keep it in my hand or to give it to my relations, but that for the good of my soul and of the souls of all of you, I give it to the new hospital at Babwell,\engnotenum{} for the support of the poor and for the use of the hospital.'' So he spoke, and so it was done, and the act was afterwards confirmed by a charter from the king.\engnotenum{}

\switchcolumn*

\begin{otherlanguage}{latin}
H\ae{}c et consimilia facta, scriptis et laudibus \ae{}ternanda, fecit abbas Samson. Nichil tamen se dixit agere, nisi posset facere in diebus suis dedicari ecclesiam nostram; post quod factum, asseruit se velle mori: ad cujus facti sollemnitatem dixit se esse paratum expendere duo milia marcas argenti, dummodo dominus rex ibi esset pr\ae{}sens, et res debito honore peragi possit.
\end{otherlanguage}

\switchcolumn

These and other like things did abbot Samson, which are worthy to be written down and to be praised for all time. Yet he declared that he would have done nothing, unless in his time he could bring to pass that our church should be dedicated; and when that had been accomplished, he asserted that he was ready to die. Moreover, he said that for the doing of this thing he would expend two thousand marks of silver, if so the king might be present and the affair carried through with due ceremony.

\switchcolumn*

\begin{otherlanguage}{latin}
\blockhead[\textsc{a.d}.\ \oldstylenums{1183}]{Concerning the church of Woolpit, and how it was secured for the abbey.}{4}{-.65cm}
Nuntiatum est abbati, quod ecclesia de Wlpet vacaret, Waltero de Constantiis\footnote[\textdagger]{Walter of Coutances, who, when archdeacon of Oxford, had been one of the witnesses to Henry II.'s will, was elected to Lincoln in July \oldstylenums{1183}, and consecrated in France, by Richard, archbishop of Canterbury. In the following year he was made archbishop of Rouen. Many interesting letters from him to Ralph de Diceto, dean of London, are preserved in the \emph{Ymagines Historiarum} of the latter.}\engnotetext{Elected bishop of Lincoln and consecrated in \oldstylenums{1183}. In the next year he was translated to Rouen. He was one of Richard I.'s most trusted servants, and was sent to England to settle the disputes which arose between Longchamp and John after the departure of Richard on the crusade. He died in \oldstylenums{1207}.} electo ad episcopatum de Lincolnia. Mox convocavit priorem et magnam partem conventus, et incipiens narrationem suam, et\footnote[\ddag]{\emph{Sic.}} ait: ``Bene scitis quod multum laboravi propter ecclesiam de Wlpet, propter quam habendam in proprios usus vestros iter arripui versus Romam per consilium vestrum, tempore scismatis inter papam Alexandrum et Octavianum,\engnotetext{On the death of Adrian IV., in \oldstylenums{1159}, Cardinal Roland was elected by a majority only, and took the name of Alexander III. The imperialist party, however, declared Cardinal Octavian pope under the name of Victor IV. Octavian was supported by the citizens of Rome and by the emperor, and recognised by the imperial synod of Pavia. Alexander took refuge in France. Victor died (\oldstylenums{1164}) without the dispute having been finally settled. (Cp.\ Gregorovius, \emph{City of Rome in the Middle Ages} (Eng.\ trans.) Vol.\ IV, p.\ \oldstylenums{563} ff.)} transivique per Italiam, illa tempestate qua omnes clerici qui portabant literas domini pap\ae{} Alexandri capiebantur, et quidam incarcerabantur, quidam suspendebantur, quidam, truncatis naso et labiis, remittebantur ad papam in dedecus et confusionem ipsius. Ego vero simulavi me esse Scottum,\engnotetext{Arnold (\emph{Mem}.\ I., xliii.) gives as the reason for Samson's action the fact that Scotland favoured the party of Octavian, in opposition to the English support of Alexander.} et Scotti habitum induens, et gestum Scotti habens, s\ae{}pe illis qui mihi illudebant baculum meum excussi, ad modum teli quod vocatur \emph{gaveloc}, de more Scottorum voces comminatorias proferens. Obviantibus et interrogantibus, quis essem, nichil respondi, nisi: `\emph{Ride, ride Rome, turne Cantwereberei.}'\engnotetext{Arnold (\emph{Ibid}., note \oldstylenums{1}) gives as the meaning of this, ``I am riding towards Rome, turning from Canterbury.'' He adds, ``If he had meant to say, `returning from Canterbury,' he would at once have been taken for an English adherent of Alexander.''} Sic feci, ut me et propositum meum celarem,
\vspace{-0.3cm}
\begin{verse}
{\footnotesize
Tutius et peterem, Scotti sub imagine, Romam.}
\end{verse}
\vspace{-0.3cm}
Impetratis autem literis a domino papa pro voto meo, in redeundo transivi per quoddam castellum, sicut via mea ducebat ab urbe; et ecce ministri de castro circumdederunt me, capientes et dicentes: `Iste solivagus, qui Scottum se facit, vel explorator est, vel portitor literarum falsi pap\ae{} Alexandri.' Et dum perscrutabantur panniculos meos et caligas, et femoralia, et etiam sotulares veteres, quos super humeros portavi ad consuetudinem Scottorum, injeci manum meam in peram quam portavi cuteam, in qua scriptum domini pap\ae{} continebatur, positum sub ciffo parvo, quo bibere solebam: et Domino Deo volente, et sancto \AE{}dmundo, simul extraxi scriptum illud cum ciffo, ita quod, brachium extendens in altum, breve tenui sub ciffo. Ciffum quidem viderunt, sed breve non perceperunt. Et sic evasi manus eorum in nomine Domini. Quicquid monet\ae{} habui abstulerunt a me, unde oportuit me ostiatim mendicare, sine omni expensa, donec in Angliam venirem. Audiens autem quod ecclesia illa data esset Galfrido Ridello,\engnotetext{Archdeacon of Canterbury, and a strenous opponent of Becket. He was elected to Ely in \oldstylenums{1173}, and died in \oldstylenums{1189}. In the text of the \emph{Chronicle} there will be found an account of a dispute between him and Samson, as to the bishop's right to demand timber (p.\ \oldstylenums{113} ff.).} contristata est anima mea, eo quod in vanum laboravi. Veniens ergo domum, feretro Sancti \AE{}dmundi latenter me supposui, timens ne dominus abbas me caperet, et incarceraret, qui nichil mali merueram; nec erat monachus qui mecum audebat loqui, nec laicus qui mihi auderet victum ministrare, nisi aliquis furtive. Tandem, inito consilio, misit me abbas apud Acram in exilium,\engnotetext{As to this, see note to p.\ \oldstylenums{6}.} ibique diu moram feci. H\ae{}c et multa alia mala innumerabilia passus sum propter ecclesiam de Wlpet; sed benedictas Deus, qui omnia cooperat in bonum; ecce! ecclesia, pro qua tot mala sustinui, data est in manu mea, et nunc potestatem habeo donandi eam ubi voluero, quia vacat. Et ego eam conventui reddo, et in suos proprios usus assigno antiquam consuetudinem vel pensionem x.\ marcarum, quam perdidistis plus quam lx.\ annis. Integram libentius vobis eam darem, si possem; sed scio, quod episcopus Norwicensis mihi contradiceret, vel si hoc concederet, tali occasione subjectionem et obedientiam de vobis sibi vindicaret, quod est inconsultum et inconveniens. Faciamus ergo quod de jure possumus facere; ponamus clericum vicarium, qui episcopo respondeat de spiritualibus, et vobis de decem marcis; et volo, si vos consulitis, ut vicaria ilia donetur alicui consanguineo R.\ de Hengheham, monachi et fratris vestri, qui mihi fuit consors in illo itinere versus Romam, et eisdem periculis expositus et propter idem negotium.''

\end{otherlanguage}

\switchcolumn

The abbot learnt that the church of Woolpit was vacant, for Walter of Coutances\engnotenum{} had been elected to the bishopric of Lincoln. And presently he summoned together the prior and most of the monastery, and taking up his tale, said: ``You know well what great labours I have undergone in the matter of the church of Woolpit, and how to secure it for your exclusive use I journeyed to Rome by your advice, in the days of the schism between Alexander and Octavian.\engnotenum{} And I traversed Italy at the time when all clerks bearing letters of pope Alexander were seized, some of them being imprisoned and others hanged, and others, after having their noses and lips cut off, were sent back to the pope to his shame and confusion. I, however, pretended that I was a Scot,\engnotenum{} and put on Scottish dress, and adopted the manners of a Scot. And I often shook my staff as they shake the weapon which they call a gaveloc at those who mocked me, shouting threatening words in the manner of the Scots. To those who met me and asked me who I was, I answered nothing except, `Ride, ride Rome, turne Cantwereberei.'\engnotenum{} I acted thus that so I might conceal my purpose, and as a Scot might safely reach Rome. Then when I had obtained from the lord pope such letters as I desired, on my homeward way I passed by a certain castle as the road led me from the city. And, lo! the officers of the castle surrounded me, laying hold on me, and crying, `This wanderer, who makes himself out to be a Scot, is either a spy or one bearing letters of the false pope Alexander.' And while they closely examined my clothes and boots and undergarments, and even the old shoes, which I carried on my shoulders in the Scottish manner, I put my hand into the little bag which I carried, and in which the letter of the lord pope was contained, lying under a little cup from which I was wont to drink. And the Lord God and St.\ Edmund willing it, I drew out the writing and the cup together, so that, stretching my hand on high, I held the writ underneath the cup. And they saw the cup, it is true, but they did not notice the writ. And so I escaped their hands, in the name of the Lord. Whatever money I had on me they took away, so that it was necessary for me to beg for my bread, spending nothing, until I came to England. But when I heard that the church was given to Geoffrey Ridel,\engnotenum{} my soul was grieved with the thought that my labour had been vain. So when I reached home I secretly cast myself before the shrine of St.\ Edmund, for I feared that the lord abbot would seize me and cast me into prison, who had deserved no ill. And there was no monk who dared to speak with me, and no layman who dared supply me with food, save secretly. At length, the abbot took counsel and exiled me to Acre,\engnotenum{} and there I long remained. These and many other countless ills I have suffered for the sake of the church of Woolpit. But blessed be God, Who maketh all things work together for good! Behold! the church for which I have borne so many hardships, is given into my hand, and now I have the power to give it to whomsoever I will, since it is vacant. And I restore it to the monastery, and for its sole use I assign the ancient customary due or pension of ten marks, which you have lost for more than sixty years. I would give it in its entirety to you with pleasure were I able to do so; but I know that the bishop of Norwich would forbid this, or if he were to grant it, he would make it an excuse to demand subjection and obedience from you, which is neither wise nor convenient. Therefore let us do what we may lawfully do. Let us place there a clerk as vicar to answer to the bishop for the spiritualities, and to you for the ten marks; and I wish, if you agree, that the vicarage may be given to some relative of Roger de Hengham, a monk and your brother, who was my companion in that journey to Rome, and was exposed to the same dangers as I was and for the same cause.''

\switchcolumn*

\begin{otherlanguage}{latin}
His dictis omnes surreximus et gratias egimus; et receptus est Hugo clericus, frater pr\ae{}dicti Rogeri, ad pr\ae{}dictam ecclesiam, salva nobis annua pensione x$^\text{cem}$ marcarum.
\end{otherlanguage}

\switchcolumn

At these words we all arose and gave thanks; and Hugh, a clerk and a brother of the said Roger, was received in the said church, saving our annual pension of ten marks.

\switchcolumn*

\begin{otherlanguage}{latin}
\blockhead[c.\ \textsc{a.d}.\ \oldstylenums{1186}]{How the abbot disputed with the archbishop concerning the manor of Eleigh.}{5}{-.6cm}
In manerio monachorum Cantuariensium, quod dicitur Illegga, et quod est in hundredo abbatis, contigit fieri homicidium. Homines vero archiepiscopi noluerunt pati, ut illi homicid\ae{} starent ad rectum in curia sancti \AE{}dmundi. Abbas vero conquestus est regi Henrico, dicens, quod archiepiscopus Baldewinus\engnotetext{Bishop of Worcester (\oldstylenums{1180}--\oldstylenums{83}). He was elected to Canterbury, after some dispute, on the death of Richard of Dover, December \oldstylenums{1183}. (Gervase, I., \oldstylenums{311} ff.) Died, \oldstylenums{1190}, at the siege of Acre.} vindicabat sibi libertates ecclesi\ae{} nostr\ae{}, optentu cart\ae{} nov\ae{} quam rex dederat ecclesi\ae{} Cantuariensi post mortem sancti Thom\ae{}.

\end{otherlanguage}

\switchcolumn

In a manor of the monks of Canterbury, which is called Eleigh, and which is in the hundred of the abbot, there chanced to be a murder. But the archbishop's men would not allow the murderers to take their trial in the court of St.\ Edmund. Then the abbot made complaint to king Henry, and said that archbishop Baldwin\engnotenum{} was claiming the liberties of our church for himself, on the ground of a new charter which the king had given to the church of Canterbury after the death of the blessed Thomas.

\switchcolumn*

\begin{otherlanguage}{latin}
Rex autem respondit, se nunquam fecisse cartam aliquam in pr\ae{}judicium ecclesi\ae{} nostr\ae{} nec aliquid sancto \AE{}dmundo velle auferre, quod habere sole bat. Quo audito, dixit abbas consiliariis privatis suis: ``Sanius consilium est, ut archiepiscopus conqueratur de me, quam ego de archiepiscopo. Volo me ponere in saisinam hujus libertatis, et post me defendam cum auxilio sancti \AE{}dmundi, cujus jus hoc esse cart\ae{} nostr\ae{} testantur.''
\end{otherlanguage}

\switchcolumn

Then the king answered that he had never given a charter to the prejudice of our church, and that he did not wish to take from the blessed Edmund anything which he had formerly possessed. On hearing this, the abbot said to his intimate advisers: ``It is wiser counsel that the archbishop should make complaint of me than that I should make complaint of the archbishop. I wish to place myself in possession of this liberty, and then I will defend myself with the help of St.\ Edmund, in whose right our charters bear witness that this liberty is.''

\switchcolumn*

\begin{otherlanguage}{latin}
Subito ergo, summo mane, procurante Roberto de Cokefeld, missi sunt circiter quater xx.\ homines armati ad villam de Ilegga, et ex inopinato ceperunt illos tres homicidas et ligatos duxerunt ad Sanctum \AE{}dmundum, et in fundum carceris projecerunt. Conquerente inde archiepiscopo, Ranulfus de Glanvilla justiciarius pr\ae{}cepit, ut homines illi ponerentur per vadium et plegios ad standum ad rectum in curia qua deberent stare, et summonitus est abbas, ut veniret ad curiam regis, responsurus de vi et injuria, quam dicebatur fecisse archiepiscopo. Abbas vero sine omni exonio se pluries pr\ae{}sentavit.
\end{otherlanguage}

\switchcolumn

Accordingly, unexpectedly, and very early in the morning, with the help of Robert de Cokefield, about eighty armed men were sent to the town of Eleigh, and took those three murderers by surprise and brought them bound to St.\ Edmund's, and cast them into the dungeon of the prison. And when the archbishop made complaint of this, Ranulf Glanvill, the justiciar, commanded that those men should be bound by surety and pledges to stand their trial in the court wherein they ought to stand it; and the abbot was summoned to come to the court of the  king and to make reply concerning the violence and injury which he was said to have done to the archbishop. And the abbot many times presented himself at the court, without attempting to make excuse.

\switchcolumn*

\begin{otherlanguage}{latin}
\blockhead[\textsc{a.d}.\ \oldstylenums{1187} February \oldstylenums{11}.]{}{2}{-.7cm}
Tandem in capite jejunii steterunt\engnotetext{Gervase (I., \oldstylenums{353}) mentions the meeting and supplies the date.} coram rege in capitulo Cantuariensi, et lect\ae{} sunt palam cart\ae{} ecclesiarum hinc et inde. Et respondit dominus rex: ``Ist\ae{} cart\ae{} ejusdem antiquitatis sunt et ab eodem rege \AE{}dwardo emanant. Nescio quid dicam nisi ut cart\ae{} ad invicem pugnent.''\footnote[\textdagger]{In margine hic legitur: ``Quia carta quam habemus de sancto \AE{}dwardo antiquior est, quam carta quam habent monachi Cantuarienses. Quia carta, quam habent, non data eis libertatem, nisi inter homines suos tantum: et carta nostra loquitur de tempore regis \AE{}dwardi et de tempore matris su\ae{} regin\ae{} Emm\ae{}, qu\ae{} habuit viii.\ hundredos et dimidium in dotem, ante tempora sancti \AE{}dwardi, et Mildenhale insimul'' (Roke.).} Cui abbas dixit: ``Quicquid de cartis dicatur, nos in saisina sumus, et hucusque fuimus, et de hoc ponere me volo in verumdictum duorum comitatuum, scilicet, Norfolchi\ae{} et Suthfolchi\ae{}, se hoc concedere.''\engnotetext{That is, will admit that the abbey has possessed this jurisdiction from time immemorial. (\emph{Mem}., I., \oldstylenums{256}, note.)}

\end{otherlanguage}

\switchcolumn

At last, at the beginning of the fasting time,\engnotenum{} they stood before the king in the chapter-house of Canterbury, and the charters of the two churches were read publicly. And the lord king answered, ``These charters are of equal age, and come from the same king Edward. I know not what to say, save that the charters are contradictory.'' To this the abbot replied, ``Whatever may be said about the charters, we are seised of the liberty, and have been in the past, and on this point I will submit to the verdict of the two counties, Norfolk and Suffolk, which will allow this.''\engnotenum{}

\switchcolumn*

\begin{otherlanguage}{latin}
Sed archiepiscopus Baldwinus, habito prius consilio cum suis, dixit, homines Norfolchi\ae{} et Suthfolchi\ae{} multum diligere Sanctum \AE{}dmundum, et magnam partem illorum comitatuum esse sub ditione abbatis, et ideo se nolle stare illorum arbitrio. Rex vero iratus inde et indignans surrexit, et recedendo dixit: ``Qui potest capere capiat:''\engnotetext{Matt.\ xix., \oldstylenums{12}.} et sic res cepit dilationem, ``et adhuc sub judice lis est.''\footnote[\textdagger]{Hor.\ de Arte Poet.\ \oldstylenums{78}.}
\end{otherlanguage}

\switchcolumn

Archbishop Baldwin, however, having first taken counsel with his men, said that the men of Norfolk and Suffolk loved St.\ Edmund greatly, and that a large part of those counties was under the rule of the abbot, and therefore he would not abide by their arbitration. But the king was angry and offended at that, and rising up, left the place, saying, ``He that is able to receive it, let him receive it.''\engnotenum{} And thus the matter was postponed, and is still undecided.

\switchcolumn*

\begin{otherlanguage}{latin}
Vidi tamen, quod quidam homines monachorum Cantuariensium vulnerati fuerunt usque ad mortem a rusticis de villa de Meldingis, que sita est in hundredo Sancti \AE{}dmundi; et qui sciverunt, quod actor forum rei sequi debet, maluerunt silere et dissimulare, quam inde clamorem facere abbati sive baillivis ejus, quia nullo modo voluerunt venire in curiam Sancti \AE{}dmundi ad placitandum.
\end{otherlanguage}

\switchcolumn

But I saw that some of the men of the monks of Canterbury were wounded to the death by the rustics of the township of Midling, which is situated in the hundred of St.\ Edmund, and as they knew that the prosecutor is bound to go to the court of the defendant, they preferred to be silent and to hide the matter, rather than complain of it to the abbot or his officers, since they were in nowise willing to come and plead in the court of St.\ Edmund.

\switchcolumn*

\begin{otherlanguage}{latin}
\blockhead[\textsc{a.d}.\ \oldstylenums{1191}.]{}{2}{-.7cm}
Postea levaverunt homines de Illegga quoddam trebuchet, ad faciendam justitiam pro falsis mensuris panis vel bladi mensurandi, unde conquestus est abbas domino Eliensi episcopo,\engnotetext{William Longchamp, elected to Ely and made justiciar and joint regent at the accession of Richard I. He was driven from office in \oldstylenums{1191}, owing to the opposition of John and Hugh Puiset, bishop of Durham. He died in \oldstylenums{1197}.} tunc justiciario et cancellario. Ille vero abbatem audire nolebat, quia dicebatur olfacere archiepiscopatum,\engnotetext{The archbishopric remained vacant from the death of Baldwin until \oldstylenums{1191}---over a year---when Reginald, bishop of Bath, was elected. He died less than fifteen days later. (Benedictus, II., \oldstylenums{226}--\oldstylenums{7}.) The archbishopric then remained vacant again until the election of Hubert Walter, \oldstylenums{1193}.} qui vacabat tunc temporis. Cum autem venisset apud nos, et susceptus esset ut legatus, antequam recederet, orationem fecit ad feretrum sancti martyris; abbasque, nacta opportunitate, dixit, cunctis audientibus qui aderant: ``Domine episcope, libertas, quam sibi vindicant monachi Cantuarienses, est jus sancti \AE{}dmundi, cujus corpus pr\ae{}sens est, et quia non vis me adjuvare ad tuendam libertatem ecclesi\ae{} su\ae{}, pono loquelam inter te et ipsum. Ipse de c\ae{}tero procuret jus suum.'' Cancellarius nichil dignatus est respondere;\engnotetext{The pride of Longchamp is the common theme of all the chroniclers, who unite in attributing his fall to this fault. Having obtained the legation from Clement III., he claimed to be supreme over church and state alike, refusing (perhaps in accord with Richard's instructions) to admit that he had any colleague in the office of justiciar. His character is sketched in a letter of Hugh Nunant, bishop of Coventry, which is to be found in Benedictus (II., \oldstylenums{215}--\oldstylenums{20}). ``He was a man great among all the people of the west, as one having power over the kingdom and the authority of the apostolic see, being as it were ambidextrous. . . . He seemed, indeed, to divide the elements with God, leaving heaven alone to the God of heaven. . . . From sea to sea he was feared as God, and were I to say more than God, I should not lie, for God is long-suffering and merciful, but he did all things ill and in haste.'' William of Newburgh calls him ``that rhinocerous.'' Richard of Devizes calls him ``that three-named and threefold man.''} qui infra  annum Angliam exire compulsus est, et divinam ultionem expertus est.

\end{otherlanguage}

\switchcolumn

After these things the men of Eleigh set up a certain measure for the doing of justice in cases where bread and corn had been measured with false measures, and the abbot made complaint of this to the lord bishop of Ely,\engnotenum{} who was at that time justiciar and chancellor. But he would not hear the abbot, because he was alleged to be scenting the archbishopric,\engnotenum{} which was then vacant. When, however, he had come among us, and was received as legate, before he departed, he made prayer at the shrine of the holy martyr. And the abbot, seizing the opportunity, said in the hearing of all who were present, ``My lord bishop, the liberty, which the monks of Canterbury claim, is the right of St.\ Edmund, whose body is here, and as you will not assist me to protect the liberty of his church, I put a complaint between you and him. Henceforth he may secure his right.'' The chancellor did not condescend to make any answer,\engnotenum{} and within a year was forced to leave England, and suffered divine vengeance.

\switchcolumn*

\begin{otherlanguage}{latin}
\blockhead[\textsc{a.d}.\ \oldstylenums{1193}.]{}{2}{-.7cm}
Cum autem idem cancellarius redisset de Almannia,\engnotetext{Longchamp, after failing to persuade the regency to permit his return before, came back to England in \oldstylenums{1193}, with letters from Henry VI.\ as to the ransom of Richard. (Hoveden, III., \oldstylenums{211}.)} et applicuisset apud Gippewic, et pernoctasset apud Heggham, venit rumor ad abbatem, quod cancellarius vellet transire per Sanctum \AE{}dmundum, apud nos missam in crastino auditurus. Prohibuit ergo abbas, ne celebrarentur divina, dum cancellarius esset in ecclesia pr\ae{}sens, quia dixit se audisse, apud Londonias, Londoniensem episcopum pronuntiasse cancellarium esse excommunicatum, et excommunicatum\footnote[\textdagger]{In Dr.\ Lingard's \emph{Hist.\ of Engl}.\ no mention is made of this excommunication. Yet the Chronicle of Bromton (Twys.\ \oldstylenums{1225}) distinctly says, that when in Oct.\ \oldstylenums{1191}, Geoffrey, Archbishop of York, on being released from confinement, came up to London, the archbishop of Rouen (Walter do Coutances) and six English bishops pronounced a sentence of excommunication against the chancellor who had imprisoned him. Diceto probably gives the real facts; he says that the archbishop and bishops pronounced the excommunication at Reading, in general terms, against ``omnes qui consilium, vel auxilium, vel mandatum dederunt, ut archiepiscopus Eboracensis extraheretur ab ecelesia, tractaretur indigne, \&c.,'' and specially against two confidential knights of the chancellor, Alberic de Marines and Alexander Puintel. (ii.\ \oldstylenums{98}, Rolls ed.). With this agrees in substance the account given by Gervase (i.\ \oldstylenums{507}, Rolls ed.). Newburgh, Hoveden, and bishop Hugh de Nunant (in the witty and abusive letter preserved by Benedictus Abbas) make no mention of any excommunication.} recessisse ab Anglia, coram sex episcopis, et nominatim pro violentia illata archiepiscopo Eboracensi\engnotetext{Richard had forbidden Geoffrey to visit England for three years, but the archbishop landed at Dover. There he was arrested by order of Longchamp, but released on the intervention of John. (Benedictus, II., \oldstylenums{106}; \oldstylenums{209}--\oldstylenums{11}.) As to the excommunication of Longchamp, this is mentioned by Benedictus (II., \oldstylenums{211}--\oldstylenums{12}), and by Richard of Devizes (pp.\ \oldstylenums{36}, \oldstylenums{43}, \oldstylenums{56}). Ralph de Diceto (II., \oldstylenums{98}) makes the excommunication general only.} apud Doffram.\footnote[\textdagger]{Geoffrey, who had just been consecrated at Tours, by pope Celestine's order, to the see of York, had sworn to the king, his half brother, that he would not return to England without his leave. He did return however; and on his landing at Dover (Sept.\ \oldstylenums{1191}) was arrested by the chancellor's orders, and thrown into prison.}

\end{otherlanguage}

\switchcolumn

But when the same chancellor had returned from Germany\engnotenum{} and had landed at Ipswich, and spent the night at Hitcham, a report came to the abbot that the chancellor wished to pass through St.\ Edmund's, and to hear mass with us on the morrow. Therefore the abbot forbade the celebration of the divine offices while the chancellor was present in the church, for he said that he had heard in London that the bishop of London had pronounced the chancellor excommunicate, in the presence of six bishops, especially for the violence which he had done to the archbishop of York,\engnotenum{} at Dover, and that the said chancellor, while excommunicate, had departed from England.

\switchcolumn*

\begin{otherlanguage}{latin}
Veniens ergo\footnote[\ddag]{Longchamp fled the kingdom in \oldstylenums{1191}, after his fall from power. ``He came to England in the following year, but was not suffered to proceed farther than Canterbury, and crossed the seas again. In \oldstylenums{1193} the chancellor returned, bearing letters from the emperor, and met the Regency at St.\ Alban's. It was on this occasion that he passed through St.\ Edmundsbury, coming from his manor of Hitcham, after landing at Ipswich.'' (Rokewode.)} in crastino cancellarius apud nos, non invenit qui missam ei cantaret, nec clericum, nec monachum. Immo, sacerdos stans ad primam missam ad canonem miss\ae{}, et ceteri sacerdotes, ad altaria cessaverunt, stantes inmotis labiis, donec nuntius veniret dicens, illum recessisse ab ecclesia. Cancellarius omnia dissimulans, multa gravamina intulit abbati, donec, procurantibus amicis, hinc et inde ad pacis osculum reversi sunt.
\end{otherlanguage}

\switchcolumn

Accordingly, when the chancellor came among us on the morrow, he found no one to chant mass for him, either clerk or monk. But the priest, indeed, who stood at the first mass and at the canon of the mass, and the other priests by the altars, ceased, and stood with unmoved lips, until a messenger came and said that he had left the chnrch. The chancellor took no notice openly, but he did many ills to the abbot, until, by the mediation of friends, they both returned to the kiss of peace.

\switchcolumn*

\begin{otherlanguage}{latin}
\blockhead[\textsc{a.d}.\ \oldstylenums{1188} January \oldstylenums{21}.]{How the abbot wished to take the cross, and how he offered to seek king Richard in Germany.}{5}{-.55cm}
Cum rex Henricus accepisset crucem\engnotetext{At Gisors, \oldstylenums{1188}.} et venisset infra mensem ad nos orationis gratia, abbas ipse sibi fecit crucem occulte de lineo panno, et tenens in una manu crucem et acum et filum, petivit licentiam a rege, ut acciperet crucem; sed denegata est ei licentia, procurante episcopo Norwicensi Johanne,\engnotetext{John of Oxford, elected to Norwich in \oldstylenums{1175}; died, \oldstylenums{1200}. As a matter of fact, John did not go to Palestine. He was waylaid and robbed on his journey, and obtained absolution from his vow from the pope. Richard made this excuse for levying a heavy fine. (Hoveden, III., \oldstylenums{42}. Richard of Devizes, p.\ \oldstylenums{12}.)} et dicente, quia non expediret patri\ae{}, nec tutum esset comitatibus Norfolchi\ae{} et Sutfolchi\ae{}, si episcopus Norwicensis et abbas Sancti \AE{}dmundi simul recederent.

\end{otherlanguage}

\switchcolumn

When king Henry had taken the cross\engnotenum{} and was come less than a month later that he might pray among us, the abbot secretly made for himself in one hand the cross and a needle and thread, he sought leave from the king that he might take the cross. But leave was refused him, for John, bishop of Norwich,\engnotenum{} opposed it, and said that it was not well for the land, nor safe for the counties of Norfolk and Suffolk, that the bishop of Norwich and the abbot of St.\ Edmund's should go away at the same time.

\switchcolumn*

\begin{otherlanguage}{latin}
\blockhead[\textsc{a.d}.\ \oldstylenums{1193}.]{}{2}{-.7cm}
Cum rumor venisset Lundoniis de captione regis Ricardi,\engnotetext{Richard was captured in December, \oldstylenums{1192}, near Vienna, and imprisoned at Durrenstein. In the following year he was handed over to the emperor Henry VI.\ and imprisoned in various castles in Germany. Hoveden (III., \oldstylenums{198}) only mentions the abbots of Boxley and Robertsbridge as going to search for Richard, and it would seem that Samson merely offered to go, though it appears that he did subsequently visit Germany (see text, p.\ \oldstylenums{87}).} et incarceratione ejus in Alemannia, et barones convenissent pro consilio accipiendo, prosiliit abbas coram omnibus, dicens, se esse paratum qu\ae{}rere dominum suum regem, vel in tapinagio vel alio modo, donec eum inveniret, et certam notitiam de eo haberet; ex quo verbo magnam laudem sibi adquisivit.

\end{otherlanguage}

\switchcolumn

When news had reached London of the capture of king Richard,\engnotenum{} and of his imprisonment in Germany, and the barons had met to take counsel on the matter, the abbot stood forth in their presence, and said that he was ready to seek his lord the king. He said that he would search for him in disguise or in any other way, until he found him and had certain knowledge of him. And from this speech he gained great praise for himself.

\switchcolumn*

\begin{otherlanguage}{latin}
\blockhead[\textsc{a.d}.\ \oldstylenums{1190}.]{How the abbot resisted the authority of the legate.}{4}{-.45cm}
Cum cancellarius, episcopus scilicet Eliensis, legati fungeretur officio et concilium celebraret apud Londoniam,\footnote[\textdagger]{Gervase (i.\ \oldstylenums{488}, Rolls ed.), whose eyes seldom travel outside of Canterbury, describes the battle about precedence and privelege which ushered in this council, but has not one word to say of its proceedings. Matthew Paris (or rather Wendover), and Hemingford, speaking of the chancellor with strong prejudice, say that little or nothing was done in the council that was of any benefit to the Church. (Wilkins, Conc.\ i.\ \oldstylenums{493}.)}\engnotetext{This council is mentioned by Benedictus (II., \oldstylenums{106}). Richard of Devizes (p.\ \oldstylenums{14}) mentions that at this council Longchamp gave judgment that the monks should be expelled from Coventry, in accordance with the wishes of Hugh Nunant.} et qu\ae{}dam decreta proposuisset contra nigros monachos, loquens de vagatione eorum ad sanctum Thomam et ad sanctum \AE{}dmundum peregrinationis obtentu, et contra abbates loquens, pr\ae{}finiens eis certum numerum equorum: respondit abbas Samson: ``Nos non recipimus aliquod decretum contra regulam sancti Benedicti, qu\ae{} permittit abbatibus liberam dispositionem habere de monachis suis. Ego vero baroniam sancti \AE{}dmundi servo et regnum ejus; nec sufficiunt mihi tredecim equi, sicut quibusdam aliis abbatibus, nisi plures habeam ad executionem regi\ae{} justiti\ae{} conservand\ae{}.''

\end{otherlanguage}

\switchcolumn

When the chancellor, the bishop of Ely, filled the office of legate and held a council at London,\engnotenum{} he proposed certain decrees against the black monks, talking of their wandering to the shrines of St.\ Thomas and of St.\ Edmund under pretence of pilgrimage, and speaking against the abbots, mentioning a certain number of horses which they ought to have. Then abbot Samson answered, ``We will not receive any decree which is contrary to the rule of St.\ Benedict, which allows the abbots a free hand in the control of their monks. And I serve the barony of St.\ Edmund and his realm; thirteen horses are not enough for me, as they may be for some other abbots, unless I have additional horses for the administration of the justice of the king.''

\switchcolumn*

\begin{otherlanguage}{latin}
\blockhead[\textsc{a.d}.\ \oldstylenums{1193}.]{Of the conduct of the abbot while king Richard was in captivity.}{4}{-.65cm}
Cum esset werra in tota Anglia, capto rege Ricardo, abbas cum toto conventu sollemniter excommunicavit omnes factores werr\ae{} et pacis turbatores, non timens comitem Johannem fratrem regis nec alium, unde abbas magnanimus dicebatur. Post quod  factum ivit ad obsidionem de Windleshor,\engnotetext{John, acting in concert with Philip Augustus, availed himself of the opportunity afforded by Richard's capture to stir up disorder in England. He captured Windsor, and demanded the fealty of the kingdom from the justiciars. On their refusal, he fortified his castles against them. The justiciars, however, under the direction of Hubert Walter, acted promptly, and soon compelled the surrender of all John's castles, except Nottingham and Tickhill. Windsor was handed over to the custody of Eleanor. (Hoveden, III., \oldstylenums{204}--\oldstylenums{7}.)} ubi armatus cum quibusdam aliis abbatibus Angli\ae{}, vexillum proprium habens, et plures milites ducens ad multas expensas, plus ibi consilio quam probitate nitens.\footnote[\textdagger]{The construction is incomplete.}

\end{otherlanguage}

\switchcolumn

After the capture of king Richard, there was war throughout England. And the abbot with the whole monastery solemnly excommunicated all who stirred up war and broke the peace, showing no fear of earl John, the king's brother, or of any other, whence he was called a great-souled abbot. And after doing this, he went to the siege of Windsor,\engnotenum{} where he appeared in arms with some other abbots of England, and had his own standard. He had there also with him many knights at great expense, and he gained a reputation rather for skill in the council than for virtue.

\switchcolumn*

\begin{otherlanguage}{latin}
Nos vero claustrales tale factum periculosum judicavimus, timentes consequentiam, ne forte futurus abbas cogatur in propria persona ire in expeditionem bellicam. Datis indutiis illo tempore ivit in Alemanniam, et ibi visitavit regem cum donis plurimis.
\end{otherlanguage}

\switchcolumn

But we who were cloistered monks considered this course of action to be fraught with danger, fearing lest some future abbot might be compelled to go to war in person. Then, when a truce had been granted, he went to Germany, and there sought the king with many gifts.

\switchcolumn*

\begin{otherlanguage}{latin}
\blockhead[\textsc{a.d}.\ \oldstylenums{1193}.]{Concerning that which befel certain knights who desired to hold a tournament contrary to the wish of the abbot.}{6}{-.55cm}
Post reditum regis Ricardi in Angliam, data est licentia torneandi\engnotetext{The object of this provision is explained by Diceto (II., \oldstylenums{120}--\oldstylenums{1}), and was to raise money for the king by the sale of licences. The rates at which the licences were sold were twenty silver marks for an earl, ten for a baron, three for a knight holding land, and two for a knight not holding land. (Hoveden, III., \oldstylenums{268}.)} militibus. Ad quod faciendum convenerunt multi inter Theford et Sanctum \AE{}dmundum, sed prohibuit eos abbas; qui resistentes, votum suum impleverunt. 

\end{otherlanguage}

\switchcolumn

After the return of king Richard to England, leave was granted to knights to hold tournaments,\engnotenum{} and many gathered between Thetford and St.\ Edmund's for this purpose. Then the abbot forbade them, but they resisted him, and did all that they desired.

\switchcolumn*

\begin{otherlanguage}{latin}
\blockhead[June \oldstylenums{28}.]{}{2}{-.3cm}
Alia vice venerunt quatuor viginti juvenes cum sectis suis, filii nobilium, ad vindicium faciendum cum plenis armis ad pr\ae{}dictum locum. Quo perfecto, redierunt in villam istam causa hospitandi. Abbas vero hoc audiens, portas jussit obserari, et eos omnes includi. Crastinus dies erat vigilia apostolorum Petri et Pauli. Fide ergo interposita, promittentes se non exire, nisi per licentiam, manducaverunt omnes cum abbate illo die; sed post prandium, abbate eunte in talamum suum, surrexerunt omnes et inceperunt carolare et cantare, mittentes in villam propter vinum, bibentes, et postea ululantes, abbati et conventui somnum suum auferentes, et omnia in derisum abbatis facientes, et diem usque ad vesperam hoc modo deducentes, nec propter mandatum abbatis voluerunt desistere.

\end{otherlanguage}

\switchcolumn

On another occasion twenty-four young men, sons of nobles, came fully armed with their followers to have their revenge at the same place. And when they had made an end, they returned to the town to lodge there. But when the abbot heard of it, he commanded the gates to be closed and all of them to be shut within the town. The following day was the vigil of the apostles Peter and Paul. Therefore, when they had given their word that they would not leave the town without permission, they all ate with the abbot on that day. After the meal, when the abbot had retired to his private chamber, they all arose and began to laugh and sing. And sending to the town for wine, they drank and afterwards shouted loudly, and so they kept the abbot and the monastery from their sleep, and did everything to mock the abbot. So they spent the day until the evening, nor would they cease at the abbot's command.

\switchcolumn*

\begin{otherlanguage}{latin}
Vespere vero adveniente, seras portarum vill\ae{} fregerunt, et vi exierunt. Abbas vero omnes sollemniter excommunicavit, per consilium tamen archiepiscopi Huberti\engnotetext{Hubert Walter, dean of York (\oldstylenums{1188}), and bishop of Salisbury (\oldstylenums{1189}). He went on the crusade, and was present at the siege of Acre. He was elected archbishop in \oldstylenums{1193}, and died in \oldstylenums{1205}.} justiciarii tunc temporis; quorum multi venerunt ad emendationem, absolutionem petentes.
\end{otherlanguage}

\switchcolumn

Finally, when evening was come, they broke the bars of the gates of the city, and forced their way out. The abbot solemnly excommunicated them all, by the advice of archbishop Hubert,\engnotenum{} who was then justiciar, and many of them made reparation and sought absolution.

\switchcolumn*

\begin{otherlanguage}{latin}
\blockhead[\textsc{a.d}.\ \oldstylenums{1182}, \oldstylenums{1187}, \& \oldstylenums{1188}.]{Concerning the missions of the abbot to the papal court.}{4}{-.65cm}
Romam misit abbas s\ae{}pius nuntios suos, non vacuos. Primi quos misit, statim postquam fuit benedictus, impetraverunt m genere omnes libertates et consuetudines qu\ae{} concess\ae{} fuerant prius, etiam tempore schismatis, pr\ae{}decessoribus suis; postea impetravit, primus inter abbates Angli\ae{}, quod dare posset episcopalem benedictionem sollemniter ubicunque fuerit; et hoc est impetratum sibi et successoribus suis. Postea impetravit exemptionem generalem sibi et successoribus suis ab omnibus archiepiscopis Cantuariensibus, quam abbas H.\ pr\ae{}decessor suus specialiter sibi adquisierat. Plures et novas libertates in illis confirmationibus apponi fecit abbas Samson, ad majorem libertatem et securitatem ecclesi\ae{} nostr\ae{}.

\end{otherlanguage}

\switchcolumn

The abbot sent many messengers to Rome, and not in vain. The first whom he sent, immediately after his benediction, obtained in detail all the liberties and rights which had been granted to his predecessors, even in the days of schism. Afterwards he obtained that he might give episcopal benediction wherever he might be, and he was the first of the abbots of England to gain this. This right he won for himself and for his successors. At a later date he acquired complete exemption for himself and for his successors from all the archbishops of Canterbury, a privilege which abbot Hugh had secured for himself alone. In these confirmations of privileges abbot Samson caused the inclusion of many new liberties, to the great freedom and safety of our church.

\switchcolumn*

\begin{otherlanguage}{latin}
Venit quidam clericus ad abbatem portans literas petitorias de redditu ecclesiastico habendo. Et abbas extrahens de scrinio suo septem scripta apostolica cum bullis pendentibus, ita respondit: ``Ecce scripta apostolica, quibus diversi apostolici diversis clericis ecclesiastica beneficia petunt dari. Cum ergo illos pacavero qui pr\ae{}venerunt, tibi redditum dabo, quia qui prius venit ad molendinum prius molere debet.''
\end{otherlanguage}

\switchcolumn

Then a certain clerk came to the abbot, bearing letters asking for the grant of some ecclesiastical benefice. And the abbot drew from his desk seven apostolic letters, with leaden seals hanging to them, and answered as follows: ``See the apostolic letters, whereby different popes seek that ecclesiastical benefices may be given to this or that clerk. When then I have satisfied those who come first I will give you a benefice, since he who first comes to the mill ought to grind first.''

\switchcolumn*

\begin{otherlanguage}{latin}
\blockhead{How the abbot met the claim of Earl de Clare to bear the standard of Saint Edmund.}{4}{-.65cm}
Facta est summonitio magna in hundredo de Risebrigga, ut audiretur querela et rectum comitis de Clara\engnotetext{Richard de Clare, fourth earl of Hertford (d.\ \oldstylenums{1218}), one of the twenty-five barons appointed to enforce the observance of Magna Carta.} apud Witham. Ipse vero constipatus multis  baronibus et militibus, comite Alberico\engnotetext{Alberic de Vere, first earl of Oxford (d.\ \oldstylenums{1194}). For his subsequent quarrel with Samson, see text, p.\ \oldstylenums{106}.} et multis aliis assistentibus, dixit; quod ballivi sui fecerunt ei intelligere, quod ipsi solebant annuatim accipere ad opus suum v.\ solidos de hundredo et ballivis hundredi, et nunc detinerentur injuste; et allegabat, quod pr\ae{}decessores sui fuerunt feoffati, ad captionem Angli\ae{}, de terra Alfrici filii Withari;\footnote[\textdagger]{Wisgari, Lib.\ Domesday, \oldstylenums{389} b.; Withgari, ibid.\ \oldstylenums{390}. (Roke.)}\engnotetext{For details as to his holding, see Rokewode, p.\ \oldstylenums{129}.} qui quondam fuerat dominus illius hundredi.

\end{otherlanguage}

\switchcolumn

A general summons was issued in the hundred of Risbridge that the complaint and trial of earl de Clare\engnotenum{} might be heard at Witham. And he was accompanied by a great crowd of barons and knights, while Earl Alberic\engnotenum{} and many others were present, and he said that his officers had given him to understand that they were wont to receive five shillings for his use from the hundred and from the officers of the hundred, which were now wrongfully withheld. He also alleged that his predecessors had been enfeoffed with the land of Alfric, son of Withgar,\engnotenum{} who was formerly lord of that hundred, at the time of the conquest of England.

\switchcolumn*

\begin{otherlanguage}{latin}
Abbas vero sibi consulens, nec de loco se movens, respondit: ``Mirum videtur domine comes; quod dicis deficit. Rex \AE{}dwardus dedit Sancto \AE{}dmundo et carta sua confirmavit hunc hundredum integre, et de illis v.\ solidis nulla fit ibi mentio. Dicendum est tibi, pro quo servitio, vel qua ratione exigis illos v.\ solidos.'' Et comes, habito consilio suorum, respondit se debere portare vexillum Sancti \AE{}dmundi in exercitu, et ob hanc causam illos v.\ solidos sibi deberi. Et respondit abbas: ``Certe, inglorium esse videtar si tantus vir, utpote comes Clarensis, tam parvum donum pro tali servitio recipiat: abbati autem Sancti \AE{}dmundi parvum gravamen est dare v.\ solidos. Comes R.\ Bigot\engnotetext{Second earl of Norfolk, died \oldstylenums{1221}. One of the twenty-five barons of Magna Carta.} se saisiatum tenet, et saisiatum se asserit officio portandi vexillum Sancti \AE{}dmundi,\engnotetext{Rokewode (pp.\ \oldstylenums{129}--\oldstylenums{31}) quotes a passage from Lydgate's metrical life of St.\ Edmund on the subject. The whole poem was printed by C.\ Horstmann in his \emph{Altenglische Legenden}, Heilbron, \oldstylenums{1881}. (See \emph{Mem}.\ III., l.--li.) A reproduction of the standard is published as a tailpiece to Rokewode's edition.} qui illud portavit quando comes Lehecestri\ae{} fuit captus et Flandrenses destructi. Thomas etiam de Mendham\engnotetext{He held a knight's fee in Livermere. (See Rokewode, p.\ \oldstylenums{132}, who appends other details.) He was possibly descended from a brother of abbot Baldwin, and seems to have been a constable to the monastery.} dicit hoc esse jus suum. Cum vero dirationaveris versus eos hoc esse jus tuum, ego libenter v.\ solidos, quos qu\ae{}ris, persolvam.'' Comes vero respondit, se esse locuturum inde cum comite R.\ cognato suo, et sic res cepit dilationem usque hodie.
\end{otherlanguage}

\switchcolumn

But the abbot, considering in his own mind and not stirring from his place, answered, ``My lord earl, your words astonish me! Your case is not proven. King Edward gave to St.\ Edmund, and confirmed the gift in his charter, this whole hundred, and no mention was made of these five shillings. Do you tell us for what service, or on what grounds, you claim those five shillings!'' And the earl, having consulted his men, replied that he had the right to bear the standard of St.\ Edmund in the host, and for this reason those five shillings were due to him. Then the abbot answered, ``In truth, it seems an unworthy thing that so great a man as the earl de Clare should receive so small a gift for so important a service, but it is small harm to the abbot of St.\ Edmund's to give five shillings. Earl Roger Bigod\engnotenum{} holds that he is seised of the land, and declares that he is seised of the duty of bearing the standard of St.\ Edmund,\engnotenum{} and he bore it when the earl of Leicester was taken and the Flemings destroyed. Moreover, Thomas de Mendham\engnotenum{} says that this right belongs to him. But when you shall have proved against them that this right is yours, I will willingly pay the five shillings which you demand.'' Then the earl answered that he would speak of the matter to earl Roger his relative, and so the matter has not been decided to this day.

\switchcolumn*

\begin{otherlanguage}{latin}
\blockhead[\textsc{a.d}.\ \oldstylenums{1191}.]{Concerning the case of Adam de Cokefield.}{4}{-.45cm}
Mortuo Roberto de Kokefeld, venit Adam filius ejus et cum eo parentes sui, comes R.\ Bigot, et alii multi potentes, et sollicitantes abbatem de tenementis pr\ae{}dicti Ad\ae{}, et pr\ae{}cipue de dimidio hundredo de Cosford tenendo per annuum censum c.\ solidorum, tanquam hoc esset jus suum h\ae{}reditarium, et dicentes quod pater ejus et avus ejus tenuerunt illud quater xx.\ annis retro et plus.

\end{otherlanguage}

\switchcolumn

Now, when Robert de Cokefield was dead, his son Adam came with his relatives, earl Roger Bigod, and many other powerful men. And they sought from the abbot the holdings of the said Adam, and especially for half the hundred of Cosford, to be held at an annual rent of a hundred shillings, as if this was his by hereditary right, and declared that his father and grandfather had held it for eighty years past and more.

\switchcolumn*

\begin{otherlanguage}{latin}
Abbas vero, nacta opportunitate loquendi, apponens duos digitos suos ad duos oculos suos, dixit: ``Illa die et illa hora perdam oculos istos, qua alicui concedam hundredum h\ae{}reditarie tenendum, nisi rex inde vim mihi faciat, qui mihi potest auferre abbatiam et vitam.'' Ostendensque rationem dicti, ait: ``Si aliquis teneret hundredum h\ae{}reditarie, et ipse forisfaceret versus regem aliquo modo ita quod exh\ae{}reditari deberet, statim vicecomes Sutfolchi\ae{} et ballivi regis saisiarent hundredum; et exercerent potestatem suam infra terminos nostros; et si haberent custodiam hundredi, periclitaretur libertas octo hundredorum\engnotetext{The grant of this land, which had been held by Emma of Normandy, wife of Ethelred II., was granted to St.\ Edmund by Edward the Confessor on the occasion of his visit to the monastery in \oldstylenums{1044}. (\emph{Mem}., I., \oldstylenums{363}, cp.\ Rokewode, pp.\ \oldstylenums{132}--\oldstylenums{33}, and Hermannus, in \emph{Mem}., I., \oldstylenums{48}.) The hundreds were Thingoe, Thedwastre, Blackburne, Bradborn, Bradmere, Risbridge, Babery, and half that of Exning. (\emph{Mem}., I., xxviii., note \oldstylenums{1}.) The charter is printed in Dugdale's \emph{Monasticon} (III., \oldstylenums{138}, No.\ vii., ed.\ \oldstylenums{1846}).} et dimidii.''\footnote[\textdagger]{See ante, pp.\ \oldstylenums{48}, \oldstylenums{128}.}
\end{otherlanguage}

\switchcolumn

But the abbot, when he had opportunity for speaking, put his two fingers to his two eyes, and said, ``May I lose these eyes in that day and hour in the which I grant a hundred to any man to hold by hereditary right, unless the king, who is able both to take away the abbacy from me and to deprive me of life, force me to do this thing.'' And he gave the reason for this saying, and added, ``If any man hold a hundred by hereditary right, and if he commit an offence of any sort against the king for the which he ought to be disinherited, forthwith the sheriff of Suffolk and the officers of the king would seize the hundred, and would exercise authority within our borders. And if they should have the wardenship of this hundred, then would the liberty of eight hundreds and a half\engnotenum{} be imperilled.''

\switchcolumn*

\begin{otherlanguage}{latin}
Convertensque sermonem ad Adam, ait: ``Si tu, qui clamas h\ae{}reditatem in illo hundredo, acciperes in uxorem aliquam liberam feminam, qu\ae{} teneret saltem unam acram terr\ae{} de rege in capite, rex post mortem tuam saisiaret totum tenementum tuum et wardam filii tui, si esset infra \ae{}tatem; et ita ballivi regis intrarent in hundredum Sancti \AE{}dmundi in pr\ae{}judicium abbatis. Ad hoc, pater tuus recognovit mihi, se nihil juris h\ae{}reditarii vendicare de hundredo; et, quia servitium suum mihi placuit, permisi eum tenere omnibus diebus vit\ae{} su\ae{}, meritis suis exigentibus.''
\end{otherlanguage}

\switchcolumn

Then he addressed himself to Adam, and said, ``If you, who claim hereditary right in this hundred, were to marry any free woman, who held even a single acre of land from the king in chief, the king after your death would have seisin of the whole of your land and the wardship of your son, if he were a minor, and thus the officers of the king would enter the hundred of St.\ Edmund to the prejudice of the abbot. For the matter of that, your father made acknowledgment to me that he claimed no hereditary right in that hundred, and because his service pleased me, I permitted him to hold it all the days of his life, as his deserts warranted.''

\switchcolumn*

\begin{otherlanguage}{latin}
His dictis, oblata est abbati pecunia multa: sed nec potuit flecti nec prece nec pretio. Convenit tandem inter eos ita: Adam renunciavit juri suo quod ore dicebat se habere in hundredo, et abbas confirmavit ei omnes alias terras suas,\engnotetext{The lands confirmed to him were those which he held in Lafham, Whetfeld, Wordsey, Navelton, Lilesey, and Bretenham. (Rokewode, p.\ \oldstylenums{132}.)} sed de villa nostra de Cokefeld nulla fuit facta mentio, nec cartam creditur inde habere; Semere et Grotene tenebit ad tempus vit\ae{} su\ae{}.
\end{otherlanguage}

\switchcolumn

After he had thus spoken, much money was offered to the abbot, but he could not be turned from his purpose either by prayer or present. At the last it was thus agreed between them: Adam abandoned the right which he had verbally claimed in the hundred, and the abbot confirmed to him all his other lands.\engnotenum{} No mention was made, however, of our township of Cokefield, and it is believed that he had no charter concerning it. Semere and Groton he was to hold for the term of his life.

\switchcolumn*

\begin{otherlanguage}{latin}
\blockhead{How the mill which Herbert the dean had built was overturned.}{4}{-.45cm}
Herbertus decanus\engnotetext{He received a grant of lands from abbot Ording, his cousing. (Rokewode, p.\ \oldstylenums{133}.)} levavit molendinum ad ventum super Hauberdun:\engnotetext{``Within the bounds of St.\ Edmundsbury, lies behind Southgate street, extending to the river Lark, contiguous to the parish of Rougham.'' (Rokewode, p.\ \oldstylenums{133}.)} quod cum audisset abbas, tanta ira excanduit, quod vix voluit comedere, vel aliquod verbum proferre. In crastino, post missam auditam, pr\ae{}cepit sacrist\ae{} ut sine dilatione faceret carpentarios suos illuc ire et omnia subvertere, et materiam lignorum in salvam custodiam reponere.

\end{otherlanguage}

\switchcolumn

Herbert the dean\engnotenum{} built a windmill on Haberdon.\engnotenum{} When the abbot heard this, he was so wroth that he would hardly eat, or speak a single word. On the morrow, after he had heard mass, he commanded the sacristan that without delay he should cause his carpenters to go thither and overturn everything, and place the wood with which it had been built in safe keeping.

\switchcolumn*

\begin{otherlanguage}{latin}
Audiens hoc decanus, venit dicens, se hoc de jure posse facere super liberum feudum suum, nec beneficium venti alicui homini debere denegari, et dixit se velle suum proprium bladum ibi molere, non alienum, ne forte putaretur hoc facere in vicinorum molendinorum detrimentum.
\end{otherlanguage}

\switchcolumn

And the dean heard this, and he came, saying that he was by law able to do this on his freehold land, and that the profit which may come from the wind ought to be denied to no man. He said also that he intended to grind his own corn there and not that of other men, that he might not be thought to have done this thing to the detriment of neighbouring mills.

\switchcolumn*

\begin{otherlanguage}{latin}
Et respondit abbas adhuc iratus: ``Gratias tibi reddo ac si ambos pedes meos amputasses; per os Dei, nunquam panem manducabo, donec fabrica illa subvertatur. Senex es, et scire debuisti, quod nec regi nec justitiario licet aliquid immutare vel constituere infra \emph{bannamleucam} sine abbate et conventu; et tu tale quid pr\ae{}sumsisti? Nec hoc sine detrimento meorum molendinorum est, sicut asseris, quia ad tuum molendinum burgenses concurrent, et bladum suum molerent pro beneplacito suo, nec [in] eos possem de jure advertere, quia liberi homines sunt. Nec etiam molendinum celerarii noviter levatum stare sustinerem, nisi quia levatum fuit antequam fui abbas. Recede,'' inquit, ``recede; antequam domum tuam veneris, audies quid fiet de molendino tuo.''
\end{otherlanguage}

\switchcolumn

And the abbot, still angry, replied, ``I thank you as much as if you had cut off both my feet; by the face of God, I will never eat bread until that building be overturned. You are an old man, and you ought to know that neither the king nor his justiciar may change anything or build anything within the jurisdiction of the monastery, without the leave of the abbot and the house. And who are you, that you are so very presumptuous? Nor is this without harm to my mills, as you pretend, for the burghers go to your mill, and grind their corn at their pleasure, while I cannot lawfully hinder them, since they are free men. And I would not allow even the mill of the cellarer, which has been newly built, to stand, had it not been that it was built before I was abbot. Depart,'' he went on, ``depart. Before you come to your house, you will hear what has come to pass in the matter of your mill.''

\switchcolumn*

\begin{otherlanguage}{latin}
Decanus autem timens a facie abbatis, consilio filii sui magistri Stephani, famulos sacrist\ae{} pr\ae{}veniens, molendinum illud elevatum a propriis famulis suis, sine omni mora, erui fecit; ita quod, venientibus servientibus sacrist\ae{}, nichil subvertendum invenerunt.
\end{otherlanguage}

\switchcolumn

Then the dean, fearing before the face of the abbot, took counsel with his son, master Stephen, and anticipated the servants of the sacristan, and caused the mill which had been built by his own servants to be destroyed without delay. And so it was, that when the men of the sacristan came there, they found nothing to overthrow.

\switchcolumn*

\begin{otherlanguage}{latin}
\blockhead[\textsc{a.d}.\ \oldstylenums{1194} \textsc{\& later}.]{How the right of the abbot to present to certain churches was disputed, and what befel in the matter.}{5}{-.6cm}
Quarumdam ecclesiarum advocationem calumniatus est abbas et obtinuit. Quasdam etiam calumniatas retinuit; ecclesiam de Westle, de Meringetorp, de Brethenham, de Wenelinga, de Pakeham, de Neutona, de Bradefelda in Norfolcia, medietatem ecclesi\ae{} de Bocsford, ecclesiam de Scaldewella, et ecclesiam de Endegate.\engnotetext{Rokewode (p.\ \oldstylenums{133}--\oldstylenums{4}) collects the evidence of the possession of these churches by St.\ Edmund.} Omnes ab aliis calumniatas retinuit, et tres portiones de ecclesia de Diccleburcha ad jus advocationis su\ae{} revocavit, et illarum portionum tenementa ad liberum feudum ecclesi\ae{} reduxit, salvo servitio quod inde debetur aul\ae{} de Tiveteshala.

\end{otherlanguage}

\switchcolumn

The right of the abbot to present to certain churches was disputed, and he gained the point. He also retained the right to present in certain other cases where that right had been disputed, in the cases of the churches of Westley, Meringthorpe, Bretenham, Weneling, Pakenham, Newton, and Bradfield in Norfolk, half the church of Boxford, the church of Scaldwell and the church of Endgate.\engnotenum{} All these he retained against the claim of others, and he recovered the right of presentation to three parts of the church of Dickleburgh. And in those portions he restored the holdings to the freehold of the church, saving the service due thence to the manor of Titshall.

\switchcolumn*

\begin{otherlanguage}{latin}
\blockhead[\textsc{a.d}.\ \oldstylenums{1188}.]{}{2}{-0.7cm}
Ecclesia vero de Bocsford vacante, cum summonita fuisset inde recognitio, venerunt quinque milites tentantes abbatem, et qu\ae{}rentes quid inde deberent jurare. Abbas autem noluit eis aliquid dare, vel promittere, sed dixit: ``Cum ad juramentum perventum fuerit, dicite rectum secundum conscientiam vestram.'' Ipsi vero indignantes recesserunt, et ei per juramentum suum advocationem illius ecclesi\ae{}, scilicet ultimam pr\ae{}sentationem,\footnote[\textdagger]{A special assize for trying cases of \emph{Nouvel Disseisin, Mort d'Ancestor}, and \emph{Darrein Presentment}, was erected by clause \oldstylenums{18} of Magna Charta. See Stubbs, \emph{Const.\ Hist}.\ i.\ \oldstylenums{617}.}\engnotetext{An inquest for the purpose of discovering the person who last presented to a benefice.} abstulerunt; quam tamen postea recuperavit, multis factis expensis, et datis decem marcis.

\end{otherlanguage}

\switchcolumn

But when the church of Boxford fell vacant, and an inquest was ordered in the matter, five knights came, tempting the abbot, and asking him what they should swear in that matter. But the abbot would not either give or promise them anything, and said, ``When the time comes to make oath, say that which is right according to your conscience.'' Then they went away in anger, and by their oath, in an inquest of darrein presentment,\engnotenum{} deprived him of the right of presenting to that church. This right he afterwards regained at great expense and by a payment of ten marks.

\switchcolumn*

\begin{otherlanguage}{latin}
Ecclesiam de Hungetona non vacantem sed calumniatam retinuit abbas, tempore Durandi de Hostesli, licet ipse monstraverit in testimonium juris sui cartam W.\ Norwicensis episcopi,\engnotetext{William de Turbeville, bishop of Norwich, \oldstylenums{1146} to \oldstylenums{1174}. He was a strong adherent of Becket.} qua continetur, quod Robertas de Valoniis socer ejus dederit illam ecclesiam \AE{}rnaldo Luvello.
\end{otherlanguage}

\switchcolumn

The abbot retained the church of Honington, which had not been vacant, but to which his claim had been disputed in the time of Daniel de Hostelli, though he produced in evidence of his right a charter of William, bishop of Norwich,\engnotenum{} wherein it was provided that Robert de Valoniis, his father-in-law, had given that church to Ernald Lovell.

\switchcolumn*

\begin{otherlanguage}{latin}
\blockhead[Before \textsc{a.d}.\ \oldstylenums{1191}.]{}{2}{-0.7cm}
Vacante medietate ecclesi\ae{} de Hopetuna, mota est controversia inde inter abbatem et Robertum de Ulmo, positoque die concordi\ae{}\footnote[\textdagger]{A ``dies concordi\ae{}'' was probably a ``love day.'' Comp.\ Wyclif, in \emph{Grete Sentens of Curs} (Works, Clar.\ Press, iii.\ \oldstylenums{322}); rich men's followers, he says, ``schal holde wrongis at love dayes, and bere doun trouþe and pore men in here riȝt, by colour of lordis knelynge in þe chapel.''} apud Hopetonam, post multas altercationes dixit abbas ad pr\ae{}dictum R., nescio quo impetu animi ductus: ``Tu jura in propria persona, quod hoc tuum jus est, et ego concedo quod tuum sit.'' Cumque miles ille renuisset jurare, delatum est juramentum per consen sum utriusque partis sexdecim legalibus de hundredo, qui juraverunt hoc esse jus abbatis. Gilbertus filius Radulfi et Robertus de Cokefeld, domini illius feudi, affuerunt et consenserunt. 

\end{otherlanguage}

\switchcolumn

When half the church of Hopeton was vacant, a dispute arose on this matter between the abbot and Robert de Ulmo. And when a day for settling the matter was appointed at Hopeton, after much disputing, the abbot said to the same Robert, being moved thereto by some sudden impulse, ``Swear in your own person that this right is yours, and I will grant that it is.'' And when the knight refused to swear, the matter was transferred with the assent of both parties to sixteen legal men of the hundred, who made oath that the right belonged to the abbot. Gilbert Fitz Ralph and Robert de Cokefield, the lords of that fief, were present, and gave their assent.

\switchcolumn*

\begin{otherlanguage}{latin}
\blockhead{How abbot Samson disputed with Jordon de Ros.}{4}{-.45cm}
Prosiliens ibi magister Jordanus de Ros, habens tam cavtam H.\ abbatis, quam pr\ae{}dicti R., et hinc inde ut uter eorum dirationaret ecclesiam, ipse personatum haberet, dixit se esse personam totius ecclesi\ae{} et clericum proximo mortuum fuisse vicarium ejus, reddendo ei annuam pensionem de illa medietate, et inde ostendit cartam Walchelini archidiaconi. 

\end{otherlanguage}

\switchcolumn

Then master Jordan de Ros, who had charters both from abbot Hugh and the said Robert, came forward, and addressing both parties, said that whichever of them proved his claim to the church, he ought to have the position of rector. He declared that he was rector of the whole church, and that the recently deceased clerk was his vicar, paying to him an annual income from that half, and in evidence he produced a charter of Walchelin the archdeacon.

\switchcolumn*

\begin{otherlanguage}{latin}
Abbas vero turbatus et iratus erga eum, nunquam eum in gratiam amiciti\ae{} recepit, donec ipse Jordanus in capitulo monachorum de Theford, instante abbate, reconsignavit in manus episcopi ibi pr\ae{}sentis illam medietatem pr\ae{}cise, sine omni conditione et spe recuperandi eam, coram multitudine clericorum. Quo facto, dixit abbas: ``Domine episcope, ego ex promisso teneor dare redditum alicui clerico vestro: et ego dabo hanc medietatem hujus ecclesi\ae{} cui ex vestris volueritis.'' Et episcopus petivit, ut amicabiliter redderetur eidem magistro Jordano, et sic ex pr\ae{}sentatione abbatis eam suscepit Jordanus.
\end{otherlanguage}

\switchcolumn

Then the abbot was perturbed and angered with him, nor did he ever receive him into his favour until the same Jordan in the chapter of the monks of Thetford, owing to the abbot's insistence, resigned into the hands of the bishop, who was present there, that half absolutely, without any conditions or hope of recovery, in the presence of a number of clerks. When this had been done, the abbot said, ``My lord bishop, I am bound on a promise to give the revenue to one of your clerks, and I will give the half of this church to whichsoever of your clerks you will.'' And the bishop asked that he would restore it in a friendly manner to the same Jordan, and thus Jordan received it by presentation from the abbot.

\switchcolumn*

\begin{otherlanguage}{latin}
Postea mota est controversia inter abbatem et eundem Jordanum de terra Herardi in Herlava, utrum esset liberum feudum ecclesi\ae{}, an non. Cumque inde summonita esset recognitio duodecim militum in curia regis facienda, facta est in curia abbatis apud Herlavam per licentiam Rannulfi de Glanvilla, et juraverunt recognitores se nunquam scivisse illam terram fuisse separatam ab ecclesia, sed tamen illam terram debere abbati tale servitium quale debet terra Eustachii, et qu\ae{}dam ali\ae{} terr\ae{} laicorum in eadem villa. 
\end{otherlanguage}

\switchcolumn

After this a dispute arose between the abbot and this same Jordan about the land of Herard in Harlow, as to whether or no it was a free fief of the church. And when an inquest of twelve knights was ordered on this matter, to be made in the king's court, by leave of the justiciar Ranulf Glanvill it was made in the court of the abbot at Harlow. And the jurors swore that they had never known that land divided from the church, but yet that this land owed to the abbot the same service as the land of Eustace and certain other land of laymen in the same township owed to him.

\switchcolumn*

\begin{otherlanguage}{latin}
Tandem convenit inter eos ita: magister Jordanus in plena curia recognovit illam terram esse laicum feudum, et se nichil inde vindicare, nisi per gratiam abbatis; et illam terram tenebit omnibus diebus vit\ae{} su\ae{}, reddendo inde annuatim abbati xij.\ denarios pro omnibus servitiis.
\end{otherlanguage}

\switchcolumn

At last an agreement was reached between them on the following terms: master Jordan in full court acknowledged that this land was a lay fief, and that he had no rights in it except by the grace of the abbot. And he was to hold that land all the days of his life, paying an annual rent to the abbot for it of twelve pence in commutation of all services.

\switchcolumn*

\begin{otherlanguage}{latin}
\blockhead{How the author made a list of the abbot's churches as a gift to the abbot, and the names of those churches.}{4}{-.65cm}
Cum juxta consuetudinem Anglorum multi multa darent munera abbati, ut domino, die Circumcisionis Dominic\ae{}, cogitavi ego Jocelinus quid dare possem Et incepi in scriptum redigere omnes ecclesias qu\ae{} sunt in donatione abbatis, tam de nostris maneriis  quam de suis, et rationabilia pretia earum, sicut possent poni ad firmas, tempore quo bladus mediocriter venditur. Et intrante anno subsequente, dedi abbati schedulam illam pro munere ejus, quam valde gratanter accepit.

\end{otherlanguage}

\switchcolumn

Now when, in the manner of the English, many men gave many presents to the abbot, as to their lord, on the day of the Circumcision of the Lord, I, Jocelin, thought what I could give to him. And I began to write out a list of all the churches which are in the gift of the abbot, both on our manors and on his, and the right values of the same, according as they might be placed at firm at a time when the price of corn was moderate. And when the beginning of the next year came, I gave to the abbot that schedule as my gift to him, and he received it with great pleasure.

\switchcolumn*

\begin{otherlanguage}{latin}
Ego vero, quia placui tunc temporis in conspectu ejus, cogitavi in corde meo quod dicerem ei, ut aliquam ecclesiam daret conventui et assignaret in usum hospitalitatis, sicut desideravit quando pauper monachus claustralis fuit, et sicut ipse voluit ante electionem suam, ut fratres jurarent, ut super quemcumque sors abbati\ae{} caderet, hoc faceret. Sed dum hoc cogitavi, occurrit mihi memori\ae{}, quod quidam alius prius dixerat ei idem verbum, et audieram abbatem respondentem, se non posse demembrare baroniam, scilicet nec debere minuere libertatem et dignitatem, quam H.\ abbas et ceteri pr\ae{}decessores sui habuerunt de ecclesiis donandis, qui nullam vel vix contulerunt conventui; et idcirco tacui.
\end{otherlanguage}

\switchcolumn

Then I, because I was then pleasing in his sight, ``thought in my heart'' that I might say to him that he should give some church to the monastery and should assign it to the maintenance of hospitality, as he had himself desired when he was a poor cloistered monk, and as before his election he had wished the brothers to swear, that he on whom the lot of the abbacy should fall should do this. But while I so thought I suddenly called to mind that some other had already said to him this word, and that I had heard the abbot answer that he could not dismember his barony, that is, that he ought not to reduce the liberty and dignity which abbot Hugh and others, his predecessors, had possessed in the matter of giving churches, which brought little or no gain to the monastery. And for this cause I was silent.

\switchcolumn*

\begin{otherlanguage}{latin}
Scriptum tale fuit:
\end{otherlanguage}

\switchcolumn

Now the writing was as follows:---

\switchcolumn*

\begin{otherlanguage}{latin}
``H\ae{} sunt ecclesi\ae{} de maneriis et sochagiis abbatis. Ecclesia de Meleford valet xl.\ libras, Geventona x.\ marcas, Saxham xij.\ marcas, Hargrava v.\ marcas, Brethenham v.\ marcas, Bocsford centum solidos, major Fornbam c.\ solidos, Stowa c.\ solidos, Hunegetona v.\ marcas, Helmeswell tres marcas, Cottuna xij.\ marcas, Brocford v.\ marcas, Palegrava x.\ marcas, major Horningesherd v.\ marcas, Cunegestuna iiij$^\text{or}$ marcas, Herlava xix.\ marcas, Stapelforda tres marcas, Tiveteshala c.\ solidos, Wirlingword cum Bedingfeld xx.\ marcas, Saham vi.\ marcas, medietas ecclesi\ae{} de Wortham c.\ solidos, Rungetona xx.\ marcas, Torp vi.\ marcas, Wlpet pr\ae{}ter pensionem c.\ solidos, Ressebroc v.\ marcas, medietas ecclesi\ae{} de Hopetona lx.\ solidos, Richingale vi.\ marcas, tres partes ecclesi\ae{} de Dicleburch, qu\ae{}libet pars valet xxx.\ solidos et plus, medietas ecclesi\ae{} de Gislingham quatuor marcas, Ichelingham vj.\ marcas. De ecclesia de Mildenhala, qu\ae{} valet xl.\ marcas, et de medietate ecclesi\ae{} de Wederdena quid dicam? Wenelinge c.\ solidos, ecclesia de Len x.\ marcas, ecclesia de Scaldewelle v.\ marcas, de Werketona---.
\end{otherlanguage}

\switchcolumn

``These are the churches on the manors and socages of the abbot: The church of Melford, valued at forty pounds; Chevington, ten marks; Saxham, twelve marks; Hargrave, five marks; Bretenham, five marks; Boxford, a hundred shillings; the greater Fornham, a hundred shillings; Stowe, a hundred shillings; Honington, five marks; Elmswell, three marks; Cotton, twelve marks; Brocford, five marks; Palgrave, ten marks; the greater Horningsherth, five marks; Kingston, four marks; Harlow, nineteen marks; Stapleford, three marks; Tivetshall, a hundred shillings; Worlingworth and Bedingfeld, twenty marks; Saham, six marks; half the church of Wortham, a hundred shillings; Rungton, twenty marks; Thorp, six marks; Woolpit, with the pension excluded, a hundred shillings; Rushbrook, five marks; half the church of Hopeton, sixty shillings; Richinghall, six marks; three parts of the church of Dickleburgh, at the rate of over thirty shillings for each part; half the church of Gislingham, four marks; Icklingham, six marks; the church of Mildenhall, which is worth forty marks, and half the church of Wederden, some unknown quantity; Weneling, a hundred shillings; the church of Len, ten marks; that of Scaldwell, five marks; Warkton . . .

\switchcolumn*

\begin{otherlanguage}{latin}
``H\ae{} sunt ecclesi\ae{} de maneriis conventus; Mildenhal, Bertona, et Horningesherth xxv.\ marcas pr\ae{}ter pensionem, Rutham xv.\ marcas pr\ae{}ter pensionem, Bradefeld v.\ marcas, Pakeham xxx.\ marcas, Suthreia c.\ solidos, Riseby xx.\ marcas, Neutona iiij$^\text{or}$ marcas, Wepsted xiiij.\ marcas, Fornham Sancte Genovefe xv.\ marcas, Herningeswell ix.\ marcas, Fornham Sancti Martini iij.\ marcas, Ingham x.\ marcas, Lacford c.\ solidos, Alvedena x.\ marcas, Kokefeld xx.\ marcas, Semere xij.\ marcas, Grotone v.\ marcas, medietas ecclesi\ae{} de Frisingfeld xiiij.\ marcas, Beccles xx.\ marcas, Broc xv.\ marcas, Hildercle x.\ marcas, Werketona x.\ marcas, Scaldewell v.\ marcas, Westle v.\ marcas, ecclesia in Norwico duas marcas pr\ae{}ter pensionem alleciorum, et du\ae{} ecclesi\ae{} in Colecestra iii.\ marcas pr\ae{}ter pensionem iiij.\ solidorum, Chelesword c.\ solidos, Meringetorp iiij.\ marcas, medietas ecclesi\ae{} de Bradefeld in Norfolchia tres marcas: staffacres, et foracres,\engnotetext{``These staffacres, it would seem, were certain payments or fees due to the abbot's staff or crozier.'' (Rokewode, p.\ \oldstylenums{134}.) Arnold (\emph{Mem}.\ I., \oldstylenums{268}, note) suggests that the foracres were similar payments for the right of attending the market (\emph{forum}).} et terti\ae{} partes decimarum dominiorum Wrabenesse, vj.\ marcas.''
\end{otherlanguage}

\switchcolumn

``These are the churches on the manors of the monastery:---Mildenhall, Barton, and Horningsherth, valued at twenty-five marks, exclusive of the pension; Rougham, fifteen marks, exclusive of the pension; Bradfield, five marks; Pakenham, thirty marks; Southrey, a hundred shillings; Risby, twenty marks; Newton, four marks; Whepstead, fourteen marks; Fornham St.\ Genevieve, fifteen marks; Herningswell, nine marks; Fornham St.\ Martin, three marks; Ingham, ten marks; Lackford, a hundred shillings; Alpheton, ten marks; Cokefield, twenty marks; Semere, twelve marks; Groton, five marks; half the church of Fresingfeld, fourteen marks; Beccles, twenty marks; Broc, fifteen marks; Heldcercle, ten marks; Warkton, ten marks; Scaldwell, five marks; Westley, five marks; a church in Norwich, two marks, excluding the pension of herrings; and two churches in Colchester, three marks, excluding the pension of four shillings; Chelsworth, a hundred shillings; Meringthorp, four marks; half the church of Bradfield in Norfolk, three marks; staffacres, and foracres,\engnotenum{} and a third of the tithes of the demesnes of Wrabnesse, six marks.''

\switchcolumn*

\begin{otherlanguage}{latin}
\blockhead[\textsc{a.d}.\ \oldstylenums{1187}.]{How the abbot freed his church from contribution to the fine inflicted on Norfolk and Suffolk.}{5}{-.6cm}
Duo comitatus Norfolchia et Suthfolchia positi fuerunt in misericordia regis a justiciariis errantibus propter quoddam forisfactum, et posit\ae{} fuerunt l.\ marc\ae{} super Norfolchiam, et xxx.\ super Sutfolciam. Et cum qu\ae{}dam portio de illa communi misericordia poneretur super terras Sancti \AE{}dmundi et acriter  exigeretur, abbas sine omni mora adiit dominum regem, et invenimus eum apud Clarendonam; ostensaque ei carta regis \AE{}dwardi, qu\ae{} liberas facit terras  Sancti \AE{}dmundi de omnibus geldis et scottis, pr\ae{}cepit rex per literas suas, ut sex milites de comitatu de Norforchia et sex de Sutfolchia summonerentur ad recognoscendum coram baronibus scaccarii, utrum dominia Sancti \AE{}dmundi deberent esse quieta de communi misericordia; 

\end{otherlanguage}

\switchcolumn

The two counties of Norfolk and Suffolk were ad judged by the itinerant justices to be fined at the king's discretion for some fault, and fifty marks were imposed upon Norfolk and thirty upon Suffolk. And when some portion of that common fine was placed upon the lands of St.\ Edmund and sternly exacted, the abbot without delay went to the lord king, and we found him at Clarendon. And when a charter of king Edward, which freed the lands of St.\ Edmund from all gelds and scot, had been shown to him, the king ordered by his letters that six knights of the county of Norfolk and six of that of Suffolk should be summoned to make inquest in the presence of the barons of the exchequer, as to whether the demesnes of St.\ Edmund ought to be quit of that general fine.

\switchcolumn*

\begin{otherlanguage}{latin}
et electi sunt tantum sex milites, ut ita parceretur laboribus et expensis, et ideo quia habuerunt terras in utroque comitatu, scilicet Hubertus de Briseword, W.\ filius Hervei, et Willielmus de Franchevilla, et tres alii, qui Londonias iverunt nobiscum, et ex parte duorum comitatuum libertatem ecelesi\ae{} nostr\ae{} recognoverunt. Justiciarii autem assidentes verumdictum illorum inrollaverunt.
\end{otherlanguage}

\switchcolumn

Then six knights were chosen, that so trouble and expense might be saved, and these because they held lands in both counties, namely, Hubert de Brisewood, W.\ FitzHervey, and William de Francheville, and three others. They went with us to London, and on behalf of the two counties adjudged this liberty to belong to our church. Then the justices who were present enrolled their verdict.

\switchcolumn*

\begin{otherlanguage}{latin}
\blockhead[\textsc{a.d}.\ \oldstylenums{1196}.]{How the abbot disputed with his knights.}{4}{-.45cm}
Abbas Samson iniit certamen cum militibus\engnotetext{Rokewode (p.\ \oldstylenums{134}--\oldstylenums{5}) gives a list of the knights in \oldstylenums{3} \emph{Ric}.\ I. As to the struggle, compare the text, p.\ \oldstylenums{43}.} suis ipse contra omnes, et omnes contra eum,---(proposuit  eis quod deberent ei facere integre servitium quinquaginta militum in scutagiis, et in auxiliis, et in consimilibus, quia, ut aiebant, feudos tot militum tenebant,)---quare decem ex illis quinquaginta militibus essent sine servitio, vel qua ratione vel cujus auctoritate illi quadraginta reciperent servitium decem militum. Responderunt omnes una voce talem fuisse consuetudinem, ut decem ex illis semper adjuvarent quadraginta, nec se velle inde, nec debere inde respondere, nec in placitum intrare.

\end{otherlanguage}

\switchcolumn

Abbot Samson began a struggle with his knights,\engnotenum{} he against all and all against him. He declared that they ought to make to him the full service of fifty knights in scutages, aids, and the like, since, as they admitted, they held the fees of that number of knights. He disputed as to why ten of those fifty knights should be free from service, and as to the reason or authority those forty received the service of ten knights. They all with one voice answered that such was the custom, that ten of them should ever help the forty, and they said that on this matter they would neither make answer nor stand trial, as they ought not to do so.

\switchcolumn*

\begin{otherlanguage}{latin}
Cum ergo fuissent summoniti inde responsuri in curia regis, quidam exoniaverunt se ex industria, quidam apparuerunt in dolo, dicentes, se non respondere sine paribus suis. Alia vice se pr\ae{}sentaverunt, qui prius se absentaverunt, dicentes similiter, se non debere respondere sine paribus suis, qui in eadem querela fuerunt.
\end{otherlanguage}

\switchcolumn

But when they were summoned to make answer on this matter in the court of the king, some craftily excused themselves and others craftily appeared, the latter saying that they ought not to answer without their peers. On another occasion, those who had at first absented themselves appeared, and said the same thing, that they ought not to make answer without their peers, who were with them in the dispute.

\switchcolumn*

\begin{otherlanguage}{latin}
Cumque sic s\ae{}pius illusissent abbati, et in magnis et in gravibus expensis vexassent, conquestus est inde abbas H.\ archiepiscopo tunc justiciario; qui respondit in generali concione quemlibet militem pro se ipso debere loqui et pro suo proprio tenemento. Et dixit palam abbatem bene scientem et bene potentem esse ad dirationandum jus ecclesi\ae{} su\ae{} contra omnes et singulos.
\end{otherlanguage}

\switchcolumn

And when they had thus many times mocked the abbot, and had vexed him with great and heavy expenses, the abbot complained on this matter to archbishop Hubert, who was then justiciar. And he answered in the full council that each knight ought to reply for himself and for his own holding. And he said openly that he knew well, and was well able to secure, the right of his church against all and every one of them.

\switchcolumn*

\begin{otherlanguage}{latin}
Comes ergo R.\ Bigot primus sponte confessus est in jure se debere abbati domino suo servitium trium militum integre, et in releviis et in scutagiis et in auxiliis, sed de warda facienda ad castellum Norwici tacuit.
\end{otherlanguage}

\switchcolumn

Then earl Roger Bigod was the first to acknowledge freely that in law he owed his lord, the abbot, the full service of three knights, and both in reliefs, and scutages, and aids. But as to the wardenship of the castle of Norwich, he was silent.

\switchcolumn*

\begin{otherlanguage}{latin}
Venerunt postea duo ex militibus, postea tres, postea plures, postea s\ae{}pe omnes, et ad exemplum comitis idem servitium recognoverunt; et quia non sufficiebat recognitio inde facta in curia Sancti \AE{}dmundi, secum ducebat omnes abbas Lundonias ad suos sumptus, et uxores et mulieres, qui erant terrarum h\ae{}redes, ut recognitionem facerent in curia regis, et singuli singulos chirographos acceperunt. 
\end{otherlanguage}

\switchcolumn

After that there came two of the knights, and then three, and then more, and eventually almost all, and, following the example of the earl, admitted the same service. And as a recognition made on this matter in the court of St.\ Edmund was not enough, the abbot took with him to London at his own expense the wives and women, who were heiresses of lands, that they might make recognition in the king's court, and each received a separate charter.

\switchcolumn*

\begin{otherlanguage}{latin}
Albericus de Ver, et Willelmus de Hastinga, et duo alii fuerunt in servitio regis ultra mare, quando h\ae{}c fiebant, et ideo loquela de eis differri debuit. Albericus de Ver ultimus erat qui abbati resistebat; abbas vero averia ejus cepit et vendidit, unde oportuit eum venire in curiam et respondere sicut pares sui. Inito ergo consilio, recognovit tandem Sancto \AE{}dmundo et abbati jus suum.
\end{otherlanguage}

\switchcolumn

Alberic de Vere and William de Hastings, and two others were in the service of the king, across the sea, when these things were done, and therefore this dispute had to await final settlement. Alberic de Vere was the last who resisted the abbot; but the abbot took and sold his cattle, by which means he was driven to come to the court and make answer as his peers had done. And so he took counsel, and at last recognised the right as belonging to St.\ Edmund and the abbot.

\switchcolumn*

\begin{otherlanguage}{latin}
Superatis ergo omnibus militibus, ex tali victoria tale lucrum poterit abbati,\footnote[\textdagger]{Desiderari videtur ``accrescere.'' Roke.} nisi abbas voluerit aliquibus parcere; quotiens xx.\ solidi ponentur super scutum, remanebunt abbati xij.\ libr\ae{}, et si plus vel minus ponatur, plus vel minus ei remanebit secundum debitam portionem. Item solebat abbas et antecessores sui semper in fine xx.\ ebdomadarum dare vij.\ solidos ad wardam castelli de Norwico\footnote[\textdagger]{In \oldstylenums{1173} the careless guard kept at Norwich had been the cause of the town's being taken and sacked by the Flemings (Will.\ de Neub.\ ii.\ \oldstylenums{30}); it was therefore to be expected that the defences of the castle would in after times be looked to more closely.} de sua bursa pro defectu trium militum, quos comes R.\ Bigot tenet de Sancto \AE{}dmundo, et solebant singuli milites de quatuor constabiliis dare xxviij.\ denarios, quando intrabant ad wardas faciendas, et unum denarium marescaldo, qui illos denarios colligebat, et ideo xxviij.\ denarios et non amplius dabant, quia decem milites de quinta constabilia solebant adjuvare ceteros quadraginta; ita quod, ubi debebant dare tres solidos integre, dederunt tantummodo xxix.\ denarios, et qui debebat intrare ad wardam faciendam in fine iiij$^\text{or}$ mensium, intravit in fine xx.\ ebdomadarum. Modo autem dant singuli milites plene tres solidos et remanet abbati superexcrescentia qu\ae{} excrescit ultra xxix.\ denarios, unde poterit se adquietare de pr\ae{}nominatis vii.\ solidis.
\end{otherlanguage}

\switchcolumn

Therefore when all the knights had been overcome, the abbot might have made great gain of money from so notable a victory, had he not desired to spare some of them. For whenever twenty shillings are charged on a fee, there will remain for the abbot twelve pounds, and if more or less be charged, then more or less will remain according to the due proportion. Further, the abbot and his predecessors were wont at the end of twenty weeks to give seven shillings for the wardship of the castle of Norwich from his own money, to supply the defect of the three knights; fees which earl Roger Bigod holds of St.\ Edmund. And all the knights of the four constabularies used to pay twenty-eight pence when they began to make wards, and one penny to the marshal who collected those pennies, and they gave twenty-eight pence and no more, because the ten knights of the fifth constabulary were wont to assist the other forty. Therefore, when they ought to have given three shillings in full, they gave only nineteen pence, and one who should have entered upon the duty of making ward at the end of four months, entered at the end of twenty weeks. Now, however, all the knights paid three shillings in full, and the amount by which it has been increased beyond twenty-nine pence remains to the abbot, wherefore he can repay himself the said seven shillings.

\switchcolumn*

\begin{otherlanguage}{latin}
Ecce patet quam vim obtinuerunt comminationes abbatis, quas fecit die prima, quando recepit homagium de militibus suis, sicut pr\ae{}scriptum est, quando singuli milites promiserunt ei xx.\ solidos, et statim se retraxerunt, nolentes dare ei in summa nisi xl.\ libras, dicentes quod decem milites deberent adjuvare ceteros quadraginta in auxiliis et wardis faciendis, et in omnibus consimilibus.
\end{otherlanguage}

\switchcolumn

Thus was seen how the threats of the abbot, which he uttered on his first day when he received the homage of his knights, had their fulfilment, when, as has been related, all the knights promised him twenty shillings and at once went back on their promise, refusing to give him a total sum of more than forty pounds in making aids and performing wards and in all like things.

\switchcolumn*

\begin{otherlanguage}{latin}
Est autem qu\ae{}dam terra in Tivetteshale de feudo abbatis qu\ae{} reddere solet vigilibus de castello Norwici \emph{Waite-fe}, id est, xx.\ solidos per annum, scilicet quinque solidos in quolibet jejunio quatuor temporum. Antiqua est h\ae{}c consuetudo, quam abbas libenter vellet mutare, si posset; sed impotentiam suam considerans in hac parte adhuc tacet et dissimulat.
\end{otherlanguage}

\switchcolumn

In Tivetshall there is some land on the fee of the abbot which used to pay to the watchmen of the castle of Norwich waite-fee, which is to say, twenty shillings a year, at the rate of five shillings at each of the four fasting times. This is an ancient custom, which the abbot was very anxious to change if he could. But as he saw his weakness in this matter, he still is silent and conceals his wish.

\switchcolumn*

\begin{otherlanguage}{latin}
\blockhead{Concerning Henry of Essex.}{2}{-.75cm}
Ad beati regis et martyris memoriam diffusius dilatandam, pr\ae{}scriptis non incontinue, ut credimus, istud connectimus;\engnotetext{This passage, as would appear from the terms of the first paragraph, is an interpolation by some other writer than Jocelin. Gervase (i., \oldstylenums{165}) mentions the conduct of Henry of Essex on the Welsh expedition, and the disastrous result on the English army. He adds, ``Owing to this accident, Henry, who had been the noblest among the princes of England, suffered everlasting shame and disinheritance.'' Diceto (I., \oldstylenums{310}) mentions the duel. (Cp.\ \emph{Mem}.\ I., \oldstylenums{272}, note.)} non quod ego tantillus et nullius fere momenti istud memoriali titulo commendaverim,\footnote[\ddag]{``Sed'' desideratur. Ro.} quia dominus Jocelinus, elemosinarius noster, vir religionis eximi\ae{}, potens in sermone et opere, ad potestatis preces imperiosas, sic tandem exorsus est; qu\ae{} mea reputo, quia, juxta pr\ae{}ceptum Senec\ae{}, quicquid ab aliquo bene dictum est, mihi inpr\ae{}sumptuose ascribo.\footnote[\textdagger]{The cowardice of Henry of Essex, the hereditary standard bearer, who, in \oldstylenums{1157}, during an expedition into Flintshire, when the Welsh made a sudden attack, dropped the standard, and so brought the king and army into great peril, made a great sensation. Gervase (i.\ \oldstylenums{165}, Rolls ed.) briefly summarises the whole story; Diceto (\emph{Ymag.\ Hist}.\ i.\ \oldstylenums{310}, Rolls ed.) records the duel fought between Henry and Robert de Montfort in \oldstylenums{1163}, and its result. Carlyle, with even more than his usual force and beauty of language, has made use of the monk's narrative in ``Past and Present.'' Who this monk was is not known. The compiler in Bod.\ \oldstylenums{240}, p.\ \oldstylenums{654}, says it was Jocelin himself; but that seems inconsistent with the tenor of the opening sentence.}\footnote[\textdagger]{Many things resembling this sentiment occur in the \oldstylenums{109}th Epistle of Seneca; but probably the passage meant is somewhere else in his works.}

\end{otherlanguage}

\switchcolumn

In order to spread far and wide the memory of the blessed king and martyr, we have added what follows to our writings, not improperly we hope.\engnotenum{} It is not that I, who am unimportant and almost of no account at all, should put this out with an historical title, but that master Jocelin, our almoner, a man of renowned piety, one mighty in word and deed, at the pressing request of the powers that be, at last thus began it in this way; and I regard the work as mine, since, according to the precept of Seneca, I may without presumption ascribe to myself whatever has been well said by another.

\switchcolumn*

\begin{otherlanguage}{latin}
Cum venisset abbas apud Radingas, et nos cum eo sicut decuit, suscepti sumus a monachis ejusdem loci; inter quos Henricus de Esexia devotus occurrit, qui, nacta loquendi opportunitate, abbati et omnibus assidentibus narravit qualiter victus fuit in duello, et qualiter Sanctus \AE{}dmundus et ob quam causam confudit eum in ipsa hora pugnandi. Ego vero narrationem ejus in scriptum redegi, domino abbate pr\ae{}cipiente, et scripsi in h\ae{}c verba:
\end{otherlanguage}

\switchcolumn

When the abbot was come to Reading, and we with him, we were rightly received by the monks of that place. And among them was Henry of Essex as a professed monk, who, when he had a chance to speak to the abbot and to those who were present, told us how he had been conquered in a trial by battle, and how and why St.\ Edmund confounded him in the very hour of conflict. But I wrote down his tale by command of the lord abbot, and I wrote it also in these words.

\switchcolumn*

\begin{otherlanguage}{latin}
Quum non potest malum vitari nisi cognitum, actus et excessus Henrici de Estsexia memoriali scripto tradere dignum ducimus, ad cautelam quidem, non ad usum. Utilis et indempnis solet esse castigatio, quam persuadent exemplaria.
\end{otherlanguage}

\switchcolumn

Inasmuch as it is impossible to avoid unknown evil, we have thought it well to commit to writing the acts and crimes of Henry of Essex, that they may be a warning, and not an example. Stories often convey an useful and salutary warning.

\switchcolumn*

\begin{otherlanguage}{latin}
Pr\ae{}dictus itaque Henricus dum floreret in prosperis, inter primates regni vir magni nominis habebatur, genere clarus, armis conspicuus, regis signifer, verendus omnibus privilegio potestatis. Ceteri comprovinciales ecclesiam beati \AE{}dmundi regis et martyris in rebus et redditibus ampliabant; ille vero non solum clausis oculis hoe pr\ae{}teribat, verum etiam vi et injuriis, et per injuriam, annuum redditum quinque solidorum abstraxit, et in proprios usus convertit.
\end{otherlanguage}

\switchcolumn

he said Henry, then, while he enjoyed great prosperity, had the reputation of a great man among the nobles of the realm, and he was renowned by birth, noted for his deeds of arms, the standard-bearer of the king, and feared by all men owing to his might. And when others who lived near him enriched the church of the blessed king and martyr Edmund with goods and rents, he on the contrary not only shut his eyes to this fact, but further violently, and wrongfully, and by injuries took away the annual rent of five shillings, and converted it to his own use.

\switchcolumn*

\begin{otherlanguage}{latin}
Processu vero temporis, cum in curia Sancti \AE{}dmundi ageretur causa de raptu cujusdam virginis, accessit idem H.\ protestans et asserens, loquelam illam in curia sua debere tractari ratione nativitatis ejusdem puell\ae{}, qu\ae{} in dominio suo de Lailand nata fuerat. Cujus rationis pr\ae{}textu, curiam Sancti \AE{}dmundi in itineribus et innumerabilibus expensis longo temporis tractu vexare pr\ae{}sumpsit.
\end{otherlanguage}

\switchcolumn

In the course of time, moreover, when a case arose in the court of St.\ Edmund concerning a wrong done to a certain maiden, the same Henry came thither, and protested and declared that the trial ought to be held in his court because the place where the said maiden was born was within his lordship of Lailand. With the excuse of this affair, he dared to trouble the court of St.\ Edmund for a long while with journeyings and countless charges.

\switchcolumn*

\begin{otherlanguage}{latin}
\blockhead[\textsc{a.d}.\ \oldstylenums{1157}.]{}{2}{-0.7cm}
In his interim et consimilibus arridens ad votum, prosperitas perpetui subintulit causam doloris, et, sub fantasia jocundi principii, tristes in eum exitus moliebatur; ex usu etenim est ei arridere ut s\ae{}viat, blandiri ut fallat, extollere ut deprimat. Nec mora; insurrexit in eum Robertus de Monteforti, ipsius consanguineus, nec genere nec viribus impar, in conspectu principum terr\ae{} damnans et accusans eum de proditione regis. Asseruit nempe  eum in expeditione belli apud Waliam in difficili transitu de Coleshelle vexillum domini regis fraudulenter abjecisse, et mortem ejus sublimi voce proclamasse, et in pr\ae{}sidium ejus venientes, in fugam convertisse. In rei veritate, pr\ae{}dictus Henricus de Esexia inclitum regem Henricum secundum, Walensium fraudibus interceptum, diem clausisse credidit extremum; quod revera factum fuisset, nisi Rogerus comes Clarensis,\engnotetext{In the Latin there is here a play upon words: ``comes Clarensis, clarus,'' etc., which cannot be rendered in a translation.} clarus genere et militari clarior exercitio, cum suis Clarensibus maturius occurrisset, et domini regis vexillum elevasset, ad corroborationem et animationem totius exercitus. 

\end{otherlanguage}

\switchcolumn

But fortune, which had assisted his wishes in these and other like matters, brought upon him a cause for lasting grief, and after mocking him with a happy beginning, planned a sad conclusion for him; for it is the custom of fortune to smile, that she may rage; to flatter, that she may deceive; and to raise up only that she may cast down. For presently there rose against him Robert de Montfort, his relative, and a man not unequal to him in birth and power, and slandered him in the presence of the princes of the land, accusing him of treason to the king. For he asserted that Henry, in the course of the Welsh war, in the difficult pass of Coleshill, had treacherously cast down the standard of the lord king, and proclaimed his death in a loud voice; and that he had induced those who were coming to the help of the king to turn in flight. As a matter of fact, the said Henry of Essex believed that the renowned king Henry the Second, who had been caught in an ambush by the Welsh, had been slain, and this would have been the truth, had not Roger, earl of Clare,\engnotenum{} a man renowned in birth and more renowned for his deeds of arms, hastened up quickly with his men of Clare, and raised the standard of the lord king, which revived the strength and courage of the whole army.

\switchcolumn*

\begin{otherlanguage}{latin}
\blockhead[\textsc{a.d}.\ \oldstylenums{1163}.]{}{2}{-0.35cm}
Henrico quidem resistente pr\ae{}dicto Roberto in concione, et objecta penitus inficiante, evoluto brevi temporis spatio, ad corporale duellum perventum est. Convenerunt autem apud Radingas pugnaturi in insula\footnote[\textdagger]{The fight on the Thames eyot, ``now a verdant meadow,'' is still traditionally remembered at Reading; see Murray's Handbook for Berks., p.\ \oldstylenums{35}.} quadam satis abbati\ae{} vicina; convenit et gentium multitudo, visura quem finem res sortiretur. Et factum est, cum Robertus duris et crebris ictibus viriliter intonasset, et audax principium fructum victori\ae{} promisisset, Henricus parumper deficiens circumquaque respexit, et ecce in confinio terr\ae{} et fluminis vidit gloriosum regem et martyrem \AE{}dmundum armatum et quasi in aere volitantem, et cum quadam vultus austeritate versus eum crebro capitis motu minas iracundi\ae{} et indignationis plenas pr\ae{}tendentem; vidit et alium cum eo militem, Gilbertum de Cerivilla, non solum quantum ad apparentiam gradu dignitatis inferiorem, sed et ab humeris supra statura minorem, oculos quasi indignantes et iracundos in eum convertere. Hic ad pr\ae{}ceptum ipsius Henrici, vinculis et tormentis afflictus, diem clausit extremum, intrusus occasione et accusatione uxoris Henrici, qu\ae{} propriam nequitiam in innocentem deflectens, dicebat se petitiones precarias Gilberti de illicito amore non posse sustinere. 

\end{otherlanguage}

\switchcolumn

Then Henry resisted the said Robert in the council, and utterly denied the charge, so that after a little while the matter came to a trial by battle. Then when they met at Reading to fight on an island somewhat near the abbey, there gathered there also a multitude of persons, to see how the affair would end. And it came to pass, that when Robert manfully made his armour ring again with hard and frequent blows, and his bold beginning promised the fruit of victory, Henry's strength began to fail him a little. And as he looked round about, behold! on the edge of land and water, he saw the glorious king and martyr Edmund, armed and as it were flying in the air, and looking towards him with an angry countenance, often shaking his head in a threatening manner, and showing himself full of wrath. And Henry also saw with the saint another knight, Gilbert de Cereville, who appeared not only less than the saint in point of dignity, but also head and shoulders shorter; and he looked on him with accusing and angry glances. This Gilbert, by order of the said Henry, afflicted with bonds and tortures, had died, as the result of an accusation brought against him by the wife of Henry, who cast the penalty for her own ill-doing on an innocent man, and said that she could not endure the evil suggestions of the said Gilbert.

\switchcolumn*

\begin{otherlanguage}{latin}
Hos itaque, tam sollicitus quam timidus, intuens, Henricus antiquum scelus novum ferre pudorem recordatur. Et jam totus desperans, et rationem in impetum convertens, impugnantis non defendentis, assumpsit officium. Qui dum fortiter percussit, fortius percussus est; et dum viriliter impugnabat, virilius inpugnabatur. Quid multa? victus occubuit.
\end{otherlanguage}

\switchcolumn

When he saw these sights, then, Henry grew alarmed and fearful, and called to mind that an old crime brings new shame. And now, giving up all hope, and abandoning skilful fighting for a blind rush, he took the part of one who attacks rather than that of one who defends himself. And when he gave hard blows, he received harder; and while he fought manfully, he was more manfully resisted. In a word, he fell conquered.

\switchcolumn*

\begin{otherlanguage}{latin}
Cumque mortuus crederetur, ad magnam petitionem magnatum Angli\ae{}, ejusdem Henrici consanguineorum, concessum est monachis ejusdem loci, ut darent ejus corpus sepultur\ae{}. Postea tamen convaluit, et, resumpto sanitatis beneficio, sub regulari habitu, superioris \ae{}vi labem detersit, et, longam dissolut\ae{} \ae{}tatis ebdomadam uno saltem sabbato curans venustare studia virtutum in frugem felicitatis excoluit.
\end{otherlanguage}

\switchcolumn

And as he was thought to be dead, in accordance with the earnest request of the magnates of England, the relatives of the said Henry, the monks of that place were allowed to give burial to his corpse. But he afterwards revived, and when he had regained the blessing of health, under the regular habit, he wiped out the stain of his former life, and taking care to purify the long week of his dissolute past with at least one sabbath, he cultivated the study of the virtues, to bring forth the fruit of happiness.


\switchcolumn*

\clearpage

\begin{otherlanguage}{latin}
\blockhead{How the abbot deceived the bishop of Ely for the good of his church.}{4}{-.1cm}
Galfridus Ridellus episcopus Eliensis petiit ab abbate materiem lignorum ad qu\ae{}dam magna \ae{}dificia  facienda apud Glemesford; quod et abbas concessit, sed invitus; non ausus tunc eum offendere. Abbate moram apud Meleford faciente, venit quidam clericus episcopi, rogans ex parte domini sui ut ligna promissa possent capi apud \AE{}lmeswell; et erravit in verbo, dicens \AE{}lmeswell, ubi dicere deberet \AE{}lmessethe, quod est nomen cujusdam nemoris de Meleford. Et mirabatur abbas de mandato, quia talia ligna non potuerunt inveniri apud \AE{}lmeswell. 


\end{otherlanguage}

\switchcolumn

Geoffrey Ridel, bishop of Ely, sought from the abbot a supply of wood for making some great buildings at Glemesford, and the abbot granted this request against his will, for he did not at that time dare to offend the bishop. But while the abbot was staying at Melford, a certain clerk of the bishop came and asked on behalf of his lord that they might be allowed to take the said wood at Elmswell; and he made a mistake in his speech, saying Elmswell where he should have said Elmset, the latter being the name of a certain wood at Melford. And the abbot marvelled at the message, for such wood was not to be found at Elmswell.

\switchcolumn*

\begin{otherlanguage}{latin}
Quod cum audisset Ricardus forestarius de eadem villa, dixit occulte abbati, episcopum misisse proxima ebdomada pr\ae{}terita carpentarios suos tanquam exploratores in boscum de \AE{}lmessethe, et eligisse\footnote[\textdagger]{lege, \emph{elegisse}.} meliora ligna totius bosci, et signa sua imposuisse. Quo audito, subito comperit abbas nuntium episcopi errasse in mandato, respondens ei se facere libenter voluntatem episcopi. 
\end{otherlanguage}

\switchcolumn

Then when Richard the forester of the same township had heard this, he told the abbot privately that the bishop in the preceding week had sent his carpenters as spies into the wood of Elmset, and that they had chosen the best trees in the whole wood, and marked them with their signs. At this news, the abbot saw at once that the messenger of the bishop had delivered his message wrongly, and told him that he would gladly meet the wish of the bishop.

\switchcolumn*

\begin{otherlanguage}{latin}
In crastino recedente nuntio, statim post missam auditam ivit abbas cum carpentariis suis in boscum pr\ae{}nominatum, et omnes quercus prius signatas, cum plusquam centum aliis, suo signo signari fecit ad opus Sancti \AE{}dmundi, et ad culmen magn\ae{} turris, pr\ae{}cipiens ut quantocius succiderentur.
\end{otherlanguage}

\switchcolumn

On the morrow, after the messenger had departed, as soon as he had heard mass, the abbot went with his carpenters into the said wood, and caused all the oaks which had been already marked, and more than a hundred others, to be marked for the use of St.\ Edmund, and for the completion of the great tower; and he ordered that they should be cut down as rapidly as possible.

\switchcolumn*

\begin{otherlanguage}{latin}
Episcopus autem, cum ex responso sui nuntii intellexit ligna pr\ae{}dicta apud \AE{}lmeswell esse capienda, eundem nuntium multis contumeliis affectum remisit ad abbatem, ut verbum in quo erraverat corrigeret, dicendo \AE{}lmesethe non \AE{}lmeswell; sed antequam venisset ad abbatem, jam succisa erant omnia ligna qu\ae{} episcopus desideraverat, et carpentarii sui signaverant. Unde et eum ligna alia et alibi capere oporteret si vellet. Ego autem, quando hoc videbam, ridebam, et dicebam in corde meo: ``Sic ars deluditur arte.''
\end{otherlanguage}

\switchcolumn

But the bishop, when he learned from the report of his messenger that the needed wood was to be taken at Elmswell, overwhelmed the messenger with much abuse, and sent him back to the abbot that he might correct the word which he had said wrongly, that is, when he said Elmswell for Elmset. But before he had come to the abbot, all the trees, which the bishop desired, and which his carpenters had marked, had been cut down. It was therefore necessary for the bishop to take other trees and in another place, if he would. But I, when I saw this, laughed and said in my heart, ``Thus craft is defeated by craft.''

\switchcolumn*

\begin{otherlanguage}{latin}
\blockhead[\textsc{a.d}.\ \oldstylenums{1182}.]{How there were disputes concerning the appointment of bailiffs for the town.}{5}{-.6cm}
Mortuo Hugone abbate, voluerunt custodes abbati\ae{} deponere pr\ae{}fectos vill\ae{} Sancti \AE{}dmundi et novos constituere sua auctoritate, dicentes hoc pertinere ad regem, in cujus manu abbatia fuit. Nos autem inde conquerentes, misimus nuntios nostros domino Ranulfo de Glanvilla, tunc justiciario; qui respondit se bene scire xl.\ libras debere reddi de villa\engnotetext{Rokewode (p.\ \oldstylenums{136}) states that Eugenius III.\ confirmed the appropriation of the rents of the town to the use of the sacrist, in the days of abbot Ording.} annuatim sacristi\ae{} nostr\ae{}, et nominatim ad luminaria ecclesi\ae{}; et dixit H.\ abbatem pro voluntate sua et in talamo suo sine consensu conventus pr\ae{}fecturam dedisse quotiens voluit et quibus voluit, salvis xl.\ libris altari reddendis; et ideo non esse mirandum si bailivi regis hoc ipsum exigerent ex parte regis, et acerbe loquens, nos omnes monachos stultos nominavit, ex hoc quod passi sumus abbatem nostrum talia fecisse; non advertens quod monachorum summa religio tacere est, et excessus suorum pr\ae{}latorum clausis oculis pr\ae{}terire; nec attendens quod baratores dicuntur, si in aliquo, sive juste sive injuste, contradicimus, et quandoque rei l\ae{}s\ae{} majestatis, vel carceris vel exilii p\oe{}na damnamur; unde et sanius consilium mihi et consimilibus meis videtur, ut confessores, quam ut moriamur martyres.

\end{otherlanguage}

\switchcolumn

On the death of abbot Hugh, the wardens of the abbacy desired to depose the bailiffs of the town of St.\ Edmund's, and to appoint new bailiffs by their authority, alleging that this right pertained to the king, in whose hand the abbey was. But when we made complaint on this matter and sent our messengers to the lord Ranulf Glanvill, who was then justiciar, he answered that he was well aware that forty pounds ought to be rendered from the town\engnotenum{} to our sacristan annually, and especially for the lights of the church. And he added that abbot Hugh, according to his pleasure and in his chamber, without the assent of the monastery, had given the office of bailiff as often and to whomsoever he would, saving the forty pounds of revenue for the altar; it was therefore not remarkable if the officials of the king exacted this right on the king's behalf. Then in rough tones, he called all us monks fools in that we had allowed our abbot to do such things, not thinking that the chief duty of monks is to keep silence, and to shut their eyes to all the faults of their prelates, nor thinking of the fact that we are called cheats if in any matter, either rightly or wrongly, we raise opposition, and that sometimes we are accused of treason, and sometimes condemned to imprisonment and exile. For these reasons, it seems the wiser counsel to me and those in my position, rather to die as confessors than as martyrs.

\switchcolumn*

\begin{otherlanguage}{latin}
Redeunte ad nos nuntio nostro, et narrante qu\ae{} audierat et viderat, quasi inviti et coacti inivimus consilium ut, communi voluntate et conventus et custodum abbati\ae{}, deponerentur veteres pr\ae{}fecti vill\ae{}, reluctante Samsone subsacrista nobiscum, quantum potuimus.
\end{otherlanguage}

\switchcolumn

When our messenger returned to us and told what he had heard and seen, we took counsel, as it were unwillingly and under compulsion, to the effect that, by the common action of the monastery and of the wardens of the abbey, the old bailiffs of the town should be deposed; this was as far as possible to be a joint act, though Samson, who was our subsacristan, was opposed to the plan.

\switchcolumn*

\begin{otherlanguage}{latin}
Samson autem abbas factus, non immemor injuri\ae{} conventui illat\ae{}, in crastino Pasch\ae{} proxim\ae{} post electionem suam fecit conveniri in capitulo nostro milites et clericos et multitudinem burgensium, et coram omnibus dixit istam villam pertinere ad conventum et ad altare, nominatim ad invenienda luminaria ecclesi\ae{}; et se velle renovare antiquam consuetudinem, ut coram conventu et cum communi assensu tractaretur de pr\ae{}fectura vill\ae{} et consimilibus qu\ae{} ad conventum pertinebant. 
\end{otherlanguage}

\switchcolumn

At a later date, when he had been made abbot, Samson was not unmindful of the injury which had been done to the abbey, and on the morrow of the Easter next after his election, he caused the knights and clerks and many burgesses to be gathered together in our chapter, and in the presence of all of them said that this town belonged to the monastery and to the altar, especially in the matter of finding lights for the church. And he said that he wished to renew the old custom, that all that concerned the bailiwick of the town and such like matters pertaining to the monastery, should be decided in the presence of the monastery and by common assent.

\switchcolumn*

\begin{otherlanguage}{latin}
Et nominati sunt eadem hora duo burgenses, Godefridus et Nicholaus, ut essent pr\ae{}fecti, habitaque disputatione de cujus manu cornu acciperent, quod dicitur \emph{mot-horn},\engnotetext{The horn which was sounded to summon to assemblies of the moots or meetings, whether of the shire or boroughs. (Cp.\ Rokewode, p.\ \oldstylenums{136}--\oldstylenums{7}.)} tandem illud receperunt de manu prioris, qui post abbatem caput est de rebus conventus. Illi autem duo pr\ae{}fecti bailivam suam pacifice custodierunt per plures annos, quousque dicerentur remissi in justitia regis custodienda: dictante autem ipso abbate, ut major securitas daretur conventui super hac re, remotis illis, recepit Hugo sacrista villam in manu sua, novos servientes constituens, qui ei de pr\ae{}fectura responderent; sed processu temporis, nescio quomodo, postea novi pr\ae{}fecti substituti sunt alibi quam in capitulo, et sine conventu; unde vel simile vel majus timetur periculum post decessum abbatis Samsonis, quam fuerit post decessum Hugonis abbatis.
\end{otherlanguage}

\switchcolumn

And at that same time two burghers, Godfrey and Nicholas, were named bailiffs, and there was a dispute as to the question from whose hand they should receive the horn, which is called moot-horn.\engnotenum{} At last they received it from the hand of the prior, who after the abbot is the chief man in the affairs of the monastery. Then those two bailiffs peacefully exercised their jurisdiction for many years, until they were said to be slack in administering the justice of the king. Then at the suggestion of the abbot himself, that greater security might be given to the monastery in this matter, they were removed, and the sacristan Hugh received it into his hand. He  appointed new servants, who were responsible to him for the bailiwick. In course of time, however---I know not for what reason---new bailiffs were again appointed elsewhere than in the chapter, and without the assent of the monastery, on which account there was a fear of the same or greater danger after the death of abbot Samson than there had been after the death of abbot Hugh.

\switchcolumn*

\begin{otherlanguage}{latin}
Quidam autem ex fratribus nostris, de amore et familiaritate abbatis plenius confidens, nacta opportunitate, et modeste sicut decuit, convenit inde abbatem, asserens inde murmur fieri in conventu. Abbas vero, his auditis, diu tacuit, ac si aliquantulum inde turbaretur, et tandem ita dicebatur respondisse: ``Nonne ego, ego sum abbas? nonne mea interest disponere de rebus ecclesi\ae{} mihi commiss\ae{}, dummodo sapienter egero et secundum Deum? Si defectus fuerit regi\ae{} justiti\ae{} in villa ista, ego calumniatus ero, ego ero summonitus, mihi incumbet labor itineris et expens\ae{}, et defensio vill\ae{} et pertinentium; ego stultus habebor, non prior, non sacrista, non conventus; sed ego, qui caput eorum sum et esse debeo. Per me et consilium,\footnote[\textdagger]{``Meum,'' desiderari videtur. Roke.} Domino adjuvante, erit villa servata indeniniter pro posse meo, et salv\ae{} erunt quadraginta libr\ae{} annuatim reddend\ae{} altari. Murmurent fratres; detrahant; dicant inter se, quod voluerint; pater eorum sum et abbas; quamdiu vixero, honorem meum alteri non dabo.''
\end{otherlanguage}

\switchcolumn

And one of our brothers, who was fully confident of the love and friendship of the abbot, took an opportunity to address the abbot on this matter, with due modesty, and he declared that the monastery murmured for this cause. When the abbot heard this he was long silent, as if he were somewhat troubled for this, and at last, so it is said, answered as follows: ``Am not I, even I, the abbot? Is it not my affair to dispose of the goods of the church committed to my care, provided that I act wisely and according to God? If there be a default in the administration of the king's justice in this town, I shall be accused on the matter, I shall be summoned to the court, on me will fall the labour of the journey and the expense, and the defence of the town and of that which pertains thereto. It will be I who am regarded as a fool, and not the prior or the sacristan or the monastery. No, it will be I, who am and ought to be their head. By my means and by my counsel, with God's help, the town shall be preserved unharmed so far as in me lies, and the forty pounds of annual rent to the altar shall be preserved. Let the brothers murmur; let them blame me; let them say what they will among themselves. I am their father and abbot, and while I live, I will not give mine honour to another.''

\switchcolumn*

\begin{otherlanguage}{latin}
His dictis, recessit monachus, qui eadem responsa referebat. Ego autem de talibus verbis mirabar, et contrariis motibus mecum disputavi; tandem dubitare coactus, eo quod regula juris dicit et docet, ut omnia sint in dispositione abbatis.\footnote[\textdagger]{Grat.\ Decr.\ Pars Secunda, xviii.\ \oldstylenums{2, 9}: ``Abbatis solicitudo, \emph{ad quem potestas tota pervenire convenit}.''}
\end{otherlanguage}

\switchcolumn

When he had so spoken, the monk left him and told his answers. But I marvelled at such words, and thought the thing over in secret and with care. At last I was forced to doubt still, since the rule of law says and teaches us that all things are in the disposition of the abbot.

\switchcolumn*

\begin{otherlanguage}{latin}
\blockhead{How abbot Samson disputed with the men of London about the payment of tolls.}{4}{-.65cm}
Mercatores Lundonienses voluerunt esse quieti de theloneo ad nundinas Sancti \AE{}dmundi; plures tamen inviti et coacti dederunt illud; unde multus tumultus et commotio magna facta est inter cives Lundoni\ae{} in suo hustengio. Convenientes ergo inde, abbati S.\ dixerunt, se quietos esse debere per totam Angliam auctoritate cart\ae{}, quam habuerunt de rege Henrico secundo.

\end{otherlanguage}

\switchcolumn

The merchants of London wished to be quit from toll at the fair of St.\ Edmund's. Many, however, though unwillingly and under compulsion, paid it, and on this account many tumults and a great disturbance occurred between the citizens of London in their court. Wherefore, having held a meeting about the matter, they sent word to abbot Samson that they ought to be quit of toll throughout all England, under the authority of the charter which they held from king Henry the Second.

\switchcolumn*

\begin{otherlanguage}{latin}
Quibus abbas respondit, quod, si necesse esset, bene posset trahere regem in warantum, quod nunquam aliquam cartam eis fecit in pr\ae{}judicium ecclesi\ae{} nostr\ae{}, nec in detrimentum libertatum Sancti \AE{}dmundi, cui sanctus \AE{}dwardus \emph{tollum} et \emph{themum}\engnotetext{These two words are defined in Stubbs' \emph{Select Charters} (glossary) as follows: ``Tol, duty on imports. Theam, the right of compelling the person in whose hands stolen or lost property was found to vouch to warranty, that is, to name the person from whom he received it.'' On the other hand, the terms, in conjunction with infangenthef, are often used as a mere jingle which has no particular meaning; and taken together, tol and theam may be said to mean the ordinary rights conferred on towns---that is, a vague concession.} et omnia jura regalia concessit et confirmavit, ante conquestum Angli\ae{}; et quod rex Henricus dedit Limdoniensibus quietantiam thelonei per dominia sua propria, ubi poterat dare eam; quia in villa Sancti \AE{}dmundi non poterat, quod suum non erat.
\end{otherlanguage}

\switchcolumn

To this the abbot answered that, were it needful, he could easily bring the king to warrant him that he had never made them a charter in prejudice of our church, or to the injury of the liberties of St.\ Edmund, to whom the holy Edward had granted and confirmed toll and theam\engnotenum{} and all regalian rights before the conquest of England. And he added that king Henry had given to the Londoners quittance from toll throughout his own demesnes, where he had the right to give it; for in the city of St.\ Edmund's he could not give it, for it was not his to give.

\switchcolumn*

\begin{otherlanguage}{latin}
Audientes h\ae{}c Lundonienses, communi consilio decreverunt, quod nullus ex eis veniret ad nundinas Sancti \AE{}dmundi, et duobus annis absentaverunt se, unde magnum detrimentum habuerunt nundin\ae{} nostr\ae{}, et oblatio sacrist\ae{} minorata fuit in magna parte. Tandem, episcopo Lundonensi et aliis pluribus interloquentibus, ita convenit inter nos et eos, quod ipsi venirent ad nundinas, et aliqui ex eis darent theloneum, sed statim eisdem redderetur, ut sub tali velamento utrimque libertas servaretur.
\end{otherlanguage}

\switchcolumn

When the Londoners heard this, they decreed with common assent that none of them should come to the fair of St.\ Edmund's, and for two years they did absent themselves, whence our fair suffered great loss, and the offerings in our sacristry were greatly diminished. Eventually, when the bishop of London and many others had mediated, an agreement was reached between them and us whereby they should come to the fair, and some of them should pay toll, but this should be at once returned to them, that by such a device the privilege of both parties might be maintained.

\switchcolumn*

\begin{otherlanguage}{latin}
Sed processu temporis, cum fecisset abbas concordiam cum militibus suis, et quasi in pace dormisset, ecce iterum ``Philistiim super te, Samson!''\engnotetext{Jud.\ xvi., \oldstylenums{9}.} Ecce Londonienses, una voce comminantes, quod domos lapideas, quas abbas eodem anno \ae{}dificaverat, ad terram prosternerent, vel contra, namum de hominibus Sancti \AE{}dmundi in centuplum acciperent, nisi abbas citius emendaret injuriam cis illatam a pr\ae{}fectis vill\ae{} Sancti \AE{}dmundi, qui xv.\ denarios acceperant a carettis civium Londoniensium, qu\ae{}, venientes de Gernemue, allecia portantes, transitum per nos fecerunt. Et dicebant cives Lundonienses fuisse quietos de theloneo in omni foro, et semper et ubique, per totam Angliam, a tempore quo Roma primo fundata fuit, et civitatem Lundoni\ae{}, eodem tempore fundatam, talem debere habere libertatem per totam Angliam, et ratione civitatis privilegiat\ae{}, qu\ae{} olim metropolis fuit et caput regni, et ratione antiquitatis.
\end{otherlanguage}

\switchcolumn

But as time went on, when the abbot had come to an agreement with his knights, and as it were, rested in peace, lo! again, ``The Philistines be upon thee, Samson!''\engnotenum{} For the Londoners, with one voice, threatened to level with the earth the stone houses, which the abbot had built in the same year, or to take distress a hundredfold from the men of St.\ Edmund, if the abbot not at once make reparation to them for the wrong which they had suffered from the bailiffs of the town of St.\ Edmund's. For they had taken fifteen pence from the carts of the citizens of London, which were coming from Yarmouth and carrying herrings, and which passed through our town. And the citizens of London said that they had been quit of toll in every market, and always and in every place, throughout all England, from the time when the city of Rome was first founded, at which time the city of London was also founded. They said that they ought to have this privilege throughout all England, both on the ground that their city was a privileged city, which had been the metropolis and capital of the kingdom, and on the score of the antiquity of the city.

\switchcolumn*

\begin{otherlanguage}{latin}
Abbas vero competentes indutias qu\ae{}sivit inde, usque ad reditum domini regis in Angliam, ut eum super hoc consuleret; et, habito consilio cum juris-discretis, replegiavit calumniatoribus illos xv.\ denarios, reservata utrique parti qu\ae{}stione de jure suo.
\end{otherlanguage}

\switchcolumn

The abbot, however, asked for a truce on this dispute for a reasonable time, until the return of the king to England, that he might consult with him on this matter; and taking the advice of men skilled in the law, he handed back to the complainants those fifteen pence as a pledge, without prejudice to the question of the right of either party.

\switchcolumn*

\begin{otherlanguage}{latin}
\blockhead[\textsc{a.d}.\ \oldstylenums{1192}.]{How there was a dispute with the burghers as to the dues from the town.}{4}{-.65cm}
Decimo anno abbati\ae{} S.\ abbatis, de communi consilio capituli nostri, conquesti sumus abbati in curia sua, dicentes redditus et exitus omnium bonarum villarum et burgorum Angli\ae{} crescere et augmentari, in commodum possidentium et emendationem dominorum, pr\ae{}ter villam istam, qu\ae{} xl.\ libras dare solet, et nunquam ad plus extenditur; et burgenses vill\ae{} esse in causa hujusmodi rei, qui tantas et tot purpresturas tenent in foro, de sopis et seldis, et stalagiis, sine assensu conventus, et ex solo dono pr\ae{}fectorum vill\ae{}, qui annuales firmarii et quasi servientes sacrist\ae{} fuerunt, pro beneplacito ejus removendi. 

\end{otherlanguage}

\switchcolumn

In the tenth year of the abbacy of abbot Samson, by common counsel of our chapter, we made complaint to the abbot in his court and said that the receipts from all the goods of the towns and boroughs of England were increased, and had grown to the advantage of the possessors and the greater profit of their lords, save in the case of this town, which had been wont to pay forty pounds and had never had its dues increased. And we said that the burghers of the city were responsible for this, since they held so many and such large stands in the market-place, shops and sheds and stalls, without the assent of the monastery, and at the sole gift of the bailiffs of the town, who were annual holders of their offices, and as it were servants of the sacristan, being removable at his good pleasure.

\switchcolumn*

\begin{otherlanguage}{latin}
Burgenses vero summoniti responderunt, se esse in assisa regis, nec de tenementis, qu\ae{} illi et patres eorurn tenuerunt, bene et in pace, uno anno et uno die, sine calumnia, se velle respondere contra libertatem vill\ae{} et cartas suas; et dixerunt talem fuisse consuetudinem antiquam, ut pr\ae{}fecti darent, inconsulto conventu, loca soparum et seldarum in foro per aliquem redditum pr\ae{}fectur\ae{} annuatim reddendum. Nos autem reclamantes volumus, ut abbas dissaisiaret eos de talibus tenementis, unde warantum nullum habuerunt.
\end{otherlanguage}

\switchcolumn

But when the burghers were summoned, they answered that they were under the jurisdiction of the king, and that they ought not to make reply, contrary to the liberty of the town and their charters, concerning that which they had held and their fathers well and in peace, for one year and a day without dispute. And they said that it was the old custom that the bailiffs should, without consulting the monastery, give to them places for shops and sheds in the market-place, in return for some annual payment to the bailiwick. But we disputed this, and wished the abbot to dispossess them of such things as they held without having any warrant for them.

\switchcolumn*

\begin{otherlanguage}{latin}
Abbas vero veniens ad consilium nostrum, tanquam unus ex nobis, secreto nobis dixit, se velle nobis rectum tenere pro posse suo; sed ordine justiciario se debere procedere, nec sine judicio curi\ae{} posse dissaisiare liberos homines suos de terris vel redditibus suis, quos per plures annos tenuerunt, sive juste, sive injuste: quod si faceret, dicebat se cadere in misericordiam regis per assisam regni.
\end{otherlanguage}

\switchcolumn

Then the abbot came to our council, as if he had been one of ourselves, and privately informed us that he wished, so far as he could, to do right to us; but that he had to proceed in a judicial manner, and that he could not, without the judgment of the court, dispossess his free men of their lands and revenues, which they had, whether rightly or wrongly, held for many years. He added that if he were to do this, he would be liable to punishment at the discretion of the king and at the assizes of the kingdom.

\switchcolumn*

\begin{otherlanguage}{latin}
Burgenses ergo, ineuntes consilium, optulerunt conventui redditum c.\ solidorum pro bono pacis, et ut tenerent tenementa sua, sicut solebant. Nos vero hoc noluimus concedere, malentes ponere loquelam in respectum, sperantes forsitan, tempore alterius abbatis, vel omnia recuperare, vel locum nundinarum mutare; et ita res cepit dilationem per plures annos. 
\end{otherlanguage}

\switchcolumn

The burghers, therefore, took counsel and offered the monastery a revenue of a hundred shillings for the sake of peace, and that they might hold that which they held as they had been accustomed. But we would not grant this, preferring to postpone the matter, and perchance hoping that in the time of another abbot, either we might recover all, or change the place of the fair; and so the matter for many years advanced no further.

\switchcolumn*

\begin{otherlanguage}{latin}
\blockhead[\textsc{a.d}.\ \oldstylenums{1194}.]{Concerning the charter granted to the town by the abbot.}{4}{-.65cm}
Cum autem redisset abbas de Allemannia, optulerunt ei burgenses lx.\ marcas, et petierunt confirmationem suam de libertatibus vill\ae{}, sub eadem forma verborum, sicut pr\ae{}decessores ejus Anselmus et Ordingus et Hugo eis confirmaverant; quod et abbas Samson benigne annuit eis.\engnotetext{Samson's charter is to be found in the \emph{Monasticon} (Rokewode, p.\ \oldstylenums{137}).} Nobis autem murmurantibus et grunnientibus, facta est eis carta, sicut eis promiserat: et quia pudor esset ei et confusio, si non posset implere quod promiserat, noluimus ei contradicere, nec ad iracundiam provocare. Burgenses autem, ex quo habuerunt cartam abbatis S.\ et conventus, confidebant plenius, quod nunquam tempore abbatis S.\ amitterent tenementa sua nec libertates suas; unde nunquam postea, sicut prius, voluerunt pr\ae{}nominatum redditum centum solidorum dare nec offerre. 

\end{otherlanguage}

\switchcolumn

But when the abbot had returned from Germany, the burghers offered him sixty marks, and sought his confirmation of the liberties of the town, under the same form of words as that in which his predecessors Anselm and Ording and Hugh had confirmed them to them. And this also the abbot graciously conceded.\engnotenum{} But while we murmured and grumbled, a charter was made for them, as he had promised to them; and as it would have been a source of shame and confusion to him if he had been unable to do that which he had promised, we would not oppose him, or provoke him to anger. But the burghers, from the time when they had the charter of abbot Samson and of the monastery, were full of confidence that they would never lose their holdings and liberties in the time of abbot Samson; and, therefore, never afterwards would they, as they had done before, give or offer the said revenue of a hundred shillings.

\switchcolumn*

\begin{otherlanguage}{latin}
Abbas autem, hoc tandem advertens, convenit burgenses super hoc, dicens quod, nisi facerent pacem cum conventu, prohiberet seldas eorum \ae{}dificari ad nundinas Sancti \AE{}dmundi. Illi vero responderunt, se velle dare singulis annis cappam sericam, vel aliquod aliud ornamentum ad pretium centum solidorum, sicut prius promiserunt; sed tamen hoc pacto, ut quieti essent imperpetuum de decimis denariorum, quos sacrista acriter ab eis exigebat.
\end{otherlanguage}

\switchcolumn

The abbot, however, at length turned his attention to this, and assembling the burghers together about the matter, said that if they would not make their peace with the monastery, he would forbid their sheds to be put up at the fair of St.\ Edmund's. Then they answered that they would give every year a silken cope, or some other ornament to the value of a hundred shillings, as they had before promised. But they offered this on condition that they should be quit for ever from the tenths on their money which the sacristan sternly exacted from them.

\switchcolumn*

\begin{otherlanguage}{latin}
Abbas autem et sacrista hoc contradixerunt, et ideo posita est iterum loquela in respectum. Nos vero illos c.\ solidos huc usque amisimus, secundum quod vulgariter solet dici: ``Qui non vult capere quando potest, non capiet quando volet.''
\end{otherlanguage}

\switchcolumn

The abbot and the sacristan opposed this offer, and so the dispute was again left unsettled. But we have lost those hundred shillings from that day to this, according as is said in the proverb, ``He that will not when he may, when he will he shall have nay.''

\switchcolumn*

\begin{otherlanguage}{latin}
\blockhead{How the monastery was troubled with incompetent cellarers.}{4}{-.45cm}
Celerarii celerariis plures pluribus succedebant, et quilibet eorum in fine anni magno debito deprimebatur. Dabantur celerario in auxilium xx.\ libr\ae{} de Mildenhal, nec sufficiebant. Assignat\ae{} sunt postea quinquaginta libr\ae{} celerario singulis annis de eodem manerio; et adhuc dicebat celerarius hoc non sufficere.

\end{otherlanguage}

\switchcolumn

Cellarers succeeded each other in rapid succession, and each one of them at the end of the year was oppressed with great debt. The twenty pounds from Mildenhall were given to assist the cellarer, and proved insufficient. Afterwards there were assigned to him forty pounds a year from the same manor, and still the cellarer said that this was not enough.

\switchcolumn*

\begin{otherlanguage}{latin}
Volens ergo abbas indemnitati et utilitati tarn su\ae{} quam nostr\ae{} providere, sciens in omni defectu nostro ad eum, tanquam ad patrem monasterii, esse recurrendum, quendam clericum de mensa sua, magistrum Ranulfum nomine, celerario nostro associavit, ut ei tanquam testis et socius assisteret, et in expensis et in receptis.
\end{otherlanguage}

\switchcolumn

Accordingly the abbot, wishing to provide for his own safety and comfort and for ours, as he knew that in the case of any deficiency we must appeal to him, as the father of the monastery, associated with our cellarer one of the clerks of his table, master Ranulf by name, that he might assist him as a witness and companion both in the paying out and in the receiving of money.

\switchcolumn*

\begin{otherlanguage}{latin}
Et ecce! multi multa loquuntur. Densescunt murmurationes, fabricantur mendacia, consuuntur detractiones detractionibus, nec est angulus in domate, qui venenoso non resonet sibilo. Dicit aliquis alicui: ``Quid est quod factum est? quis vidit talia? nunquam tale dedecus factum est conventui. Ecce! abbas constituit clericum super monachum; ecce! clericum constituit magistrum et custodem super celerarium, ut nichil boni possit facere sine eo. Monachos suos vilipendit abbas, monachos suspectos habet, clericos consulit, clericos diligit. `Quomodo obscuratum est aurum, mutatus est color optimus.'\hspace{1pt}''\engnotetext{Lament.\ iv., \oldstylenums{1}.}
\end{otherlanguage}

\switchcolumn

And behold! there was much diversity of opinion. Murmurs rose; lies were invented; slanders are joined to slanders, nor is there a spot in the house which is not resounding with poisonous hissings. One says to another, ``What is this which has been done? Who has seen such things? Never was such a disgrace inflicted on a monastery. See! the abbot has made a clerk superior to a monk. See! he has made a clerk master and guardian over the cellarer, that he may be able to do no good without him. The abbot little regards his monks; he suspects them: he consults the clerks; he loves clerks. How is the gold become dim! how is the most fine gold changed!''\engnotenum{}

\switchcolumn*

\begin{otherlanguage}{latin}
Dixit item amicus amico: ``Facti sumus in opprobrium vicinis nostris. Omnes nos monachi vel infideles, vel improvidi reputamur; clerico creditur, non monacho: magis confidit abbas de clerico quam de monacho. Numquid clericus ille magis fidelis est, vel magis sapiens, quam aliquis monachus?'' Item dixit socius socio: ``Celerarius et subcelerarius nonne sunt, vel esse possunt, tam fideles homines ut sacrista, vel ut camerarius? Consequens ergo est, ut iste abbas, vel successor ejus, clericum ponat cum sacrista, clericum cum camerario, clericum cum subsacristis, ad colligendam oblationem ad feretrum, et sic cum singulis officialibus, unde nos erimus in subsannationem et derisum omni populo.''
\end{otherlanguage}

\switchcolumn

And friend said to friend, ``We are become a reproach to our neighbours. All we monks are held to be either dishonest or extravagant; credit is given to a clerk, not to a monk; the abbot has more confidence in a clerk than in a monk. Why should this clerk be more faithful, or wiser, than some monk?'' And one said to his companion, ``Are not our cellarer and sub-cellarer, or cannot they be, as faithful men as the sacristan or as the chamberlain? The result is that this abbot or his successor will associate a clerk with the sacristan, and a clerk with the chamberlain, and a clerk with the subsacristans, to collect the offerings at the shrine, and so with all the officials, whereby we shall become a laughing-stock and a scorn to all people.''

\switchcolumn*

\begin{otherlanguage}{latin}
Ego autem talia audiens, solebam respondere: ``Si ego essem celerarius, vellem utique, ut clericus mihi testis esset in omnibus agendis; quia si bene facerem, ipse testimonium perhiberet de bono; si vero in fine anni aliquo debito oppressus essem, credi possem et excusari per clericum illum.''
\end{otherlanguage}

\switchcolumn

But when I heard these things, I used to answer, ``If I were cellarer, I should wish to have a clerk to be my witness in all that had to be done, and so if I did well, he would bear witness of the good, but if at the end of the year I were burdened with any debt, I should be able by means of the clerk to gain credence and be excused.''

\switchcolumn*

\begin{otherlanguage}{latin}
Unum autem ex fratribus nostris, virum utique discretum et litteratum, audivi quiddam dicentem, quod movit me et plures alios: ``Non est,'' inquit, ``mirandum, si dominus abbas de nostris rebus custodiendis partes suas interponat, qui portionem abbati\ae{}, qu\ae{} eum contingit, sapienter regit et domui su\ae{} sapienter disponit, cujus interest defectum nostrum supplere, si ex incuria vel inpotentia nostra contingat. Sed unum,'' inquit, ``restat periculum, post mortem S.\ abbatis futurum, quale nunquam nobis contigit diebus vit\ae{} nostr\ae{}. Venient sine dubio bailivi regis et saisiabunt abbatiam in manu sua, scilicet baroniam qu\ae{} pertinet ad abbatem, sicut olim factum est post mortem c\ae{}terorum abbatum; et sicut olim post mortem H.\ abbatis voluerunt ballivi regis constituere novos pr\ae{}fectos in villa Sancti \AE{}dmundi, auctoritate\footnote[\textdagger]{So J.; Mr.\ Gage Rokewode read ``auctoritatem.'' The meaning is the same in either case, ``alleging in warrant,'' (or, ``as their warrant'') ``that abbot Hugo had done this.''} allegantes H.\ abbatem hoc fecisse, consimili ratione, processu temporis, ballivi regis ponent clericum suum ad custodiendum cellarium, ut per eum et ad pr\ae{}ceptum ejus omnia fiant; et dicturi sunt se debere hoc facere, quia abbas Samson sic fecit; et ita poterunt commisceri et confundi res et redditus abbatis et conventus, quos abbas Robertus\engnotetext{Robert II.\ (\oldstylenums{1107}--\oldstylenums{12}). The charter of Henry I.\ confirming this division is to be found in the \emph{Monasticon} (ed.\ \oldstylenums{1846}, III., \oldstylenums{152}, No.\ xiv).}\footnote[\ddag]{Abbot Robert II., who died in \oldstylenums{1107}, a few weeks after he had been blessed by St.\ Anselm; see App., B. Mr.\ Rokewode follows Battely here in his mistaken belief that Robert survived till \oldstylenums{1112}; see his note on the passage.} bon\ae{} memori\ae{} requisito consilio distinxit, et ab invicem separavit.'' Cum hoc et consimilia verba audirem a viro magni consilii et provido, stupui et tacui, domnum abbatem de tali facto nec volens condemnare, nec volens excusare.
\end{otherlanguage}

\switchcolumn

I heard a certain one of our brothers, a man discreet and learned indeed, say a thing which moved me and many others. ``It is not wonderful,'' he declared, ``if the lord abbot intervenes in matters connected with the safe conduct of our affairs. For he rules wisely the part of the abbey, which belongs to him, and he makes wise dispositions for his house; and it is his concern to supply that which is lacking to us, if a want arises owing to our carelessness or poverty. But,'' he continued, ``there is yet one possible danger which may arise after the death of abbot Samson, and such a danger has never been in all the days of our life. There is no doubt that the officers of the king will come and will take the abbey into his hand, that is, the barony which pertains to the abbot, as has been done hitherto, after the death of other abbots, and as was done before on the death of abbot Hugh, the officials of the king will desire to appoint new bailiffs in the town of St.\ Edmund, alleging as their authority that abbot Hugh did this. And for the same reason, in course of time, the officials of the king will appoint a clerk of theirs to watch over the cellar, that all things may be done through him and by his command, and they will say that this ought to be done, because abbot Samson did so. And in this way they will be able to mingle and to confound the affairs and revenues of the abbot and of the monastery, which abbot Robert,\engnotenum{} of happy memory, after taking counsel, separated and divided from each other.'' And when I heard these and other like words from a man of great wisdom and prudence, I was overwhelmed and was silent, wishing not to condemn and wishing not to excuse the lord abbot in so great a matter.

\switchcolumn*

\begin{otherlanguage}{latin}
\blockhead[About \textsc{a.d}.\ \oldstylenums{1198}.]{How the abbot resisted Hubert Walter when he claimed legatine authority over the abbey.}{5}{-.6cm}
Hubertus Walteri\footnote[\textdagger]{Hubert's father, Harvey Walter, was descended from Hubert, the first Norman settler, who received grants of land at the time of the Conquest in Norfolk and Suffolk. Theobald, the favourite brother of the archbishop, migrated to Ireland, assumed the name of Boteler, and became the ancestor of the great house of Ormond. (Hook's Archbishops of Canterbury.)} archiepiscopus Cantuariensis, et legatus apostolic\ae{} sedis et justiciarius Angli\ae{}, cum multas ecclesias visitasset, et multa mutasset et innovasset jure legati\ae{},\engnotetext{Hubert Walter received the legation from Celestine III.\ in \oldstylenums{1195}. An account of his proceedings on the visitation of York, and the decrees which he issued there, are to be found in Hoveden (III., \oldstylenums{293}--\oldstylenums{7}). Hoveden (III., \oldstylenums{299}) also relates the deposition of the abbot of Thorney by Hubert Walter at this time.} rediens de matre sua carnali\footnote[\ddag]{This was Maude, daughter of Theobald de Valoines, a sister of Bertha, the wife of Ranulf de Glanville. (Hook.)} morante et moriente apud Derham,\footnote[*]{At West Dereham in \oldstylenums{1188}, Hubert, being then Dean of York, founded a house of Pr\ae{}monstratensian canons, for the benefit of his own and his parents' souls, and of those of Ranulf de Glanville and Bertha his wife, ``qui nos nutreierun.'' Dugdale, vi.\ \oldstylenums{899}.}\engnotetext{West Dereham, in Norfolk, was the birthplace of Hubert Walter.} transmisit ad nos ij.\ clericos suos, portantes literas domini sui signatas, quibus continebatur, ut dictis et factis eorum fidem haberemus. Illi vero abbati et conventui proposuerunt interroganda, utrum vellemus recipere dominum suum legatum ad nos venientem, sicat debet recipi legatus, et recipitur ab aliis ecclesiis. Quod et si concederemus, in brevi veniret ad nos, una cum consilio abbatis et conventus de rebus et negotiis ecclesi\ae{} nostr\ae{} secundum Deum dispositurus; quod et si ei hoc nollemus concedere, clerici illi duo mandatum domini sui nobis plenius exprimerent.

\end{otherlanguage}

\switchcolumn

Hubert Walter, archbishop of Canterbury, and legate of the apostolic see and justiciar of England, after he had visited many churches, and had changed many things and introduced innovations by his legatine authority,\engnotenum{} on his return from his carnal mother, who was dying at Dereham\engnotenum{} where she dwelt, he sent to us two of his clerks, who bore sealed letters from their lord, in which it was announced that we should give credence to their sayings and acts. Then these clerks made enquiry of the abbot and monastery as to whether we would receive their lord, the legate, who was coming to us, in such a manner as a legate ought to be received, and as he had been received in other churches. And if we would grant this, then he would shortly come to us, to settle the business and affairs of our church according to God, with the advice of the abbot and monastery. But if we would not grant this, those two clerks would explain to us more in detail the message of their lord.

\switchcolumn*

\begin{otherlanguage}{latin}
Abbate vero convocante plures de conventu, tale consilium inivimus, ut clericis ad nos missis benigne responderemus, dicentes, nos velle dominum suum recipere, ut legatum, cum omni honore et reverentia, et mittere simul cum eis nuntios nostros, qui hoc domino legato ex parte nostra dicerent; et consultum habuimus ut, sicut prius feceramus episcopo Eliensi et aliis legatis, ei omnem honorem exhiberemus cum processione, et campanis resonantibus, et cum ceteris solempnibus, eum reciperemus, donec veniretur ad scrutinium in capitulo forsitan faciendum: quod si vellet facere, tunc demum ei omnes unanimiter resisteremus in facie, Romam appellantes et cartis nostris innitentes. 

\end{otherlanguage}

\switchcolumn

Then the abbot summoned the majority of the monastery and we took counsel that we should give a favourable answer to the clerks who had been sent to us, saying that we would receive their lord, as legate, with all honour and respect, and that we should send with them our messengers, who should say the same to the lord legate on our behalf. And our idea was that, as we had previously done in the case of the bishop of Ely and other legates, we should show him all honour with a procession and the ringing of bells, and receive him with the other due ceremonies, until perchance he should come to hold a visitation in the chapter. And if he wished to do this, then at last we should all resist him in common to the face, appealing to Rome and trusting to our charters.

\switchcolumn*

\begin{otherlanguage}{latin}
Dixitque dominus abbas: ``Si ad pr\ae{}sens voluerit legatus ad nos venire, ita faciemus sicut supradictum est; si vero adventum suum ad nos distulerit ad tempus, interim dominum papam consulemus, qu\ae{}rentes, quam vim habere debeant privilegia ecclesi\ae{} nostr\ae{}, ab eo et antecessoribus suis impetrata, contra archiepiscopum, qui super omnes privilegiatas ecclesias Angli\ae{} a sede apostolica potestatem impetravit.'' Tale fuit consilinm nostrum.
\end{otherlanguage}

\switchcolumn

The lord abbot said also, ``If the legate now wishes to come to us, let us act in this manner as has been said. But if he delay his coming to us for a while, let us again seek counsel of the lord pope, and ask what force the privileges of our church, which we have obtained from him and his predecessors, ought to have as against the archbishop, who has obtained power from the apostolic see over all the other privileged churches of England.'' Such then was our counsel.

\switchcolumn*

\begin{otherlanguage}{latin}
Cum autem audisset archiepiscopus, quod vellemus recipere eum ut legatum, nuntios nostros gratanter recepit, et cum gratiarum actione. Et factus est domino abbati in omnibus negotiis suis benignus et propitius, et adventum suum ad nos quibusdam causis emergentibus distulit ad tempus. Omni ergo dilatione postposita, misit abbas domino pap\ae{} easdem literas, quas legatus miserat ei et conventui, in quibus continebatur quod ipse venturus esset ad nos auctoritate legati\ae{} su\ae{}, et auctoritate domini pap\ae{}, in quibus scribebatur, quod et ei data fuit potestas super omnes exemptas ecclesias Angli\ae{}, non obstantibus literis Eboracensi ecclesi\ae{} vel alicui impetratis. 
\end{otherlanguage}

\switchcolumn

And when the archbishop heard that we would receive him as legate, he received our messengers graciously, and with giving of thanks. And he became kind and friendly to the lord abbot in all his affairs, and delayed his coming among us for a while on account of some urgent business. Therefore, without any delay the abbot sent to the lord pope the same letters which the legate had sent to the monastery, which announced that he would come to us by the authority of his legation and by the authority of the lord pope, and in which it was written that power had been given him over all the exempt churches of England, notwithstanding letters granted to the church of York and others.

\switchcolumn*

\begin{otherlanguage}{latin}
Instante autem nuntio abbatis, scripsit dominus papa\engnotetext{Innocent III., who became pope in \oldstylenums{1198}, confirmed the privileges of St.\ Edmund, especially those granted by Alexander and Urban (Rokewode, p.\ \oldstylenums{137}). Arnold (\emph{Mem}., I., \oldstylenums{285}, note) suggests that as no such letter as that described in the text can be found, Jocelin misunderstood the statement of Samson, and that he really refers to the letter addressed by Innocent to the abbot and convent of St.\ Edmund.} domino Cantuariensi,\footnote[\textdagger]{No such letter is to be found in the Regesta of of Innocent III.'s correspondence in \oldstylenums{1198} and \oldstylenums{1199}, printed in Migne's \emph{Patrologie} (vol.\ ccxiv.). It seems probable that Jocelin misunderstood what he heard from Samson, and that the letter referred to was that of the \oldstylenums{1}st Dec.\ \oldstylenums{1198} (№ \oldstylenums{457} of the Regesta), which the pope addressed, not to the archbishop, but to the abbot and convent of St.\ Edmund. This letter opens thus, ``Cum ecclesia vestra ecclesi\ae{} Roman\ae{} sit filia specialis [note the resemblance of phrase], et ad eam nullo pertineat mediante, vos . . . . ad apostolicam sedem tanquam ad caput vestrum in arduis ducitis recurrendum.''} asserens ecclesiam nostram spiritualem filiam suam nulli legato respondere, nisi legato a latere domini pap\ae{} misso, et prohibuit, ne in nos manum extenderet; et adjecit dominus papa de suo, prohibens etiam, ne in aliquam aliam ecclesiam exemptam potestatem exerceret. Rediit ad nos nuntius noster, et absconditum fuit hoc aliquot diebus. Significatum tamen hoc fuit domino Cantuariensi a familiaribus suis de curia domini pap\ae{}.

\end{otherlanguage}

\switchcolumn

So at the instance of the messenger of the abbot, the lord pope wrote\engnotenum{} to the lord of Canterbury and declared that our church, his spiritual daughter, was accountable to no legate, unless it might be to a legate \emph{a latere}, sent by the lord pope, and forbade him to stretch forth his hand over us. And the lord pope added, on his own account, that he further forbade him to exercise power over any other exempt church. Our messenger returned to us and the matter was hidden for some days. But it was made known to the lord of Canterbury by his friends in the court of the lord pope.

\switchcolumn*

\begin{otherlanguage}{latin}
Cum autem in fine anni visitationem suam faceret legatus per Norfolch et Suthfolch, et venisset primo apud Colecestriam, misit legatus ad abbatem nuntium occulte, mandans ei, quod bene audivit dici, quod abbas impetraverat literas contra legatiam suam, et petens, ut mitteret ei amicabiliter literas illas. Et ita factum est. Habuit enim abbas duo paria literarum sub eadem forma. Abbas vero nec visitavit legatum, nec per se, nec per interpositam personam, quamdiu fuit in episcopatu Norwicensi, ne putaretur velle facere finem cum legato de hospitio ei faciendo, sicut ceteri monachi et canonici fecerunt. Legatus autem turbatus et iratus, et timens excludi, si ad nos veniret, per Norwicum, per Acram, per Derham, transivit in Heli, Lundoniam tendens. 
\end{otherlanguage}

\switchcolumn

Now when at the end of the year the legate made his visitation in Norfolk and Suffolk, and had come first to Colchester, he sent a secret messenger to the abbot, and informed him that he had heard it said many times that the abbot had obtained letters against his legation, and asked that he would send those letters to him in a friendly way. And so it was done, for the abbot had two copies of the letters in identical terms. But the abbot visited the legate neither in person nor through any intermediary, as long as he was in the bishopric of Norwich, lest he might be thought to desire to come to terms with the legate as to the matter of providing entertainment for him, as the other monks and canonical persons had done. The legate, for his part, was disturbed and angry, and fearing that he would be shut out if he should come to us, passed through Norwich, Acre, and Dereham to Ely, making his way towards London.

\switchcolumn*

\begin{otherlanguage}{latin}
Abbate autem apparente, infra mensem, coram legato, inter Waltham et Lundoniam, in via regia, convenit eum quod noluit ei occurrere utpote justitiario domini regis quando fuit in regione ista. Abbas autem respondit, eum non isse ut justitiarium, sed ut legatum, facientem scrutinium in singulis ecclesiis, et allegavit rationem temporis, et quod passio Domini instabat, et oportebat eum interesse divinis obsequiis et claustralibus observantiis. Cum autem verba verbis, objectiones objectionibus opposuisset abbas, nec posset minis terreri nec flecti, respondit legatus cum indignatione se bene scire quod disputator bonus esset, et illum esse meliorem clericum, quam legatus esset.
\end{otherlanguage}

\switchcolumn

Then within a month, the abbot appeared before the legate between Waltham and London, in the king's highway, and the legate blamed him because he would not come to him as justiciar of the lord king, when he was in that district. But the abbot answered that he had been there not as justiciar, but as legate, making scrutiny in all churches, and brought forward as his excuse the season of the year, and that the Passion of the Lord drew near, at which time it was necessary for him to be engaged in divine services and in the observances of the cloister. And when the abbot met words with words and objections with objections, and could not be terrified or bent, the legate answered with anger that he well knew that he was skilled in argument, and that he was a better clerk than the legate was.

\switchcolumn*

\begin{otherlanguage}{latin}
Abbas ergo, nec tacenda timide pr\ae{}teriens, nec dicenda tumide loquens, in audientia plurium respondit se talem esse, quod nunquam sustinebit ecclesi\ae{} su\ae{} libertatem quassari, nec propter defectum scienti\ae{} nec pecuni\ae{}, etiamsi oporteret eum mori vel perpetuo exilio damnari. Finitis autem his et hujusmodi altercationibus, incepit legatus erubescere, abbate parcius loquente, et rogante, ut mitius ageret cum ecclesia Sancti \AE{}dmundi, ratione natalis soli, quia quasi nativus Sancti \AE{}dmundi et ejus nutritus\footnote[\textdagger]{Nutritius? Roke.} fuit. Erubuit quidem, quia virus, quod intus conceperat, inconsulte effuderat.
\end{otherlanguage}

\switchcolumn

Therefore the abbot, without passing over in a cowardly manner things which ought not to be mentioned, or speaking arrogantly of those things which had to be said, in the audience of many men, answered that he was the type of man who would never allow the liberty of his church to be infringed, either owing to lack of knowledge or to lack of money, even if it were necessary for him to die or be condemned to perpetual banishment. And when an end was made of these and other like altercations, the legate began to blush when the abbot spoke softly and prayed that he would deal more mildly with the church of St.\ Edmund, for the sake of his native land, for he was as it were a son of St.\ Edmund and there brought up. He blushed, indeed, because he had thoughtlessly poured forth the poison which he had conceived within him.

\switchcolumn*

\begin{otherlanguage}{latin}
\blockhead[\textsc{a.d}.\ \oldstylenums{1199}.]{}{2}{-.7cm}
In crastino nuntiatum est archiepiscopo Cantuariensi, quod dominus Eboracensis veniret legatus in Angliam, et quod ipse suggesserat multa mala domino pap\ae{} de eo, dicens quod ipse gravaverat ecclesias Angli\ae{} causa visitationis su\ae{} de triginta millibus marcis argenti acceptis.\engnotetext{After having tried in vain to obtain a decision in his favour from Celestine III., Geoffrey secured such a decision from Innocent (Hoveden, IV., \oldstylenums{67}). The report that he was to receive legatine authority was, however, unfounded.} Misit ergo legatus ad abbatem clericos suos, rogans, ut scriberet cum ceteris abbatibus domino pap\ae{}, et excusaret eum. Quod concessit abbas, et testimonium perhibuit quod dominus Cantuariensis nec ad nostram ecclesiam venit, nec illam nec aliam gravavit ecclesiam, loquens secundum conscientiam suam. Et cum abbas tradidisset literas illas nunciis archiepiscopi, dixit coram omnibus, se non timere etiamsi voluerit archiepiscopus in his literis malignari; et responderunt clerici in periculo anim\ae{} su\ae{} dominum suum nichil doli velle machinari, sed tantum velle excusari, et ita facti sunt amici archiepiscopus\footnote[\textdagger]{The sincere and faithful character of archbishop Hubert comes out in all this narrative, which is so far in full agreement with Gervase's laudatory biography in the \emph{Actus Pontificum}.} et abbas.

\end{otherlanguage}

\switchcolumn

On the morrow, word was brought to the archbishop of Canterbury that the lord of York was coming to England as legate,\engnotenum{} and that he had hinted many evil things to the lord pope concerning him, saying that he had oppressed the churches of England by taking on account of his visitation thirty thousand marks of silver. For that reason the legate sent his clerks to the abbot, asking that he would write with the other abbots to the lord pope and defend him. And then the abbot granted this, and gave witness that the lord of Canterbury had neither come to our church, nor oppressed that or any other church, speaking according to his knowledge. And when the abbot had handed over those letters to the messengers of the archbishop, he said before all men that he had no fear even if the archbishop wished to make an improper use of these letters. And the clerks answered on the peril of their master's soul, that their lord wished not to deal craftily, but only desired to excuse himself. And so the archbishop and the abbot were made friends.

\switchcolumn*

\begin{otherlanguage}{latin}
\blockhead[\textsc{a.d}.\ \oldstylenums{1198}.]{How the abbot contended with his knights as to service across the sea.}{4}{-.65cm}
Pr\ae{}cepit rex Ricardus omnibus episcopis et abbatibus Angli\ae{}, ut de suis baroniis novem milites  facerent decimum, et sine dilatione venirent ad eum in Normanniam, cum equis et armis, in auxilium contra regem Franci\ae{}. Unde et abbatem oportuit respondere de iiij.\ militibus mittendis. Cumque summoneri fecisset omnes milites suos, et eos inde convenisset, responderunt, feudos suos, quod de Sancto \AE{}dmundo tenuerunt, hoc non debere, nec se, nec patres eorum, unquam Angliam exisse, sed scutagium aliquando ad prs\ae{}ceptum regis dedisse. \engnotetext{Hoveden (IV., \oldstylenums{40}) appears to relate the same demand as is here mentioned by Jocelin. ``In this year [\oldstylenums{1198}] Richard asked, through Hubert Walter, that the men of the kingdom of England should find him three hundred knights or give him so much money as would enable him to hire three hundred knights to serve for one year, at the rate of three Angevin shillings for each knight per diem.'' All the rest assented to the demand, but Hugh Avalon, bishop of Lincoln, refused to do so.}

\end{otherlanguage}

\switchcolumn

King Richard sent orders to all the bishops and abbots of England that every nine knights should furnish a tenth,\engnotenum{} and that these should come to him in Normandy without delay, with horses and arms, to help him against the king of France. It therefore was necessary for the abbot to be responsible for the sending of four knights. And when he had caused all his knights to be summoned, and had gathered them together on this matter, they answered that their fees, which they held of St.\ Edmund, were not liable to this, and that neither they, nor their fathers, had ever gone out of England, though they had sometimes paid scutage by command of the king.

\switchcolumn*

\begin{otherlanguage}{latin}
Abbas vero in arcto positus hinc videns libertatem suorum militum periclitari, illinc timens ne amitteret saisinam baroni\ae{} su\ae{} pro defectu servitii regis, sicut contigerat episcopo Lundoniensi et multis baronibus Angli\ae{}, statim transfretavit; et fatigatus multis laboribus et expensis et exeniis quamplurimis qu\ae{} regi dedit, in primis nullum potuit facere finem cum rege per denarios. Dicenti ergo, se non indigere auro nec argento, sed iiij$^\text{or}$ milites instanter exigenti, quatuor milites stipendiarios optulit abbas. Quos cum rex recepisset, apud castellum de Hou misit. Abbas autem in instanti eis xxxvi.\ marcas dedit ad expensas xl.\ dierum. \engnotetext{This castle was restored to Richard by the treaty of Issoudun (\oldstylenums{1196}). The terms of the truce are to be found in Hoveden (IV., \oldstylenums{3}).}
\end{otherlanguage}

\switchcolumn

Then the abbot was in a difficult position, as on the one hand he saw that the liberty of his knights was endangered, and on the other feared lest he might lose seisin of his barony for default of service to the king, as had happened to the bishop of London and to many barons of England. And he at once crossed the sea, and having wearied himself with many labours and expenses and with the many presents which he gave to the king, at first he was unable to come to terms with the king by means of a money payment. And so, when the king said that he needed neither gold nor silver, but instantly demanded the four knights, the abbot offered him four mercenary knights. And the king received these and sent them to the castle of Eu.\engnotenum{} And the abbot gave them thirty-six marks at once for their expenses for forty days.

\switchcolumn*

\begin{otherlanguage}{latin}
In crastino autem venerunt quidam familiares regis, consulentes abbati, ut sibi caute provideret, dicentes werram posse durare per annum integrum, vel amplius, et expensas militum excrescere et multiplicari in perpetuum damnum ei et ecclesi\ae{} su\ae{}. Et ideo consulebant, ut antequam recederet de curia finem faceret cum rege, unde posset quietus esse de militibus pr\ae{}dictis post xl.\ dies. Abbas autem, sano usus consilio, centum libras regi dedit pro tali quietantia, et ita cum gratia domini rediit in Angliam, breve regis portans secum, ut milites sui distringerentur per feudos suos ad reddendum ei servitium regis, quod ipse fecerat pro eis.
\end{otherlanguage}

\switchcolumn

But on the morrow there came certain friends of the king, and counselled the abbot that he should proceed cautiously, saying that the war might last for a full year or more, and the cost of the knights increase and multiply to the perpetual loss of him and his church. They therefore advised that before he left the court, he should come to terms with the king, so that he might be quit of the said knights after forty days. Then the abbot, taking good advice, gave a hundred pounds to the king for this quittance, and so returned to England in high favour with his lord. And he bore with him the king's writ that distraint should be levied on the fees of his knights to repay him for the service which he had made to the king on their behalf.

\switchcolumn*

\begin{otherlanguage}{latin}
Milites vero summoniti allegabant paupertatem suam, et multa genera gravaminum, et obtulerunt domino suo duas marcas de quolibet scuto. Abbas autem, non immemor, quod ipse eos gravaverat eodem anno, et inplacitaverat de scutagio integre reddendo, volens eos conciliare in gratiam, gratanter accepit quod illi gratanter optulerunt. 
\end{otherlanguage}

\switchcolumn

When the knights were summoned, they alleged their poverty and many different grievances, and offered their lord two marks from each fee. And the abbot, not unmindful of the fact that he had himself burdened them in that same year, and had brought a suit against them to force them to pay a full scutage, wished to win back their favour, and graciously accepted that which they willingly offered to him.

\switchcolumn*

\begin{otherlanguage}{latin}
Tunc temporis, licet multas expensas fecisset abbas ultra mare, non rediit vacua manu ad ecclesiam suam; ferens crucem auream et textum pretiosum ad pretium quaterviginti marcarum. Alia quoque vice cum redisset de ultra mare, sedens in capitulo dixit, quod si esset celerarius vel camerarius, aliquem questum faceret, qui su\ae{} administrationi competeret; et cum esset abbas, aliquid adquirere deberet quod abbatem deceret. Hoc cum dixit, obtulit conventui casulam pretiosam et mitram auro intextam, et sandalia cum caligis sericis, et cambucam virg\ae{} pastoralis argenteam et bene operatam. Simili modo, quotiens de ultra mare rediit aliquod ornamentum secum portavit.
\end{otherlanguage}

\switchcolumn

At that time, though the abbot incurred great expenses over sea, he did not return empty-handed to his church, but brought with him a golden cross and a precious copy of the scriptures, which was worth twenty-four marks. And on another occasion, when he had returned from across the sea, sitting in his chapter he said that had he been cellarer or chamberlain he would have sought for something which would be helpful in the administration of his office. And he added that as he was abbot he ought to acquire something which was fitting for an abbot, and saying this he offered the monastery a costly chasuble and a mitre inlaid with gold, and sandals with silken buskins, and the head of a pastoral staff, made of silver and well worked. In the same way, as often as he returned from beyond the sea, he brought some ornament with him.

\switchcolumn*

\begin{otherlanguage}{latin}
\blockhead[\textsc{a.d}.\ \oldstylenums{1197}.]{How the abbot took charge of the cellar, and how for that cause murmuring arose in the monastery.}{5}{-.6cm} 
Anno grati\ae{} \textsc{mº c}.\ nonagesimo septimo fact\ae{} sunt qu\ae{}dam innovationes et immutationes in ecclesia nostra, qu\ae{} sub silentio pr\ae{}teriri non debent. Cum non sufficerent celerario nostro antiqui redditus sui, jussit abbas S.\ ut quinquaginta libr\ae{} de Mildenhala darentur de incremento annuatim celerario per manum prioris; non simul, sed particulatim per menses, ut singulis mensibus aliquid haberetur ad expendendum, et non totum simul effunderetur in una parte anni: et ita factum est uno anno.

\end{otherlanguage}

\switchcolumn

In the year of grace one thousand one hundred and ninety-seven, certain changes and alterations were made 
in our church, which may not be passed over in silence. When our cellarer did not find his ancient revenues sufficient, abbot Samson ordered that fifty pounds should be given him in annual increase from Mildenhall by the hand of the prior. This was not to be paid at one time, but in instalments every month, that in each month there might be something to spend, and that the whole might not be used up in one part of the year; and so it was done for one year.

\switchcolumn*

\begin{otherlanguage}{latin}
Celerarius autem cum complicibus suis inde conquestus est, dicens, quod si illam pecuniam haberet pr\ae{} manibus, sibi provideret et se instauraret. Abbas vero petitioni su\ae{} cessit, licet invitus. Intrante mensi Augusti, jam expenderat celerarius totum, et insuper xx.\ sex libras debebat, et quinquaginta debiturus erat ante festum Sancti Michaelis.
\end{otherlanguage}

\switchcolumn

But the cellarer and his assistants complained of this, and he said that if he had had that money in his hands, he would have provided for himself and gathered stock for himself. Then the abbot, against his will indeed, granted that request. And when the beginning of August came, the cellarer had already spent the whole amount, and moreover owed twenty-six pounds, and was bound to pay a debt of fifty pounds before Michaelmas.

\switchcolumn*

\begin{otherlanguage}{latin}
Audiens hoc abbas moleste tulit, et ita locutus est in capitulo: ``S\ae{}pius comminatus sum, quod  ego capiam celerariam nostram in manu mea propter defectum et improvidentiam vestram, qui singulis debito magno vos obligatis. Clericum meum cum celerario vestro posui in testimonium, ut res consultius ageretur; sed non est clericus vel monachus qui audeat mihi dicere causam debiti. Dicitur tamen, quod inmoderata convivia in hospitio prioris, per assensum prioris et celerarii, et superflu\ae{} expens\ae{} in domo hospitum per incuriam hospitiarii, sint inde causa. Videtis,'' inquit, ``magnum debitum quod instat; dicite mihi consilium, quomodo res emendari debeat.''
\end{otherlanguage}

\switchcolumn

And when the abbot heard this, he was wroth, and spoke thus in the chapter, ``I have often threatened that I would take our cellar into my own hands owing to your incompetence and extravagance, since you bind yourselves with great debt. I placed my clerk with your cellarer as a witness, that the office might be managed with greater care. But there is no clerk or monk who dares tell me the cause of the debt. It is said, indeed, that the too elaborate feasts in the prior's house, which occur with the assent of the prior and of the cellarer, and the superfluous expense in the guest-house owing to the carelessness of the guest master, are the cause of it. You see,'' he went on, ``the great debt which is pressing on us; tell me your opinion as to the way in which the matter should be remedied.''

\switchcolumn*

\begin{otherlanguage}{latin}
Multi claustrales hoc audientes, et quasi subridentes, gratum habebant quod dicebatur; dicentes, occulte, quia hoc verum est quod abbas dixerat. Retorquebat prior culpam in celerarium; celerarius vero in hospitiarium; quilibet seipsum excusabat. Veritatem quidem sciebamus; sed tacebamus, quia timebamus. In crastino venit abbas, iterum dicens conventui: ``Dicite mihi consilium vestrum, quomodo celeraria vestra consultius et melius regi possit?'' Nec erat qui aliquid respondebat pr\ae{}ter unum qui dixit, nullam omnino superfluitatem esse in refectorio, unde debitum vel gravamen deberet oriri. Tertio die dixit abbas eadem verba; et respondit unus: ``Consilium istud a vobis debet emanare, tanquam a capite nostro.''
\end{otherlanguage}

\switchcolumn

Many of the cloistered monks, hearing this, and, as it were, laughing to themselves, were pleased with what was said, and said privately that what the abbot said was true. The prior cast the blame on the cellarer, and the cellarer on the guest-master, and the guest-master made excuse for himself. We, of course, knew the true reason, but were silent from fear. On the morrow the abbot came and again said to the monastery, ``Give me your advice as to how your cellar may be more thoughtfully and better managed.'' And there was no one who would answer a word, save one who said that there was no waste at all in the refectory whence any debt or burden could arise. And on the third day the abbot said the same words, and one answered, ``The advice ought to come from you, as from our head.''

\switchcolumn*

\begin{otherlanguage}{latin}
Et dixit abbas: ``Cum nec consilium vultis dicere, nec domum vestram scitis per vos regere, mihi incumbit tanquam patri et summo custodi dispositio monasterii. Accipio,'' inquid, ``in manu mea celerariam vestram et expensam hospitum, et procurationem interius et exterius.'' Hiis dictis, deposuit celerarium et hospitiarium, et alios duos monachos substituit habentes nomina subcelerarii et hospitiarii, clericum suum de mensa sua, magistrum G.\ eis associans, sine cujus assensu nichil in cibo vel potu, nec in expensis, nec in receptis ageretur. Antiqui emptores removebantur ab emptione in foro, et per clericum abbatis cibaria emebantur; et de bursa abbatis defectus nostri supplebantur. Hospites suscipiendi suscipiebantur et honorandi honorabantur; officiales et claustrales omnes pariter in refectorio reficiebantur, et undique superflu\ae{} expens\ae{} resecabantur.
\end{otherlanguage}

\switchcolumn

Then the abbot said, ``Since you will not give advice, and cannot rule your house for yourselves, the control of the monastery falls upon me as your father and chief guardian. I receive,'' he went on, ``into my own hand your cellar and the charge of the guests, and the task of getting supplies within and without.'' And with these words, he deposed the cellarer and guest-master, and replaced them with two monks, with the titles of sub-cellarer and guest-master, and associated with them a clerk of his table, master G., without whose assent nothing was to be done in the matter of food and drink, or in expenditure or in receipts. The former buyers were removed from the work of buying in the market, and food was to be purchased by a clerk of the abbot, and our deficits were to be made good from the abbot's treasury. Guests who ought to be received were received, and those who ought to be honoured were honoured. Officials and cloistered monks alike took their meals in the refectory, and on all sides superfluous expenses were cut down.

\switchcolumn*

\begin{otherlanguage}{latin}
Dixerunt autem quidam claustrales intra se: ``Septem, utique septem, fuerunt qui bona nostra comederunt, de quorum comestionibus, si quis loquebatur, quasi reus l\ae{}s\ae{} majestatis habebatur.'' Dicebat alius, tendens ad sidera palmas:\footnote[\textdagger]{\AE{}n.\ i.\ \oldstylenums{93}.} ``Benedictus Deus qui dedit talem voluntatem abbati, ut talia corrigat.'' Et dicebant plerique, quia bene est. 
\end{otherlanguage}

\switchcolumn

But some of the cloistered monks said among them selves, ``There were seven, yes, seven, who devoured our goods, and if one had spoken of their devouring, he would have been regarded as one guilty of high treason.'' Another said, as he stretched forth his hands to heaven, ``Blessed be God, Who hath given such a desire to the abbot, that he should correct so great faults.'' And many said that it was well done.

\switchcolumn*

\begin{otherlanguage}{latin}
Alii dicebant non, talem emendationem honoris depressionem \ae{}stimantes, et discretionem abbatis feritatem lupi appellantes; revocabant nempe ad memoriam antiqua somnia, scilicet quod futurus abbas s\ae{}viret ut lupus.
\end{otherlanguage}

\switchcolumn

Others said that it was not well done, thinking so great a reformation derogatory to the honour of the house, and calling the discretion of the abbot the ravening of a wolf; and in truth they called to mind old dreams, to the effect that he who should become abbot would raven as a wolf.

\switchcolumn*

\begin{otherlanguage}{latin}
Mirabantur milites, mirabatur populus super his qu\ae{} fiebant, et dicebat aliquis in plebe: ``Mirum est quod monachi, tot et literati viri, sustinent res et redditus confundi et commisceri cum rebus abbatis; qu\ae{} semper solebant distingui et ab invicem separari. Mirum est quod sibi non cavent de periculo futuro post mortem abbatis, si dominus rex invenerit eos in tali statu.''
\end{otherlanguage}

\switchcolumn

The knights were astonished, the people marvelled, at these things which had been done, and one of the common sort said, ``It is a strange thing that the monks, being so many and learned men, should allow their affairs and revenues to be confused and mingled with the affairs of the abbot, when they had always been wont to be separated and parted asunder. It is strange that they do not guard themselves against the danger which will come after the death of the abbot, if the lord king should find things in this state.''

\switchcolumn*

\begin{otherlanguage}{latin}
Dixit quidam alius, abbatem solum sapientem esse in rebus exterioribus regendis, et eum debere regere totum, qui scit regere totum. Et erat qui dicebat: ``Si saltem unus sapiens monachus esset in tanto conventu, qui domum sciret regere, abbas talia non fecisset.'' Et ita facti sumus subsannatio et illusio his qui in circuitu nostro sunt.\engnotetext{Ps.\ xliv., \oldstylenums{13}.}
\end{otherlanguage}

\switchcolumn

A certain man again said that the abbot was the only one who was skilled in external affairs, and that he ought to rule all, who knew how to rule all. And one there was who said, ``If there were but one wise monk in so great a monastery, who might know how to rule the house, the abbot would not have done such things.'' And so we became a scorn and derision to those who were round about us.\engnotenum{}

\switchcolumn*

\begin{otherlanguage}{latin}
In tali tempestate contigit diem anniversarium Roberti abbatis\footnote[\textdagger]{Abbot Robert died on the \oldstylenums{16}th September (see App.\ B.); but his anniversary seems to have been deferred to the \oldstylenums{28}th.}\engnotetext{Robert II.\ died on \oldstylenums{16}th September, \oldstylenums{1112} (\emph{Chron.\ Bur.\ Mem.}, III., \oldstylenums{5}). His festival, however, was kept on September \oldstylenums{28}th (Arnold, \emph{Mem}., I., \oldstylenums{291}, note). Rokewode (p.\ \oldstylenums{139}), quoting the \emph{Liber Albus}, gives September \oldstylenums{16}th as his day, and for those of Ording and Hugh, January \oldstylenums{30}th and November \oldstylenums{14}th respectively.} recitari in capitulo, et decretum fuit, ut Placebo et Dirige cantarentur sollempnius solito, scilicet cum magnis campanis pulsatis, sicut in anniversariis Ordingi et H.\ abbatum, propter nobile factum pr\ae{}dicti R.\ abbatis, qui distinxit res et redditus nostros a rebus et redditibus abbatis. Fiebat autem ista sollemnitas quorumdam consilio, ut saltem sic moveretur cor domini abbatis, scilicet ad bene faciendum. Erat autem aliquis qui putabat hoc fieri in confusionem abbatis, qui dicebatur velle confundere et commiscere res et redditus nostros et suos, eo quod saisiaverat celerariam nostram in manu sua.
\end{otherlanguage}

\switchcolumn

About this time it happened that the anniversary of abbot Robert\engnotenum{} was to be celebrated in the chapter, and it was decreed that a Placebo and a Dirige should be sung more solemnly than was wont, that is, with ringing of the great bells, as on the anniversaries of abbots Ording and Hugh. The cause of this was the noble deed of the said abbot Robert, who separated our goods and revenues from those of the abbot. But this solemnity was due to the counsel of some that so the heart of the lord abbot might be moved to do well. One there was, however, who thought that this was to be done to the shame of the abbot, who was accused of wishing to confound and intermingle his affairs and revenues and ours, in that he had taken our cellar into his own hand.

\switchcolumn*

\begin{otherlanguage}{latin}
Abbas vero audiens insolitum sonitum campanarum, et bene sciens et advertens hoc contra consuetudinem fieri, causam facti sapienter dissimulavit et missam sollemniter cantavit. Die vero sequente Sancti Michaelis, volens in parte compescere murmurationes quorumdam, illum qui prius fuit subcelerarius constituit celerarium, et quendam alium jussit nominari subcelerarium, remanente tamen pr\ae{}dicto clerico cum eis, et procurante omnia sicut prius. Cum autem clericus ille metas temperanti\ae{} excederet, dicens, ``Ego sum Bu.,'' id est, celerarius, cum metas temperanti\ae{} in bibendo excessisset, et, inconsulto abbate, curiam celerarii teneret et vadia et plegios caperet, et redditus annuos reciperet et per manum suam expenderet, summus celerarius publice dicebatur a populo. 
\end{otherlanguage}

\switchcolumn

Then when the abbot heard the unusual ringing of bells, and knew well and considered that this was contrary to custom, he wisely hid the cause of the action and sang mass solemnly. But on the following Michaelmas, since he wished to silence the murmurs of some men in part, he appointed him who had been sub-cellarer to the post of cellarer, and ordered another to be nominated as sub-cellarer, though the same clerk remained with them and procured all needful things as before. But when that clerk passed the bounds of moderation, saying, ``I am Bu,''---whereby he meant that the cellarer had passed the bounds of temperance in drinking,---and when, without consulting the abbot, he held the court of the cellarer and took sureties and pledges, and received the revenues for the year and spent them with his own hand, he was publicly called chief cellarer by the people.

\switchcolumn*

\begin{otherlanguage}{latin}
Cumque s\ae{}pius per curiam vagaretur et eum tanquam magistrum et summum procuratorem multi sequerentur pauperes et divites debitores, et calumniatores, diversi divers\ae{} conditionis et pro diversis negotiis, stetit forte aliquis ex obedientiariis nostris in curia, et h\ae{}c videns, pr\ae{} confusione et pudore lacrimatus est, cogitans hoc esse dedecus ecclesi\ae{} nostr\ae{}, cogitans consequens periculum, cogitans clericum monacho preferri in pr\ae{}judicium totius conventus. Procuravit ergo, quisquis fuit ille, per mediam personam, ut h\ae{}c domino abbati congrue et rationabiliter insinuarentur, factumque est ei intelligere quod hujusmodi arrogantia in clerico, qu\ae{} fiebat in dedecus et turpitudinem universitatis, posset esse causa magn\ae{} turbationis et dlscordi\ae{} in conventu. Abbas vero, cum h\ae{}c audisset, statim fecit mandari celerarium et pr\ae{}dictum clericum, jussitque ut celerarius de cetero se haberet ut celerarium in recipiendo denarios, in tenendo placita, et in omnibus aliis rebus, salvo tamen hoc, ut pr\ae{}dictus clericus ei assisteret, non a pari, sed ut testis et consiliarius.
\end{otherlanguage}

\switchcolumn

And when the clerk often wandered through the court, and many poor and rich debtors followed him as if he had been master and chief agent, as well as claimants of divers sorts and on divers matters, perchance one of our officials stood in the court. He saw this, and wept for shame and confusion, thinking that this was a shame to our church, and thinking of the danger which would result, and thinking that a clerk was preferred to a monk to the prejudice of the whole monastery. Accordingly he, whoever he was, procured by means of another, that this should be fitly and moderately pointed out to the lord abbot, and it came to pass that it was brought to the abbot's knowledge how arrogant the clerk was, and what he did to the shame and wrong of all; and that he was the cause of great disturbance and discord in the monastery. But when the abbot heard this, he at once ordered word to be sent to the cellarer and to the said clerk, and commanded that the cellarer should henceforth regard himself as cellarer in the receipt of money, and in holding pleas, and in all other matters, saving this only, that the said clerk should assist him, not on an equality, but as a witness and adviser.

\switchcolumn*

\begin{otherlanguage}{latin}
\blockhead{Concerning the will of Hamo Blund.}{3}{-.55cm} 
Hamo Blundus, unus ex ditioribus hominibus istius vill\ae{}, agens in extremis, vix aliquid testamentum voluit facere; tandem fecit testamentum ad pretium trium marcarum, nullo audiente nisi fratre suo et uxore sua et capellano. Quod cum abbas recognovisset post mortem ejus, illos tres convenit, et acriter corripuit super hoc, quod frater ejus, qui h\ae{}res erat, et uxor ejus non sustinuerunt aliquem alium accedere ad infirmum, cupientes omnia capere:

\end{otherlanguage}

\switchcolumn

Hamo Blund, one of the richest men of that town, being on his death-bed, was scarce willing to make any will. At length, however, he left three marks by will, no one hearing his wishes except his brother and his wife and a chaplain. And when the abbot held an inquest after his death, he summoned those three and sternly rebuked them because the brother, who was the heir, and the wife, would not permit any other to come to the sick man, desiring to gain all things.

\switchcolumn*

\begin{otherlanguage}{latin}
dixitque abbas in audientia: ``Ego fui episcopus ejus, et curam habui anim\ae{} ejus; ne mihi vertatur in periculum sacerdotis et confessoris ejus inscitia, quia infirmo viventi consulere non potui absens, quod mea interest faciam saltem tarde. Pr\ae{}cipio ut omnia debita ejus et katalla mobilia, qu\ae{} valent cc.\ marcas, sicut dicitur, reducantur in scriptum, et una portio detur h\ae{}redi, et alia uxori, et tertia pauperibus consanguineis suis et aliis pauperibus. Equum autem, qui ductus est ante feretrum defuncti et oblatus est Sancto \AE{}dmundo, jubeo remetti et reddi: non decet enim ecclesiam nostram coinquinari munere ejus, qui decessit intestatus, et quem fama accusat quod ex consuetudine solebat pecuniam suam ad usuram dare. Per os Dei, si alicui de c\ae{}tero tale quid contigerit diebus meis, non sepelietur in cimiterio.'' Hiis dictis, recesserunt c\ae{}teri confusi.
\end{otherlanguage}

\switchcolumn

And the abbot said in full audience, ``I was his bishop, and I had the care of his soul; that the ignorance of his priest and confessor may not be harmful, and because I was absent and could not advise the sick man while he yet lived, it is my concern that I should do so even when it is late. I command that all his money and moveable property, which is reported to be worth two hundred marks, shall be written down, and that one portion shall be given to the heir, and a second to his wife, and a third to his poor relations and to other poor men. But the horse, which was led before the coffin of the dead man, and which was offered to St.\ Edmund, I order to be sent back and returned. For it is not right that our church should be disgraced by the present of him who died intestate, and who is accused by report of having been wont often to lend his money upon usury. By the face of God, if what has now happened occur again in the case of any man in my time, he shall not be buried in the cemetery.'' And when he had so spoken, the others went away in confusion.

\switchcolumn*

\clearpage

\begin{otherlanguage}{latin}
\blockhead{How there were riots in the cemetery, and concerning that which was done in the matter.}{4}{-.2cm} 
In crastino Natalis Domini fiebant in cimiterio conventicula, colluctationes, et concertationes\engnotetext{Rokewode (p.\ \oldstylenums{139}) says that these shows were probably miracle-plays, and notes from the text that Samson did occasionally entertain minstrels (cp.\ text, p.\ \oldstylenums{67}.)} inter servientes abbatis et burgenses de villa; perventumque fuit a verbis ad verbera, a colaphis ad vulnera, et ad effusionem sanguinis. Abbas vero, h\ae{}c audiens, convocatis clanculo quibusdam, qui ad spectaculum convenerant, sed a longe steterant, nomina malefactorum jussit in scriptum redigi, quos omnes summoneri fecit, ut starent coram eo in crastino Sancti Thom\ae{} archiepiscopi in capella Sancti Dionisii responsuri; nec interim aliquem de burgensibus vocavit ad mensam suam, sicut prius solebat facere, primis quinque diebus in Natali. 

\end{otherlanguage}

\switchcolumn

On the morrow of the Nativity of the Lord, meetings and sports\engnotenum{} and bandying of words were held in the cemetery between the servants of the abbot and the burghers of the town; and from words it came to blows, from blows to wounds, and to bloodshed. But when the abbot knew of this, he privately called together some of those who had come to the spectacle, but who had taken no part in it, and commanded the names of all the offenders to be written down. And he caused all of them to be summoned to appear to make answer before him in the chapel of St.\ Denis on the morrow of the feast of St.\ Thomas the archbishop. And in the meantime he would not invite any of the burghers to his table, as he had been wont to do in the first five days of Christmastide.

\switchcolumn*

\begin{otherlanguage}{latin}
Die ergo statuto, acceptis juramentis a sexdecim legalibus hominibus, et auditis eorum attestationibus, dixit abbas: ``Constat quod isti malefactores inciderunt in canonem\footnote[\textdagger]{On Sacrilege, and the state of the law respecting it in the twelfth century, see Gratian's \emph{Decretum}, Pars II., causa \oldstylenums{17}, qu\ae{}st.\ \oldstylenums{4}.} dat\ae{}\footnote[\ddag]{\emph{late?} (Roke.). Mr.\ Gage Rokewode's correction is evidently right. Among the glosses to the passage quoted in the preceding note from Gratian, there is one which says, that if certain acts had been condemned beforehand by the bishop on pain of excommunication, ``tum dicerum quod statim excommunicatus est sacrilegus, non ratione sacrilegii, sed \emph{ratione sententi\ae{} dudum lat\ae{}}.''} sententi\ae{}; sed quia laici sunt hinc et inde, et non intelligunt quantum facinus sit tale sacrilegium facere, ut ceteri magis timeant, istos nominatim et publice excommunicabo, et, ne in aliqua parte derogetur justiti\ae{}, a domesticis et servientibus meis incipiam.''
\end{otherlanguage}

\switchcolumn


So on the appointed day, when oath had been taken from sixteen legal men, and their evidence had been heard, the abbot said, ``It is well known that those wrong-doers have broken the canon \emph{Latae sententiae}. But as there are laymen in both parties, and they know not how great a crime is such notable sacrilege, that the others might fear the more I will excommunicate them by name and publicly, and that justice may not be in any wise lacking, I will begin with my domestics and servants.''

\switchcolumn*

\begin{otherlanguage}{latin}
Sicque factum est, acceptis stolis et accensis candelis. Exierunt ergo omnes ab ecclesia, et, accepto consilio, omnes se exspoliaverunt, et omnino nudi pr\ae{}ter femoralia prostraverunt se ante ostium ecclesi\ae{}. Cumque venissent assessores abbatis, monachi et clerici, et dicerent lacrimabiliter, quod plusquam centum homines nudi ita jacerent, lacrimatus est et abbas. Rigorem tamen juris in verbo et vultu pr\ae{}ferens, et pietatem animi dissimulans, a consiliariis suis voluit cogi, ut p\oe{}nitentes absolverentur, sciens quod misericordia superexaltanda est judicio, et quod ecclesia omnes p\oe{}nitentes recipit. Verberati ergo omnes acriter et absoluti, juraverunt omnes, quod starent judicio ecclesi\ae{} de sacrilegio perpetrato. 
\end{otherlanguage}

\switchcolumn

Then this was done, when we had robed and lighted the candles. And all went out from the church, and having taken counsel, they all stripped themselves, and, naked except for their underclothes, prostrated themselves before the door of the church. And when the assessors of the abbot had come, monks and clerks, they told him in lamentable tones that more than a hundred men lay thus naked, and the abbot was grieved. Yet showing the rigour of the law in his word and face, and concealing the kindness of his heart, he wished to be compelled by his advisers to absolve the penitents, knowing that mercy should be exalted against judgment, and that the church should receive all penitents. Therefore all were heavily scourged and absolved, and all swore that they would abide by the judgment of the church for the sacrilege which had been done.

\switchcolumn*

\begin{otherlanguage}{latin}
In crastino vero data est eis p\oe{}nitentia secundum instituta canonum, et sic omnes ad unitatem concordi\ae{} revocavit abbas, minas terribiles omnibus proponens qui in dicto vel in facto materiam discordi\ae{} pr\ae{}berent. Conventicula autem et spectacula prohibuit  publice fieri in cimiterio; et sic, omnibus ad bonum pacis reductis, burgenses cum domino suo abbate diebus sequentibus comederunt cum gaudio magno.
\end{otherlanguage}

\switchcolumn

But next day a penance was assigned to them according to the provisions of the canons, and so the abbot received all in full peace, and threatened terrible things to all who in word or deed should produce cause for dispute. Then also he publicly forbade meetings and spectacles to be held in the cemetery, and so, when everything had been restored to peace, the burghers ate with the lord abbot on the days following with great joy.




\end{paracol}


\end{document}
